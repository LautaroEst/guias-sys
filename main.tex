\documentclass[12pt]{article}
\usepackage[utf8]{inputenc}
\usepackage[spanish,es-tabla]{babel}
\usepackage[autostyle=false,style=american]{csquotes}
\MakeOuterQuote{"}


% Matemática:
\usepackage{marvosym}
\usepackage{amsmath}
\usepackage{amsfonts}
\usepackage{amssymb}

% Gráficos e imágenes:
\usepackage{graphicx}
\usepackage{caption}
\usepackage{subcaption}

% Formato:
\usepackage[left=2.3cm,right=2.3cm,top=2cm,bottom=2cm]{geometry}

% Tikz:
\usepackage{tikz}
\usetikzlibrary{arrows,shapes,positioning,babel}
\usepackage{circuitikz}
\usepackage{pgfplots}
\pgfplotsset{compat=1.18}
\usepackage{array}
\usepackage{tabularray}
\allowdisplaybreaks

% Outline:
\usepackage[bookmarks]{hyperref}



%% COMANDOS %%

% Símbolos
\newcommand{\Keyboardsym}{\Keyboard \hspace{.3em}}
\newcommand{\Realpart}[1]{\mathfrak{Re}\left\{#1\right\}}
\newcommand{\Impart}[1]{\mathfrak{Im}\left\{#1\right\}}
\newcommand{\diracdelta}[2]{
    \draw[->,ultra thick] (axis cs:#1,0) -- (axis cs:#1,#2)
}
\newcommand{\discretedelta}[2]{
    \draw[-*,ultra thick] (axis cs:#1,0) -- (axis cs:#1,#2)
}

% Teoremas, ejercicios, contadores, etc
\newcounter{ejercicioi}
\newcommand{\ejercicios}{\setcounter{ejercicioi}{0}}
\newenvironment{ejercicio}{
    \incisos
    \vspace{0.5cm}
    \hrule
    \vspace{0.3cm}
    \setcounter{equation}{0}
    \refstepcounter{ejercicioi}
    \noindent
    {\bf \arabic{ejercicioi}.} 
}{}
\newcounter{incisoi}[ejercicioi]
\newcommand{\incisos}{\setcounter{incisoi}{0}}
\newcommand{\inciso}{
    \vspace{0.2cm}
    \refstepcounter{incisoi} \noindent (\textbf{\alph{incisoi}})~
}
\newcounter{subincisoi}[incisoi]
\newcommand{\subinciso}{
    \vspace{0.2cm}
    \refstepcounter{subincisoi} \noindent (\textbf{\roman{subincisoi}})~
}


% Title 
\newcommand{\myheader}[1]{
    \begin{center}
        \LARGE\textsc{86.05 - Señales y Sistemas} \\[.5em]
        \Large\textsc{#1}\\[2em]
    \end{center}
    \Keyboardsym = Python, Julia, MATLAB, OCTAVE, etc.
    \ejercicios
}

% Unidad temática
\newcommand{\unidadtem}{0}

\begin{document}

% Short-cuts
\newcommand{\R}{\mathbb{R}}
\newcommand{\Z}{\mathbb{Z}}


\ifcase%
    \unidadtem{} {
        \myheader{Guía 1: Señales en el tiempo}

\begin{ejercicio}
Graficar las siguientes señales en el intervalo $-10 \leq n \leq 10$. Implementar en \Keyboardsym una función para graficar cualquiera de ellas.

\inciso $-2\delta(n+2)$

\inciso $2^n u(n)$

\inciso $2^{-n} u(-n+2)$

\inciso $\cos\left(\frac{\pi}{3} n \right) u(n-2)$

\inciso $\sum_{k=-\infty}^{\infty} \delta(n-2k) - \delta(n-1-2k)$

\inciso $u(n+2) - u(n+2-N)$ con $N=3,5,7$


\end{ejercicio}

\begin{ejercicio}
Evaluar, si es posible, las siguientes sumas y expresar su respuesta en forma cartesiana y polar. Implementar una función en \Keyboardsym que obtenga la suma geométrica para cualquiera de ellas. 

\begin{align*}
\inciso & \sum_{n=0}^9 e^{j\frac{\pi}{2}n} &
\inciso & \sum_{n=-2}^7 e^{j\frac{\pi}{2}n} & \inciso & \sum_{n=0}^9 \cos\left(\frac{\pi}{2}n\right) & \hfill \\[.5em]
\inciso & \sum_{n=0}^9 \left(\frac{1}{2}\right)^n e^{j\frac{\pi}{2}n} & 
\inciso & \sum_{n=0}^\infty a^n e^{j\frac{\pi}{2}n},\; |a| < 1 & 
\inciso & \sum_{n=-4}^\infty \left(\frac{4}{3}\right)^n e^{j\frac{\pi}{2}n}
\end{align*}
\end{ejercicio}

\begin{ejercicio}
Sea $d_T(t)=t-T$ la función desplazamiento a derecha ($T>0$) y $e_a(t)=\frac{t}{a}$ la función expansión ($a > 1$). Para la señal $x(t)$

\begin{center}
    \begin{tikzpicture}[scale=0.6,transform shape]
    \begin{axis}[
    	axis y line=center,
    	axis x line=middle,
    	xlabel=$t$,ylabel={\Large $x(t)$},
    	xmin=-6,xmax=24,
    	ymin=-0.3,ymax=1.9,
    	xticklabel style = {xshift=5},
    	yticklabel style = {yshift=5},
    	]
    	\addplot[
    	black,
    	ultra thick
    	] coordinates {
    	    (0,0) (0,1) (10,1) (20,0)
    	};
    \end{axis}
\end{tikzpicture}

\end{center}

graficar:

\inciso $x(d_1(e_2(t)))$

\inciso $x(e_2(d_1(t)))$

\inciso $x\left(\frac{3}{2}t + 1\right)$

\inciso $x\left(-2t -1\right)$

\inciso $x\left(\frac{t}{2} - \frac{1}{2} \right)$

\end{ejercicio}

\begin{ejercicio}
Sea la señal discreta $x(n)$
\begin{center}
    \begin{tikzpicture}[scale=0.6,transform shape]
    \begin{axis}[
        x=0.04\textwidth,y=0.1\textwidth,
    	axis y line=center,
    	axis x line=middle,
    	xlabel=$n$,ylabel={\LARGE $x(n)$},
    	xmin=-6.9,xmax=6.9,
    	ymin=-0.3,ymax=3.5,
    	xticklabel style = {xshift=0},
    	yticklabel style = {yshift=5},
    	]
    	\discretedelta{-6}{0.1};
    	\discretedelta{-5}{0.1};
    	\discretedelta{-4}{0.1};
    	\discretedelta{-3}{0.1};
    	\discretedelta{-2}{1};
    	\discretedelta{-1}{2};
    	\discretedelta{0}{3};
    	\discretedelta{1}{2};
    	\discretedelta{2}{2};
    	\discretedelta{3}{1};
    	\discretedelta{4}{0.1};
    	\discretedelta{5}{0.1};
    	\discretedelta{6}{0.1};
    \end{axis}
\end{tikzpicture}
\end{center}

Graficar cada una de las siguientes señales. Implementar en \Keyboardsym una función que grafique cada una de ellas a partir de una $x(n)$ genérica.

\inciso $x(n) u(2-n)$

\inciso $x(n-1) \delta(n-3)$

\inciso $x(n-1) \delta(n-3)$

\inciso $\frac{1}{2}x(n) + \frac{1}{2}(-1)^n x(n)$

\end{ejercicio}

\begin{ejercicio}
Graficar las partes par e impar de las siguientes señales. Implementar en \Keyboardsym una función que obtenga la parte par e impar de una señal cualquiera. 

\begin{align*}
\inciso \parbox{.3\textwidth}{\begin{tikzpicture}[scale=0.6,transform shape]
    \begin{axis}[
        x=0.12\textwidth,y=0.12\textwidth,
    	axis y line=center,
    	axis x line=middle,
    	xlabel=$t$,ylabel={\Large $x(t)$},
    	xmin=-2.3,xmax=2.3,
    	ymin=-.2,ymax=1.3,
    	xticklabel style = {xshift=5},
    	yticklabel style = {yshift=5},
    	]
    	\addplot[
    	black,
    	ultra thick
    	] coordinates {
    	    (-2,0) (0,0) (1,1) (2,0)
    	} ;
    \end{axis}
\end{tikzpicture}} & \hfill & \inciso \parbox{.3\textwidth}{\begin{tikzpicture}[scale=0.6,transform shape]
    \begin{axis}[
        x=0.12\textwidth,y=0.12\textwidth,
    	axis y line=center,
    	axis x line=middle,
    	xlabel=$t$,ylabel={\Large $x(t)$},
    	xmin=-2.3,xmax=2.3,
    	ymin=-.2,ymax=1.3,
    	xticklabel style = {xshift=5},
    	yticklabel style = {yshift=5},
    	]
    	\addplot[
    	black,
    	ultra thick
    	] coordinates {
    	    (-2,0) (-1,1) (0,0) (1,1) 
    	    (2,1) (2,0)
    	} ;
    \end{axis}
\end{tikzpicture}}
\end{align*}
\end{ejercicio}

\begin{ejercicio}
Sea $x(t)$ una señal de tiempo continuo, y sean $y_1(t) = x(2 t)$ y $y_2(t) = x(t / 2)$. Determinar si las siguientes afirmaciones son verdaderas o falsas y justificar con demostración o contraejemplo según corresponda. Obtener el período fundamental de $y_1(t)$ y $y_2(t)$.

\inciso Si $x(t)$ es periódica, entonces $y_1(t)$ es periódica.

\inciso Si $y_1(t)$ es periódica, entonces $x(t)$ es periódica.

\inciso Si $x(t)$ es periódica, entonces $y_2(t)$ es periódica.

\inciso Si $y_2(t)$ es periódica, entonces $x(t)$ es periódica.

\end{ejercicio}

\begin{ejercicio}
Sea $x(n)$ una señal de tiempo discreta, y sean $y_1(n) = x(2 n)$ y 
\begin{equation*}
y_2(n) = \begin{cases}
x(n/2) & \mbox{si $n$ es par} \\
0 & \mbox{en otro caso}
\end{cases}
\end{equation*}
Determinar si las siguientes afirmaciones son verdaderas o falsas y justificar con demostración o contraejemplo según corresponda. Obtener el período fundamental de $y_1(n)$ y $y_2(n)$.

\inciso Si $x(n)$ es periódica, entonces $y_1(n)$ es periódica.

\inciso Si $y_1(n)$ es periódica, entonces $x(n)$ es periódica.

\inciso Si $x(n)$ es periódica, entonces $y_2(n)$ es periódica.

\inciso Si $y_2(n)$ es periódica, entonces $x(n)$ es periódica.

\end{ejercicio}

\begin{ejercicio}
Implementar en \Keyboardsym \emph{sin utilizar bucles} una función que, dada una señal $x(n)$ y un $N_0\in \mathbb{Z}$, obtenga $y_1(n)=x(N_0n)$ y 
\begin{equation*}
y_2(n) = \begin{cases}
x(n/N_0) & \mbox{si $n$ es m\`{u}ltiplo de $N_0$ } \\
0 & \mbox{en otro caso}
\end{cases}
\end{equation*}
\end{ejercicio}

\begin{ejercicio}
Sea $x(t)=e^{j\omega_0t}$ la señal exponencial compleja en tiempo continuo con período fundamental $T_0=\frac{2\pi}{\omega_0}$. Considere la señal discreta obtenida al tomar muestras de $x(t)$ equiespaciadas:
\begin{equation*}
    x_d(n) = x(nT_s) = e^{j\omega_0t}\big|_{t=nT_s} = e^{j\omega_0nT_s}
\end{equation*}

\inciso Demostrar que $x_d(n)$ es periódica si y sólo si $\frac{T_0}{T_s}$ es un número racional.

\inciso Determinar el período y la frecuencia fundamental de $x_d(n)$ cuando ésta es periódica. Expresar la frecuencia fundamental como una fracción de $T_0$.

\inciso ¿Cuántos periodos de $T_0$ se necesitan para obtener las muestras que forman un solo período de $x_d(n)$ en el caso en que ésta última es periódica?
\end{ejercicio}

\begin{ejercicio}
Indicar si las siguientes afirmaciones son verdaderas o falsas:

\inciso La suma de dos señales senoidales de tiempo continuo de frecuencias $f_1$ y $f_2$ es siempre una señal periódica. 

\inciso La suma de dos señales senoidales de tiempo discreto de frecuencias $f_1$ y $f_2$ es siempre una señal periódica. 

\end{ejercicio}

\begin{ejercicio}
Graficar en \Keyboardsym las siguientes funciones y determinar si son periódicas o no:
\begin{align*}
    \inciso & \cos\left(\frac{2\pi}{12} n\right) & \inciso & \cos\left(\frac{8\pi}{31} n\right) \\ 
    \inciso & \cos\left(\frac{1}{6} n\right) & \inciso & \parbox{.5\textwidth}{$x(n)$ definida como las muestras de una senoidal de frecuencia 10Hz muestreada con $T_s=\frac{1}{1000\mathrm{Hz}}$} \\
    \inciso & x(n) = \sum_{k=0}^{50}\cos\left(\frac{\pi k}{32}n\right) & \inciso & x(n) = \sum_{k=0}^{50} \Realpart{a_ke^{\frac{\pi k}{32}n}}\; \mathrm{con}\; a_k = \frac{1}{\pi k} \sin\left(\frac{\pi}{4} k\right)  \\ 
    \inciso & x(n) = \sum_{k=0}^{50}\cos\left(\frac{\pi f(k)}{2}n\right)\; \mathrm{con} & & \hspace{-2.5em} f(k) = 10 \tan\left(\frac{3\pi}{400} k\right)
\end{align*}
\end{ejercicio}

        \newpage
        \myheader{Guía 2: Caracterización de un Sistema}

\begin{ejercicio}
    Para el sistema en tiempo continuo definido por 
    \begin{align*}
        \inciso y(t) = e^{-t} x(t) & \hfill & \inciso y(t) = 2x(t) + \cos(2t)
    \end{align*}
    se pide:

    \subinciso Determinar si el sistema es lineal.

    \subinciso Encontrar la respuesta del sistema a $x(t)=\delta(t-t_0)$ para cualquier $t_0\in \mathbb{R}$. ¿El sistema es LTI?
    
    \subinciso Decidir si es posible obtener cualquier respuesta del sistema a partir de la respuesta a $x(t)=\delta(t-t_0)$.
\end{ejercicio}
    
\begin{ejercicio}
    Dar, si es posible, un ejemplo de un sistema en tiempo continuo en donde conocer la respuesta a $x(n)=\delta(n-n_0)$ para todo $n_0\in \Z$ no sea suficiente para encontrar la salida del sistema para cualquier $x(n)$.
\end{ejercicio}
    
\begin{ejercicio}
    Para los siguientes sistemas de tiempo continuo
    \begin{align*}
        \inciso y(t) = x(t/3) & \hfill & \inciso y(t) = \int_{-\infty}^{t} x(\tau + 2) d\tau & \hfill & \inciso y(t) = 2x(t) + 1
    \end{align*}
    determinar si cada uno de ellos es lineal, invariante ante desplazamientos, estable, causal y si tiene memoria.
\end{ejercicio}
    
\begin{ejercicio}
    Para los siguientes sistemas de tiempo discreto
    \begin{align*}
        \inciso & y(n) = x(-n) & \hfill \inciso & y(n) = \begin{cases}
        1 & \mbox{para $n = 0$} \\
        \pi x(n) & \mbox{en otro caso}
        \end{cases} \\[.5em]
        \inciso & y(n) = n x(n) & \hfill \inciso & y(n) = \Realpart{x(n)}
    \end{align*}
    determinar si cada uno de ellos es lineal, invariante ante desplazamientos, estable, causal y si tiene memoria.
\end{ejercicio}
    
\begin{ejercicio}
    Calcular $x(n) * x(n)$ con $x(n)=u(n+3) - u(n-4)$.
\end{ejercicio}
    
\begin{ejercicio}
    Dado un sistema LTI cuya respuesta impulsiva está dada por un pulso triangular $h(n)=\delta(n+2)+2\delta(n+1)+3\delta(t)+2\delta(t-1)+\delta(t-2)$ al cual se le aplica una entrada $x(n)$ que consiste en un tren de impulsos de período $N$, calcular y graficar la salida $y(n)$ para los siguientes casos:
    \begin{align*}
    \inciso N=6 & \hfill & \inciso N=4 & \hfill & \inciso N=2
    \end{align*}
\end{ejercicio}
    
\begin{ejercicio}
    A un sistema LTI cuya respuesta al impulso es $h(t) = u(t) - u(t - 1)$ se le aplica una entrada $x(t) = h(t/\alpha)$.
    
    \inciso Calcular la salida del sistema
    
    \inciso Si se sabe que la derivada de la salida tiene sólo 3 discontinuidades, obtener el valor de $\alpha$.
\end{ejercicio}

\begin{ejercicio}
    Dado un sistema LTI en tiempo discreto con una respuesta al impulso $h(n) = \alpha^n u(n)$ con $\alpha<1$, encontrar la salida del sistema para cada una de las siguientes entradas:
    
    \inciso $x(n) = \delta(n) - \delta(n-1)$
    
    \inciso $x(n) = u(n) - u(n-5)$
\end{ejercicio}
    
\begin{ejercicio}
    Dado un sistema LTI en tiempo continuo, encontrar la salida del sistema cuando la entrada es $x(t) = u(t-1)\sin(t)$ si la respuesta al impulso es:
    
    \inciso $h(t) = u(n)$ 
    
    \inciso $h(t) = u(t) - 2u(t-2) + u(t-5)$
\end{ejercicio}

\begin{ejercicio}
    Sobre un único sistema invariante en el tiempo se conocen los siguientes pares entrada-salida:
    \begin{align*}
        x_1(n) = \delta(n) + 2\delta(n-2) & \hspace{1em}\longrightarrow\hspace{1em} y_1(n) = \delta(n-1) + 2\delta(n-2) \\[.5em]
        x_2(n) = 3 \delta(n-2) & \hspace{1em}\longrightarrow\hspace{1em} y_2(n) = \delta(n-1) + 2 \delta(n-3) \\[.5em]
        x_3(n) = \delta(n-3) & \hspace{1em}\longrightarrow\hspace{1em} y_3(n) = \delta(n+1) + 2 \delta(n) + \delta(n-1)
    \end{align*}
    
    \inciso ¿Se puede afirmar algo sobre la linearidad del sistema?
    
    \inciso ¿Es posible hallar la respuesta del sistema $y_4(n)$ cuando la entrada es $x_4(n)=\delta(n)$ con los datos disponibles? En tal caso, obtenerla.
    
    \inciso ¿Es posible hallar la respuesta del sistema $y_5(n)$ cuando la entrada es $x_5(n)=5\delta(n-2)$ con los datos disponibles? En tal caso, obtenerla.
\end{ejercicio}

\begin{ejercicio}
    Determinar si los siguientes enunciados son verdaderos o falsos.
    
    \inciso La conexión en cascada de sistemas LTI resulta en un sistema total que también es LTI.
    
    \inciso La conexión en cascada de sistemas no lineales es un sistema no lineal.
    
    \inciso La conexión en cascada de sistemas no invariantes en el tiempo es un sistema no invariante en el tiempo.
    
    \inciso La conexión en cascada de sistemas causales con sistemas causales es siempre no causal.
    
    \inciso El orden de conexión de sistemas no invariantes en el tiempo no altera la salida para una misma entrada.
    
    \inciso En un sistema LTI si la entrada es periódica entonces la salida también lo es.
\end{ejercicio}
    
\begin{ejercicio}
    Dos sistemas LTI en tiempo discreto con respuesta al impulso $h_1(n)$ y $h_2(n)$ son conectados en cascada en ese orden. La entrada no se conoce pero la salida $y(n)$ es como se muestra en la siguiente figura:
    \begin{center}
        \begin{tikzpicture}[scale=0.6,transform shape]
    \begin{axis}[
        x=0.1\textwidth,y=0.1\textwidth,
    	axis y line=center,
    	axis x line=middle,
    	xlabel=$n$,ylabel={\LARGE $x(n)$},
    	xmin=-7.5,xmax=7.5,
    	ymin=-1.3,ymax=2.9,
    	xticklabel style = {xshift=0},
    	yticklabel style = {yshift=5},
    	]
    	\discretedelta{-7}{0.1};
    	\discretedelta{-6}{0.1};
    	\discretedelta{-5}{0.1};
    	\discretedelta{-4}{0.1};
    	\discretedelta{-3}{0.1};
    	\discretedelta{-2}{0.1};
    	\discretedelta{-1}{1};
    	\discretedelta{0}{2};
    	\discretedelta{1}{-1};
    	\discretedelta{2}{1};
    	\discretedelta{3}{0.1};
    	\discretedelta{4}{0.1};
    	\discretedelta{5}{0.1};
    	\discretedelta{6}{0.1};
    	\discretedelta{7}{0.1};
    \end{axis}
\end{tikzpicture}
    \end{center}
    
    \inciso Si los dos sistemas son causales, ¿qué se puede decir acerca del momento en que la entrada podría haber empezado? ¿Se puede establecer el momento exacto de comienzo?
    
    \inciso La entrada $x(n)$ que produjo la salida $y(n)$ anterior es aplicada a un nuevo par de sistemas conectados en cascada donde el primero tiene una respuesta impulsiva $h_a(n) = h_1(n + 1)$ y el segundo $h_b(n) = 2h_2(n)$. Graficar la salida.
\end{ejercicio}
    
\begin{ejercicio}
    En el sistema de la figura se sabe que $\gamma \in \Z_{\geq 0}$ y $\beta, \alpha_1, \alpha_2 \in \R$ con $\alpha_1 \neq \alpha_2$.
    \begin{center}
        \begin{tikzpicture}
    \node[circle,draw,thick,inner sep=0.05cm] (plus) at (0,0) {\Large +} ;
    \node[rectangle,draw,thick,inner sep=0.4cm] (h1) at (-4,1) {$h_1(n)=\alpha_1^n u(n)$} ;
    \node[rectangle,draw,thick,inner sep=0.4cm] (h2) at (-4,-1) {$h_2(n)=\beta \delta(n-\gamma)$} ;
    \node[rectangle,draw,thick,inner sep=0.4cm] (h3) at (4,0) {$h_3(n)=\alpha^n u(n)$} ;
    \node[xshift=-4cm] (x_n) at (h1) {$x(n)$} ;
    \node[xshift=-3cm,inner sep=-0.1] (x_n_arrow_cross) at (h1) {} ;

    \node[xshift=3cm,yshift=.3cm] (y_1) at (h1) {$y_1(n)$} ;
    \node[xshift=3cm,yshift=.3cm] (y_2) at (h2) {$y_2(n)$} ;
    \node[xshift=-2.5cm,yshift=-.4cm] (y_3) at (h3) {$y_3(n)$} ;
    \node[xshift=3cm,yshift=.3cm] (y_n) at (h3) {$y(n)$} ;

    \draw[->, very thick] (x_n) -- (h1.west) ;
    \draw[->, very thick] (x_n_arrow_cross) |- (h2.west) ;
    \draw[->, very thick] (h1.east) -| (plus.north) ;
    \draw[->, very thick] (h2.east) -| (plus.south) ;
    \draw[->, very thick] (plus.east) -- (h3.west) ;
    \draw[->, very thick] (h3.east) -- ++(2,0) ;
\end{tikzpicture}
    \end{center}

    \inciso Determinar $h(n)$ tal que $y(n) = h(n) * x(n)$.

    \inciso Determinar las condiciones que deben cumplir los parámetros $\alpha_1, \alpha_2, \beta, \gamma$ para que el sistema sea estable.
\end{ejercicio}

\begin{ejercicio}
    Dado un sistema LTI con $h(t) = u(t + T/2) - u(t - T/2)$ donde $T>0$, se desea analizar cómo el sistema opera sobre señales periódicas.
    
    \inciso Demostrar que para cualquier señal periódica de período $T$ la salida es constante para todo $t$.

    \inciso Determinar el valor de la constante si se sabe que la señal periódica es impar.

    \inciso Determinar si existen señales periódicas de período $T'\neq T$ para las cuales los resultados anteriores siguen siendo ciertos.
\end{ejercicio}

\begin{ejercicio}
    Sea un sistema LTI y una señal $x(n)$ que satisface $x(n)=\delta(n) + \sum_{k=1}^M a_k x(n-k)$ donde $a_k$ son valores reales. Sea $y(n)$ la salida del sistema para la entrada $x(n)$.

    \inciso Determinar la ecuación en diferencias de $h(n)$ en función de $y(n)$ usando las propiedades básicas de los sitemas LTI.
    
    \inciso Si $y(n)$ es de duración finita, ¿qué se puede decir sobre la estabilidad del sistema?

    \inciso Si $y(n)$ es de duración infinita, obtener algún tipo de condición sobre $y(n)$ que asegure la estabilidad del sistema.
\end{ejercicio}

\begin{ejercicio}
    Sea un sistema LTI de tiempo discreto tal que la relación entre la salida y la entrada del mismo se puede escribir como:
    \begin{equation*}
        y(n) = \sum_{k=0}^{\infty} \alpha^{n-k} \left(x(k) + \beta x(k-1) + \gamma x(k-2)\right)
    \end{equation*}
    donde $|\alpha| < 1$ y $\beta, \gamma \in \R$.

    \inciso Obtener la respuesta al impulso del sistema y graficarla.
    \inciso Analizar la estabilidad del mismo.
    \inciso Determinar los valores de $\beta$ y $\gamma$ para que el sistema tenga las siguientes propiedades:
    \begin{itemize}
        \item Si $x(n) = (-1)^n$ para todo $n$ entonces $y(n) = 0$ para todo $n$.
        \item Si $x(n) = 1$ para todo $n$ entonces $y(n) = 1$ para todo $n$.
    \end{itemize}
\end{ejercicio}

\begin{ejercicio}
    Dado un sistema en tiempo discreto definido por la ecuación en diferencias 
    \begin{equation*}
        y(n) = x(n) + \frac{3}{4} y(n-1)
    \end{equation*}
    con condiciones de contorno
    \begin{align*}
    \inciso & y(-1) = 0 & \inciso & y(1) = 0 & \inciso & y(0) = 0 \\[.5em]
    \inciso & y(0) = 1 & \inciso & \lim_{n\rightarrow -\infty} y(n) = 0  & \inciso & \lim_{n\rightarrow +\infty} y(n) = 0
    \end{align*}
    se pide:
    \begin{itemize}
        \item Calcular el valor de la respuesta al impulso $h(n)$ del sistema en el intervalo $-5 \leq n \leq 5$
        \item Obtener una expresión cerrada de $h(n)$ para todo $n \in \mathbb{Z}$.
        \item Determinar si el sistema lineales, invariante ante desplazamientos, estables y causales.
    \end{itemize}
\end{ejercicio}
    
\begin{ejercicio}

    Dado un sistema en tiempo discreto definido por la ecuación en diferencias 
    \begin{equation*}
        y(n) = x(n) + \frac{1}{2} y(n-1)
    \end{equation*}
    con condiciones
    
    \inciso iniciales de reposo

    \inciso finales de reposo

    \noindent se pide:
    \begin{itemize}
        \item Determinar si el sistema lineales, invariante ante desplazamientos, estables y causales. Establecer cómo se relacionan las condiciones de contorno con cada una de las propiedades del sistema.
        \item Calcular el valor de la respuesta a las señales $\delta(n-1)$ y $\delta(n+1)$ del sistema en el intervalo $-5 \leq n \leq 5$.
        \item Obtener una expresión cerrada de la respuesta a las señales $\delta(n-1)$ y $\delta(n+1)$ para todo $n \in \mathbb{Z}$.
    \end{itemize}
\end{ejercicio}
    
\begin{ejercicio}
    Demostrar que un sistema definido por ecuaciones en diferencias FIR siempre será lineal, invariante ante desplazamientos y estable. Determinar, además, qué condición debe cumplir un sistema de este tipo para ser causal.
\end{ejercicio}
    
\begin{ejercicio}
    Obtener la respuesta al impulso para el sistema definido por la ecuación diferencial 
    \begin{equation*}
        y(n) = x(n+1) + x(n) + x(n-1) + \frac{3}{4} y(n-1)
    \end{equation*}
    en condiciones inciales de reposo. 
\end{ejercicio}
    
\begin{ejercicio}
    Obtener la respuesta al impulso para el sistema definido por la ecuación diferencial 
    \begin{equation*}
        y(n) = x(n+1) + x(n) + x(n-1) + \frac{5}{4} y(n-1)
    \end{equation*}
    en condiciones finales de reposo. 
\end{ejercicio}
    
\begin{ejercicio}
    Implementar en \Keyboardsym una función que permita obtener la respuesta al impulso de una ecuación en diferencias para condiciones iniciales o finales de reposo.
\end{ejercicio}
    
    
    
        \newpage
        \myheader{Guía 3: Series de Fourier}


\begin{ejercicio}
Sea $x(t)$ un tren de pulsos de período $T$ definido en el intervalo $[-\frac{T}{2},\frac{T}{2})$ como 
\begin{equation*}
    x(t) = \begin{cases}
    1 & \mbox{si } t \in (-\frac{T_0}{2},\frac{T_0}{2}) \\
    0 & \mbox{si } t \in [-\frac{T}{2},-\frac{T_0}{2}) \cup (\frac{T_0}{2},\frac{T}{2})
    \end{cases}
\end{equation*}
\end{ejercicio}
con $T > T_0 > 0$. 

\inciso Obtenga los coeficientes $\left\{a_k\right\}_{k\in\mathbb{Z}}$ de la Serie de Fourier de tiempo continuo de $x(t)$ y escriba los 6 primeros términos de la serie.

\inciso Grafique el espectro de aplitudes y de fase.

\inciso Obtenga los coeficientes $\left\{b_k\right\}_{k\in\mathbb{Z}}$ de la Serie de Fourier de tiempo continuo de $x_1(t) = x(t-\frac{\pi}{6})$.

\inciso Obtenga los coeficientes $\left\{c_k\right\}_{k\in\mathbb{Z}}$ de la Serie de Fourier de tiempo continuo de $x_2(t) = x(t)+1$.

\inciso Obtenga los coeficientes $\left\{d_k\right\}_{k\in\mathbb{Z}}$ de la Serie de Fourier de tiempo continuo de $x_3(t) = \sum_{n=-\infty}^{\infty} (-1)^{n-1} \delta(t-(2n+1)\frac{T_0}{2})$.

\begin{ejercicio}
Utilizando las propiedades de la serie de Fourier de tiempo continuo, calcular los coeficientes $\left\{b_k\right\}_{k\in\mathbb{Z}}$ de la señal periódica $x(t)$:
\begin{align*}
    \inciso & \parbox{.4\textwidth}{\begin{tikzpicture}[scale=0.6]
    \begin{axis}[
        x=.05\textwidth,y=0.08\textwidth,
    	axis y line=center,
    	axis x line=middle,
    	xlabel=$t$,ylabel={\Large $x(t)$},
    	xmin=-6.9,xmax=6.9,
    	ymin=-1.2,ymax=1.5,
    	xtick={-6,...,6},
    	xticklabel style = {xshift=0},
    	yticklabel style = {yshift=5}
	] 
	\addplot[
    	black,
    	ultra thick
    	] coordinates {
	(-5,-1) (-3,1) (-3,-1) (-1,1) (-1,-1) (1,1) (1,-1) (3,1) (3,-1) (5,1)
	} ;
	\node at (axis cs:6.5,0.5) {\Large $\cdots$} ;
	\node at (axis cs:-5.9,0.5) {\Large $\cdots$} ;
    \end{axis}
\end{tikzpicture}} & \hspace{\fill} 
    \inciso & \parbox{.4\textwidth}{\begin{tikzpicture}[scale=0.6]
    \begin{axis}[
        x=.05\textwidth,y=0.1\textwidth,
    	axis y line=center,
    	axis x line=middle,
    	xlabel=$t$,ylabel={\Large $x(t)$},
    	xmin=-6.9,xmax=6.9,
    	ymin=-0.2,ymax=1.4,
    	xtick={-6,...,6},
    	xticklabel style = {xshift=0},
    	yticklabel style = {yshift=5}
	] 
	\addplot[
    	black,
    	ultra thick
    	] coordinates {
	(-5, 1) (-4, 0) (-2,0) (-1,1) (1,1) (2,0) (4,0) (5,1)
	} ;
	\node at (axis cs:6.5,0.5) {\Large $\cdots$} ;
	\node at (axis cs:-5.9,0.5) {\Large $\cdots$} ;
    \end{axis}
\end{tikzpicture}} \\
    \inciso & \parbox{.4\textwidth}{\begin{tikzpicture}[scale=0.6]
    \begin{axis}[
        x=.05\textwidth,y=0.06\textwidth,
    	axis y line=center,
    	axis x line=middle,
    	xlabel=$t$,ylabel={\Large $x(t)$},
    	xmin=-6.9,xmax=6.9,
    	ymin=-0.2,ymax=2.5,
    	xtick={-6,...,6},
    	xticklabel style = {xshift=0},
    	yticklabel style = {yshift=5}
	] 
	\addplot[
    	black,
    	ultra thick
    	] coordinates {
	(-5,0) (-3,2) (-2,0) (0,2) (1,0) (3,2)
	(4,0) (5,1)
	} ;
	\node at (axis cs:6.5,1) {\Large $\cdots$} ;
	\node at (axis cs:-5.9,1) {\Large $\cdots$} ;
    \end{axis}
\end{tikzpicture}} & \hspace{\fill} 
    \inciso & \parbox{.4\textwidth}{\begin{tikzpicture}[scale=0.6]
    \begin{axis}[
        x=.05\textwidth,y=0.06\textwidth,
    	axis y line=center,
    	axis x line=middle,
    	xlabel=$t$,ylabel={\Large $x(t)$},
    	xmin=-6.9,xmax=6.9,
    	ymin=-2.2,ymax=1.9,
    	xtick={-6,...,6},
    	xticklabel style = {xshift=0},
    	yticklabel style = {yshift=5}
	] 
	\diracdelta{-5}{-2};
	\diracdelta{-4}{1};
	\diracdelta{-3}{-2};
	\diracdelta{-2}{1};
	\diracdelta{-1}{-2};
	\diracdelta{0}{1};
	\diracdelta{1}{-2};
	\diracdelta{2}{1};
	\diracdelta{3}{-2};
	\diracdelta{4}{1};
	\diracdelta{5}{-2};
	\node at (axis cs:6.5,.5) {\Large $\cdots$} ;
	\node at (axis cs:-5.9,.5) {\Large $\cdots$} ;
    \end{axis}
\end{tikzpicture}} \\
    \inciso & \parbox{.4\textwidth}{\begin{tikzpicture}[scale=0.6]
    \begin{axis}[
        x=.05\textwidth,y=0.06\textwidth,
    	axis y line=center,
    	axis x line=middle,
    	xlabel=$t$,ylabel={\Large $x(t)$},
    	xmin=-6.9,xmax=6.9,
    	ymin=-1.4,ymax=1.4,
    	xtick={-6,...,6},
    	xticklabel style = {xshift=0},
    	yticklabel style = {yshift=5}
	] 
	\addplot[
    	black,
    	ultra thick
    	] coordinates {
	(-5,0) (-5,-1) (-4,-1) (-4,0) (-2,0) (-2,1) (-1,1) (-1,0) (1,0) (1,-1) (2,-1) (2,0) (4,0) (4,1) (5,1) (5,0)
	} ;
	\node at (axis cs:6.5,0.5) {\Large $\cdots$} ;
	\node at (axis cs:-5.9,0.5) {\Large $\cdots$} ;
    \end{axis}
\end{tikzpicture}} & \hspace{\fill} 
    \inciso & \parbox{.4\textwidth}{\begin{tikzpicture}[scale=0.6]
    \begin{axis}[
        x=.05\textwidth,y=0.06\textwidth,
    	axis y line=center,
    	axis x line=middle,
    	xlabel=$t$,ylabel={\Large $x(t)$},
    	xmin=-6.9,xmax=6.9,
    	ymin=-0.2,ymax=2.9,
    	xtick={-6,...,6},
    	xticklabel style = {xshift=0},
    	yticklabel style = {yshift=5}
	] 
	\addplot[
    	black,
    	ultra thick
    	] coordinates {
	(-5,1) (-4,1) (-4,0) (-3,0) (-3,2) (-2,2) (-2,1) (-1,1) (-1,0) (0,0) (0,2) (1,2) (1,1) (2,1) (2,0) (3,0) (3,2) (4,2) (4,1) (5,1) (5,0)
	} ;
	\node at (axis cs:6.5,1) {\Large $\cdots$} ;
	\node at (axis cs:-5.9,1) {\Large $\cdots$} ;
    \end{axis}
\end{tikzpicture}} \\
    \inciso & \parbox{.4\textwidth}{\begin{tikzpicture}[scale=0.6,transform shape]
    \begin{axis}[
        x=0.036\textwidth,y=0.12\textwidth,
    	axis y line=center,
    	axis x line=middle,
    	xlabel=$t$,ylabel={\Large $x(t)$},
    	xmin=-21,xmax=21,
    	ymin=-0.4,ymax=1.4,
    	xticklabel style = {xshift=5},
    	yticklabel style = {yshift=5},
    	]
    	\addplot[
    	black,
    	ultra thick
    	] coordinates {
    	    (-15,0) (-12,1) 
    	    (-9,0) (-6,1)
    	    (-3,0) (0,1)
    	    (3,0) (6,1)
    	    (9,0) (12,1)
    	    (15,0)
    	};
    	\node at (18,0.5) {\Huge $\cdots$} ;
    	\node at (-18,0.5) {\Huge $\cdots$} ;
    \end{axis}
\end{tikzpicture}} & \hspace{\fill} & \\
\end{align*}
\end{ejercicio}

\begin{ejercicio}
Calcular los coeficientes de la serie de Fourier de tiempo continuo de la señal $x(t)$ de período $T=3$ sabiendo que en el intervalo $[-1.5,1.5)$ se define como

\inciso $x(t) = \cos(20\pi t)w_1(t)$ con \begin{equation*}
    w_1(t) = \begin{cases}
    1 & \mbox{si } t \in (-1,1) \\
    0 & \mbox{si } t \in [-1.5,\,-1) \cup (1,1.5)
    \end{cases}
\end{equation*}

\inciso $x(t) = \cos(20\pi t)w_2(t)$ con \begin{equation*}
    w_2(t) = \begin{cases}
    1 & \mbox{si } t \in (-1.5,-1) \\
    0 & \mbox{si } t \in [-1,0) \\
    2 & \mbox{si } t \in (0,1) \\
    1 & \mbox{si } t \in (1,1.5) \\
    \end{cases}
\end{equation*}

\inciso $x(t) = \cos(20\pi t)w_3(t)$ con \begin{equation*}
    w_3(t) = \begin{cases}
    e^{-|t|} & \mbox{si } t \in (-1,1) \\
    0 & \mbox{si } t \in [-1.5,-1) \cup (1,1.5)
    \end{cases}
\end{equation*}
\end{ejercicio}


\begin{ejercicio}
Utilizando las propiedades de la serie de Fourier de tiempo discreto, calcular los coeficientes $\left\{a_k\right\}_{k=0}^{N}$ de la señal periódica $x(n)$:
\begin{align*}
    \inciso & \parbox{.4\textwidth}{\begin{tikzpicture}[scale=0.6,transform shape]
    \begin{axis}[
        x=0.035\textwidth,y=0.1\textwidth,
    	axis y line=center,
    	axis x line=middle,
    	xlabel=$n$,ylabel={\LARGE $x(n)$},
    	xmin=-9.9,xmax=9.9,
    	ymin=-0.3,ymax=1.9,
    	xticklabel style = {xshift=0},
    	yticklabel style = {yshift=5},
    	]
    	\discretedelta{-8}{0.1};
    	\discretedelta{-7}{1};
    	\discretedelta{-6}{1};
    	\discretedelta{-5}{1};
    	\discretedelta{-4}{1};
    	\discretedelta{-3}{1};
    	\discretedelta{-2}{0.1};
    	\discretedelta{-1}{0.1};
    	\discretedelta{0}{1};
    	\discretedelta{1}{1};
    	\discretedelta{2}{1};
    	\discretedelta{3}{1};
    	\discretedelta{4}{1};
    	\discretedelta{5}{0.1};
    	\discretedelta{6}{0.1};
    	\discretedelta{7}{1};
    	\discretedelta{8}{1};
    	\node at (-9,0.5) {\Large $\cdots$};
    	\node at (9,0.5) {\Large $\cdots$};
    \end{axis}
\end{tikzpicture}} & \hspace{\fill} 
    \inciso & \parbox{.4\textwidth}{\begin{tikzpicture}[scale=0.6,transform shape]
    \begin{axis}[
        x=0.035\textwidth,y=0.1\textwidth,
    	axis y line=center,
    	axis x line=middle,
    	xlabel=$n$,ylabel={\LARGE $x(n)$},
    	xmin=-9.9,xmax=9.9,
    	ymin=-0.3,ymax=1.9,
    	xticklabel style = {xshift=0},
    	yticklabel style = {yshift=5},
    	]
    	\discretedelta{-8}{0.1};
    	\discretedelta{-7}{0.1};
    	\discretedelta{-6}{1};
    	\discretedelta{-5}{1};
    	\discretedelta{-4}{1};
    	\discretedelta{-3}{1};
    	\discretedelta{-2}{0.1};
    	\discretedelta{-1}{0.1};
    	\discretedelta{0}{1};
    	\discretedelta{1}{1};
    	\discretedelta{2}{1};
    	\discretedelta{3}{1};
    	\discretedelta{4}{0.1};
    	\discretedelta{5}{0.1};
    	\discretedelta{6}{1};
    	\discretedelta{7}{1};
    	\discretedelta{8}{1};
    	\node at (-9,0.5) {\Large $\cdots$};
    	\node at (9,0.5) {\Large $\cdots$};
    \end{axis}
\end{tikzpicture}} \\
    \inciso & \parbox{.4\textwidth}{\begin{tikzpicture}[scale=0.6,transform shape]
    \begin{axis}[
        x=0.035\textwidth,y=0.08\textwidth,
    	axis y line=center,
    	axis x line=middle,
    	xlabel=$n$,ylabel={\LARGE $x(n)$},
    	xmin=-9.9,xmax=9.9,
    	ymin=-1.4,ymax=2.9,
    	xticklabel style = {xshift=0},
    	yticklabel style = {yshift=5},
    	]
    	\discretedelta{-8}{-1};
    	\discretedelta{-7}{2};
    	\discretedelta{-6}{1};
    	\discretedelta{-5}{2};
    	\discretedelta{-4}{-1};
    	\discretedelta{-3}{0.1};
    	\discretedelta{-2}{-1};
    	\discretedelta{-1}{2};
    	\discretedelta{0}{1};
    	\discretedelta{1}{2};
    	\discretedelta{2}{-1};
    	\discretedelta{3}{0.1};
    	\discretedelta{4}{-1};
    	\discretedelta{5}{2};
    	\discretedelta{6}{1};
    	\discretedelta{7}{2};
    	\discretedelta{8}{-1};
    	\node at (-9,0.5) {\Large $\cdots$};
    	\node at (9,0.5) {\Large $\cdots$};
    \end{axis}
\end{tikzpicture}} & \hspace{\fill} 
    \inciso & x(n) = \sin(2\pi n / 3) \cos(\pi n/2)
\end{align*}

\noindent \hspace*{0.6em} \inciso $x(n)$ periódica con período 4, siendo $x(n) = 1-\sin(\pi n/4)$ para $0 \leq n \leq 3$ 

\noindent \hspace*{0.6em} \inciso $x(n)$ periódica con período 12, siendo $x(n) = 1-\sin(\pi n/4)$ para $0 \leq n \leq 11$
\end{ejercicio}

\begin{ejercicio}
Sea $x(n)$ una señal periódica de período $N$, hallar:
\begin{align*}
    & \inciso x(n-n_0),\; n_0 \in\mathbb{Z} & \hfill & \inciso x(n) - x(n-1) & \hfill & \inciso x(n) - x\left(n-\frac{N}{2}\right),\; \mbox{$N$ par} \\
    & \inciso x(n) + x\left(n-\frac{N}{2}\right),\; \mbox{$N$ par} & \hfill & \inciso x^*(-n) & \hfill & \inciso (-1)^n x(n),\; \mbox{$N$ par} \\ & \inciso (-1)^n x(n),\; \mbox{$N$ impar}
    & \hfill & \inciso y(n) = \begin{cases} x(n) & \mbox{si $n$ es par} \\ 0 & \mbox{en otro caso}
    \end{cases}
\end{align*}
\end{ejercicio}

\begin{ejercicio}
Sea un sistema LTI de tiempo continuo con respuesta al impulso $h(t) = e^{-4|t|}$. Hallar la salida $y(t)$ del sistema para las siguientes entradas:

\inciso $x(t)=e^{j\omega_0 t}$ con $\omega_0\in\mathbb{R}$ ¿Es $x(t)$ una autofunción del sistema?

\inciso $x(t)=e^{j\omega_0 t}u(t)$ con $\omega_0\in\mathbb{R}$. ¿Es $x(t)$ una autofunción del sistema?

\inciso $x(t)=\sum_{k=-\infty}^{\infty} (-1)^k \delta(t-k)$

\inciso $x(t)$ de período 1 tal que $x(t) = \begin{cases} 1 & \mbox{si $-1/4 \leq t < 1/4$} \\ 0 & \mbox{en otro caso} \end{cases}$ \hspace{1em} para $-1/2\leq t < 1/2$.
\end{ejercicio}

\begin{ejercicio}
Para cada uno de los siguientes pares de señales $x(n)$ e $y(n)$, determinar si existe un sistema LTI discreto cuya salida $y(n)$ pueda corresponder efectivamente a la entrada $x(n)$. En el caso de que exista, determinar si es único y qué codiciones debe cumplir la respuesta al impulso.

\inciso $x(n) = \left(\frac{1}{2}\right)^n$ e $y(n) = \left(\frac{1}{4}\right)^n$

\inciso $x(n)=e^{jn/8}$ e $y(n) = e^{j2n/8}$

\inciso $x(n)= e^{j\pi n/3}$ e $y(n) = \cos(\pi n/3)$

\inciso $x(n) = \cos(\pi n/3)$ e $y(n) = \cos(\pi n/3) + \sqrt{3} \sin(\pi n/3)$
\end{ejercicio}

\begin{ejercicio}
Decidir si existe un sistema LTI discreto que cumpla con que si $x(n)$ es la entrada, entonces $y(n)$ es una salida posible.
\begin{align*}
    \inciso & \parbox{.4\textwidth}{
        \begin{tikzpicture}[scale=0.6,transform shape]
    \begin{axis}[
        x=0.022\textwidth,y=0.1\textwidth,
    	axis y line=center,
    	axis x line=middle,
    	xlabel=$n$,ylabel={\LARGE $x(n)$},
    	xmin=-15.9,xmax=15.9,
    	ymin=-0.3,ymax=1.9,
    	xticklabel style = {xshift=0},
    	yticklabel style = {yshift=5}
    	]
    	\discretedelta{-14}{0.1};
    	\discretedelta{-13}{1};
    	\discretedelta{-12}{1};
    	\discretedelta{-11}{1};
    	\discretedelta{-10}{0.1};
    	\discretedelta{-9}{0.1};
    	\discretedelta{-8}{0.1};
    	\discretedelta{-7}{0.1};
    	\discretedelta{-6}{0.1};
    	\discretedelta{-5}{0.1};
    	\discretedelta{-4}{0.1};
    	\discretedelta{-3}{0.1};
    	\discretedelta{-2}{0.1};
    	\discretedelta{-1}{1};
    	\discretedelta{0}{1};
    	\discretedelta{1}{1};
    	\discretedelta{2}{0.1};
    	\discretedelta{3}{0.1};
    	\discretedelta{4}{0.1};
    	\discretedelta{5}{0.1};
    	\discretedelta{6}{0.1};
    	\discretedelta{7}{0.1};
    	\discretedelta{8}{0.1};
    	\discretedelta{9}{0.1};
    	\discretedelta{10}{0.1};
    	\discretedelta{11}{1};
    	\discretedelta{12}{1};
    	\discretedelta{13}{1};
    	\discretedelta{14}{0.1};
    	\node at (-15,0.5) {\Large $\cdots$};
    	\node at (15,0.5) {\Large $\cdots$};
    \end{axis}
\end{tikzpicture}
    } 
    & \hspace{\fill} 
    & \parbox{.4\textwidth}{
        \begin{tikzpicture}[scale=0.6,transform shape]
    \begin{axis}[
        x=0.022\textwidth,y=0.1\textwidth,
    	axis y line=center,
    	axis x line=middle,
    	xlabel=$n$,ylabel={\LARGE $y(n)$},
    	xmin=-15.9,xmax=15.9,
    	ymin=-0.3,ymax=1.9,
    	xticklabel style = {xshift=0},
    	yticklabel style = {yshift=5}
    	]
    	\discretedelta{-14}{1};
    	\discretedelta{-13}{0.1};
    	\discretedelta{-12}{0.1};
    	\discretedelta{-11}{0.1};
    	\discretedelta{-10}{1};
    	\discretedelta{-9}{1};
    	\discretedelta{-8}{1};
    	\discretedelta{-7}{0.1};
    	\discretedelta{-6}{0.1};
    	\discretedelta{-5}{0.1};
    	\discretedelta{-4}{1};
    	\discretedelta{-3}{1};
    	\discretedelta{-2}{1};
    	\discretedelta{-1}{0.1};
    	\discretedelta{0}{0.1};
    	\discretedelta{1}{0.1};
    	\discretedelta{2}{1};
    	\discretedelta{3}{1};
    	\discretedelta{4}{1};
    	\discretedelta{5}{0.1};
    	\discretedelta{6}{0.1};
    	\discretedelta{7}{0.1};
    	\discretedelta{8}{1};
    	\discretedelta{9}{1};
    	\discretedelta{10}{1};
    	\discretedelta{11}{0.1};
    	\discretedelta{12}{0.1};
    	\discretedelta{13}{0.1};
    	\discretedelta{14}{1};
    	\node at (-16,0.5) {\Large $\cdots$};
    	\node at (16,0.5) {\Large $\cdots$};
    \end{axis}
\end{tikzpicture}
    } \\
    \inciso & \parbox{.4\textwidth}{
        \begin{tikzpicture}[scale=0.6,transform shape]
    \begin{axis}[
        x=0.022\textwidth,y=0.1\textwidth,
    	axis y line=center,
    	axis x line=middle,
    	xlabel=$n$,ylabel={\LARGE $x(n)$},
    	xmin=-15.9,xmax=15.9,
    	ymin=-0.3,ymax=1.9,
    	xticklabel style = {xshift=0},
    	yticklabel style = {yshift=5}
    	]
    	\discretedelta{-14}{0.1};
    	\discretedelta{-13}{1};
    	\discretedelta{-12}{1};
    	\discretedelta{-11}{1};
    	\discretedelta{-10}{0.1};
    	\discretedelta{-9}{0.1};
    	\discretedelta{-8}{0.1};
    	\discretedelta{-7}{0.1};
    	\discretedelta{-6}{0.1};
    	\discretedelta{-5}{0.1};
    	\discretedelta{-4}{0.1};
    	\discretedelta{-3}{0.1};
    	\discretedelta{-2}{0.1};
    	\discretedelta{-1}{1};
    	\discretedelta{0}{1};
    	\discretedelta{1}{1};
    	\discretedelta{2}{0.1};
    	\discretedelta{3}{0.1};
    	\discretedelta{4}{0.1};
    	\discretedelta{5}{0.1};
    	\discretedelta{6}{0.1};
    	\discretedelta{7}{0.1};
    	\discretedelta{8}{0.1};
    	\discretedelta{9}{0.1};
    	\discretedelta{10}{0.1};
    	\discretedelta{11}{1};
    	\discretedelta{12}{1};
    	\discretedelta{13}{1};
    	\discretedelta{14}{0.1};
    	\node at (-15,0.5) {\Large $\cdots$};
    	\node at (15,0.5) {\Large $\cdots$};
    \end{axis}
\end{tikzpicture}
    } & \hspace{\fill} 
    & \parbox{.4\textwidth}{
        \begin{tikzpicture}[scale=0.6,transform shape]
    \begin{axis}[
        x=0.022\textwidth,y=0.1\textwidth,
    	axis y line=center,
    	axis x line=middle,
    	xlabel=$n$,ylabel={\LARGE $y(n)$},
    	xmin=-15.9,xmax=15.9,
    	ymin=-0.3,ymax=1.9,
    	xticklabel style = {xshift=0},
    	yticklabel style = {yshift=5}
    	]
    	\discretedelta{-14}{0.1};
    	\discretedelta{-13}{0.1};
    	\discretedelta{-12}{0.1};
    	\discretedelta{-11}{0.1};
    	\discretedelta{-10}{1};
    	\discretedelta{-9}{1};
    	\discretedelta{-8}{1};
    	\discretedelta{-7}{0.1};
    	\discretedelta{-6}{0.1};
    	\discretedelta{-5}{0.1};
    	\discretedelta{-4}{0.1};
    	\discretedelta{-3}{0.1};
    	\discretedelta{-2}{0.1};
    	\discretedelta{-1}{1};
    	\discretedelta{0}{1};
    	\discretedelta{1}{1};
    	\discretedelta{2}{0.1};
    	\discretedelta{3}{0.1};
    	\discretedelta{4}{0.1};
    	\discretedelta{5}{0.1};
    	\discretedelta{6}{0.1};
    	\discretedelta{7}{0.1};
    	\discretedelta{8}{1};
    	\discretedelta{9}{1};
    	\discretedelta{10}{1};
    	\discretedelta{11}{0.1};
    	\discretedelta{12}{0.1};
    	\discretedelta{13}{0.1};
    	\discretedelta{14}{0.1};
    	\node at (-15,0.5) {\Large $\cdots$};
    	\node at (15,0.5) {\Large $\cdots$};
    \end{axis}
\end{tikzpicture}
    }
\end{align*}
En caso de que se cumpla, decidir si es posible asegurar que todos los sistemas que cumplan con esta condición son realmente LTI.
\end{ejercicio}
        \newpage
        \myheader{Guía 4: Transformada de Fourier}


\begin{ejercicio}
    Hallar, utilizando propiedades, la transformada de Fourier de las siguientes señales:
    
    \begin{align*}
        \inciso & x(t) = 
        \begin{cases} 
            1 + \cos(\pi t) & |t| \leq 1 \\
            0 & |t| > 1 
        \end{cases} & \hspace{\fill} 
        \inciso & x(t) = \frac{\sin(\pi t)}{\pi t} \frac{\sin(2\pi (t-1))}{\pi (t-1)} \\
        \inciso & x(t) = 
        \begin{cases} 
            e^{-t} & 0 \leq t < 1 \\
            0 & \mbox{en otro caso}
        \end{cases} & \hspace{\fill} 
        \inciso & \parbox{.3\textwidth}{\begin{tikzpicture}[scale=0.6,transform shape]
    \begin{axis}[
    	axis y line=center,
    	axis x line=middle,
    	xlabel=$t$,ylabel=$x(t)$,
    	xmin=-1.9,xmax=3.9,
    	ymin=-1.9,ymax=2.9,
    	xticklabel style = {xshift=5},
    	yticklabel style = {yshift=5},
    	]
    	\addplot[
    	black,
    	ultra thick
    	] coordinates {
    	    (-1,0) (-1,1) (0,1) (0,2)
    	    (1,2) (1,-1) (3,-1) (3,0)
    	};
    \end{axis}
\end{tikzpicture}
} \\
        \inciso & \parbox{.3\textwidth}{\begin{tikzpicture}[scale=0.6,transform shape]
    \begin{axis}[
    	axis y line=center,
    	axis x line=middle,
    	xlabel=$t$,ylabel=$x(t)$,
    	xmin=-2.9,xmax=2.9,
    	ymin=-0.9,ymax=2.9,
    	xticklabel style = {xshift=5},
    	yticklabel style = {yshift=5},
    	]
    	\addplot[
    	black,
    	ultra thick
    	] coordinates {
    	    (-2,0) (0,2) (2,0)
    	};
    \end{axis}
\end{tikzpicture}} & \hspace{\fill} 
        \inciso & \parbox{.3\textwidth}{\begin{tikzpicture}[scale=0.6,transform shape]
    \begin{axis}[
    	axis y line=center,
    	axis x line=middle,
    	xlabel=$t$,ylabel=$x(t)$,
    	xmin=-2.9,xmax=2.9,
    	ymin=-1.9,ymax=1.9,
    	xticklabel style = {xshift=5},
    	yticklabel style = {yshift=5},
    	]
    	\addplot[
    	black,
    	ultra thick
    	] coordinates {
    	    (-2,0) (-2,-1) (-1,-1) (1,1) 
    	    (2,1) (2,0)
    	} ;
    \end{axis}
\end{tikzpicture}} \\
        \inciso & \parbox{.3\textwidth}{\begin{tikzpicture}[scale=0.6]
    \begin{axis}[
        x=.06\textwidth,y=0.1\textwidth,
    	axis y line=center,
    	axis x line=middle,
    	xlabel=$t$,ylabel=$x(t)$,
    	xmin=-7.9,xmax=7.5,
    	ymin=-0.3,ymax=2.3,
    	xticklabel style = {xshift=0},
    	yticklabel style = {yshift=5}
	]
	\diracdelta{-6}{2};
	\diracdelta{-5}{1};
	\diracdelta{-4}{2};
	\diracdelta{-3}{1};
	\diracdelta{-2}{2};
	\diracdelta{-1}{1};
	\diracdelta{0}{2};
	\diracdelta{1}{1};
	\diracdelta{2}{2};
	\diracdelta{3}{1};
	\diracdelta{4}{2};
	\diracdelta{5}{1};
	\node at (axis cs:6.5,1) {\Large $\cdots$} ;
	\node at (axis cs:-7,1) {\Large $\cdots$} ;
    \end{axis}
\end{tikzpicture}} & \hspace{\fill} & \\
    \end{align*}
    \end{ejercicio}
    
    \begin{ejercicio}
    Sea $x(t)$ la siguiente función:
    \begin{center}
        \begin{tikzpicture}[scale=0.6,transform shape]
    \begin{axis}[
        x=0.03\textwidth,y=0.2\textwidth,
    	axis y line=center,
    	axis x line=middle,
    	xlabel=$t$,ylabel=$x(t)$,
    	xmin=-21,xmax=21,
    	ymin=-0.4,ymax=1.4,
    	xticklabel style = {xshift=5},
    	yticklabel style = {yshift=5},
    	]
    	\addplot[
    	black,
    	ultra thick
    	] coordinates {
    	    (-15,0) (-12,1) 
    	    (-9,0) (-6,1)
    	    (-3,0) (0,1)
    	    (3,0) (6,1)
    	    (9,0) (12,1)
    	    (15,0)
    	};
    	\node at (18,0.5) {\Large $\cdots$} ;
    	\node at (-18,0.5) {\Large $\cdots$} ;
    \end{axis}
\end{tikzpicture}
    \end{center}
    
    \inciso Calcular la transformada de $x(t)$ utilizando la transformada $y(t) = \frac{\sin(\pi t)}{\pi t} \frac{\sin(2\pi (t-1))}{\pi (t-1)}$ calculada en el punto anterior. 
    
    \inciso ¿Qué relación existe entre la transformada de Fourier de una señal periódica y los coeficientes de su serie de Fourier? 
    
    \inciso ¿Qué característica distintiva tiene una Transformada de Fourier de una señal periódica?
    
    \end{ejercicio}
    
    
    \begin{ejercicio}
    Hallar, utilizando propiedades, la antitransformada de Fourier de las siguientes funciones:
    \begin{align*}
        \inciso & X(\omega) = \cos(4\omega + \pi/3) 
        \hspace*{10em} \inciso \hspace*{0.1em} X(\omega) = \frac{2\sin(3 (\omega-2\pi))}{\omega - 2\pi} \\
        \inciso & 
        X(\omega) \; \mathrm{tal\, que} \; |X(\omega)| = \begin{cases}
            |\omega| & \mbox{si } \omega \in [-1, 1) \\ 
            0 & \mathrm{en\, otro\, caso}
        \end{cases}
        \hspace*{1em} \mathrm{y} \hspace*{1em} \arg(X(\omega)) = -3\omega
    \end{align*}
    \end{ejercicio}
    
    \begin{ejercicio}
    Sea $X(\omega)$ la antitransformada de $x(t)$:
    \begin{center}
        \begin{tikzpicture}[scale=0.6,transform shape]
    \begin{axis}[
        x=0.1\textwidth,y=0.1\textwidth,
    	axis y line=center,
    	axis x line=middle,
    	xlabel={\Large $t$},ylabel={\Large $x(t)$},
    	xmin=-1.9,xmax=3.9,
    	ymin=-0.2,ymax=2.5,
    	xtick distance=1,
    	ytick distance=1,
    	xticklabel style = {xshift=5},
    	yticklabel style = {yshift=5},
    	]
    	\addplot[
    	black,
    	ultra thick
    	] coordinates {
    	    (-1,0) (-1,2) (0,2) (1,1)
    	    (2,2) (3,2) (3,0)
    	};
    \end{axis}
\end{tikzpicture}

    \end{center}
    
    Hallar los siguientes valores sin obtener en forma explícita la función $X(\omega)$:
    \begin{align*}
        \inciso & \angle X(\omega) & \inciso & X(0) & \inciso & \int_{-\infty}^{\infty} X(\omega) d\omega \\[.5em]
        \inciso & \int_{-\infty}^{\infty} X(\omega) \frac{2\sin(\omega)}{\omega} e^{j2\omega} d\omega & \inciso & \int_{-\infty}^{\infty} |X(\omega)|^2 d\omega
    \end{align*}
    
    \end{ejercicio}
    
    \begin{ejercicio}
    Sea $x(t) = e^{-t} (u(t) - u(t-1))$. Graficar las siguientes funciones y hallar la transformada de Fourier de todas ellas:
    \begin{align*}
        \inciso & x_1 = x(-t) + x(t) & 
        \inciso & x_2 = -x(-t) + x(t) &
        \inciso & x_3 = x(t+1) + x(t) &
        \inciso & x_4 = t x(t)
    \end{align*}
\end{ejercicio}
    
\begin{ejercicio}
    Para cada una de las siguientes funciones
    \begin{align*}
        \inciso & \parbox{.5\textwidth}{\begin{tikzpicture}[scale=0.6,transform shape]
    \begin{axis}[
        x=0.04\textwidth,y=0.1\textwidth,
    	axis y line=center,
    	axis x line=middle,
    	xlabel=$t$,ylabel={\LARGE $x(t)$},
    	xmin=-9,xmax=10,
    	ymin=-1.9,ymax=1.9,
    	xticklabel style = {xshift=5},
    	yticklabel style = {yshift=5},
    	]
    	\addplot[
    	black,
    	ultra thick
    	] coordinates {
    	    (-7,1) (-5,-1) 
    	    (-3,1) (-1,-1)
    	    (1,1) (3,-1)
    	    (5,1) (7,-1)
    	};
    	\node at (8,0.5) {\Huge $\cdots$} ;
    	\node at (-8,0.5) {\Huge $\cdots$} ;
    \end{axis}
\end{tikzpicture}} & \hspace{\fill} &
        \inciso & \parbox{.3\textwidth}{\begin{tikzpicture}[scale=0.6,transform shape]
    \begin{axis}[
    	axis y line=center,
    	axis x line=middle,
    	xlabel=$t$,ylabel=$x(t)$,
    	xmin=-2.9,xmax=2.9,
    	ymin=-0.9,ymax=2.9,
    	xticklabel style = {xshift=5},
    	yticklabel style = {yshift=5},
    	]
    	\diracdelta{1}{2};
    \end{axis}
\end{tikzpicture}} \\
        \inciso & \parbox{.5\textwidth}{\begin{tikzpicture}[scale=0.6,transform shape]
    \begin{axis}[
        x=0.08\textwidth,y=0.1\textwidth,
    	axis y line=center,
    	axis x line=middle,
    	xlabel=$t$,ylabel={\Large $x(t)$},
    	xmin=-0.9,xmax=8.9,
    	ymin=-1.3,ymax=1.3,
    	xticklabel style = {xshift=5},
    	yticklabel style = {yshift=5},
    	]
    	\addplot [
    	black, ultra thick,
    	domain=2:8, smooth
    	] {sin(deg(2*pi*x))} ;
    	\addplot[
    	black, ultra thick
    	] coordinates {(-1,0) (2,0)} ;
    	\addplot[
    	black, ultra thick
    	] coordinates {(8,0) (8.5,0)} ;
    \end{axis}
\end{tikzpicture}} & \hspace{\fill} &
        \inciso & \parbox{.3\textwidth}{\begin{tikzpicture}[scale=0.6,transform shape]
    \begin{axis}[
    	axis y line=center,
    	axis x line=middle,
    	xlabel=$t$,ylabel={\Large $x(t)$},
    	xmin=-2.9,xmax=2.9,
    	ymin=-1.9,ymax=1.9,
    	xticklabel style = {xshift=5},
    	yticklabel style = {yshift=5},
    	]
    	\addplot[
    	black,
    	ultra thick
    	] coordinates {(-2,0) (-2,-1) (2,1) (2,0)} ;
    \end{axis}
\end{tikzpicture}} \\
        \inciso & \parbox{.5\textwidth}{\pgfmathdeclarefunction{ejcuatrotresa}{1}{%
  \pgfmathparse{#1^2 * exp(-abs(#1))}%
}

\begin{tikzpicture}[scale=0.6,transform shape]
    \begin{axis}[
        x=0.04\textwidth,y=0.2\textwidth,
    	axis y line=center,
    	axis x line=middle,
    	xlabel=$t$,ylabel={\Large $x(t)=t^2 e^{-|t|}$},
    	xmin=-8.9,xmax=8.9,
    	ymin=-.3,ymax=1.3,
    	xticklabel style = {xshift=5},
    	yticklabel style = {yshift=5},
    	samples=100
    	]
    	\addplot [
    	black, ultra thick,
    	domain=-10:10, smooth
    	] {ejcuatrotresa(x)} ;
    \end{axis}
\end{tikzpicture}} & \hspace{\fill} &
        \inciso & \parbox{.3\textwidth}{\pgfmathdeclarefunction{gauss}{3}{%
  \pgfmathparse{exp(-((#1-#2)^2)/(2*#3^2))}%
}

\begin{tikzpicture}[scale=0.6,transform shape]
    \begin{axis}[
        x=0.04\textwidth,y=0.2\textwidth,
    	axis y line=center,
    	axis x line=middle,
    	xlabel=$t$,ylabel={\Large $x(t)=e^{-t^2/2}$},
    	xmin=-4.9,xmax=4.9,
    	ymin=-.3,ymax=1.3,
    	xticklabel style = {xshift=5},
    	yticklabel style = {yshift=5},
    	]
    	\addplot [
    	black, ultra thick,
    	domain=-10:10, smooth, samples=100
    	] {gauss(x,0,1)} ;
    \end{axis}
\end{tikzpicture}} \\
    \end{align*}
    indicar si cumplen algunas de estas condiciones:
    \begin{align*}
        \subinciso & \Realpart{X(\omega)} = 0 &
        \subinciso & \Impart{X(\omega)} = 0 &
        \subinciso & \exists \alpha \in \mathbb{R} \; \mbox{tal que } e^{j\omega\alpha}X(\omega)\; \mbox{es una función real} \\
        \subinciso & \int_{-\infty}^{\infty} X(\omega) d\omega = 0 &
        \subinciso & \int_{-\infty}^{\infty} \omega X(\omega) d\omega = 0 &
        \subinciso & X(\omega)\; \mbox{es periódica}
    \end{align*}
\end{ejercicio}
    
\begin{ejercicio}
    Hallar y graficar la transformada de Fourier de tiempo discreto de las siguientes secuencias:
    \begin{align*}
        \inciso & x(n) = u(n) - u(n-20) &
        \inciso & x(n) = \delta(n) - \delta(n-1) \\
        \inciso & x(n) = \left( \frac{1}{2} \right)^{-n} u(-n-1) &
        \inciso & x(n) = \sin\left(\frac{\pi}{2} n \right) + \cos(n) 
    \end{align*}
\end{ejercicio}
        
\begin{ejercicio}
    Hallar la antitransformada de Fourier en tiempo discreto de las siguientes funciones:
    
    \inciso $X(e^{j\Omega}) = 1 + 3e^{-j2\Omega} - 4e^{-j10\Omega}$
    
    \inciso $X(e^{j\Omega}) = \sum_{k=-\infty}^{\infty} (-1)^k \delta(\Omega - \frac{\pi}{2}k)$
    
    \inciso $X(e^{j\Omega}) = \frac{e^{-j\Omega} - 1/5}{1-1/5 e^{-j\Omega}}$
    \end{ejercicio}
    
    \begin{ejercicio}
    Sea $X(e^{j\Omega})$ la antitransformada de Fourier de la secuencia $x(n)$:
    
    \begin{center}
    \parbox{.7\textwidth}{\begin{tikzpicture}[scale=0.6]
    \begin{axis}[
        x=.06\textwidth,y=0.1\textwidth,
    	axis y line=center,
    	axis x line=middle,
    	xlabel=$n$,ylabel=$x(n)$,
    	xmin=-7.9,xmax=11.9,
    	ymin=-1.3,ymax=2.3,
    	xticklabel style = {xshift=0},
    	yticklabel style = {yshift=5}
	]
	\discretedelta{-7}{0.1};
	\discretedelta{-6}{0.1};
	\discretedelta{-5}{0.1};
	\discretedelta{-4}{0.1};
	\discretedelta{-3}{-1};
	\discretedelta{-2}{0.1};
	\discretedelta{-1}{1};
	\discretedelta{0}{2};
	\discretedelta{1}{1};
	\discretedelta{2}{0.1};
	\discretedelta{3}{1};
	\discretedelta{4}{2};
	\discretedelta{5}{1};
	\discretedelta{6}{0.1};
	\discretedelta{7}{-1};
	\discretedelta{8}{0.1};
	\discretedelta{9}{0.1};
	\discretedelta{10}{0.1};
	\discretedelta{11}{0.1};
    \end{axis}
\end{tikzpicture}}
    \end{center}
    
    Hallar los siguientes valores sin obtener explícitamente la función $X(e^{j\Omega})$:
    \begin{align*}
    \inciso & X(0) & \inciso & \angle X(e^{j\Omega}) & \inciso & \int_{-\pi}^{\pi} X(e^{j\Omega}) d\Omega \\ 
    \inciso & X(e^{j\pi}) & \inciso & \int_{-\pi}^{\pi}  \left|\frac{X(e^{j\Omega})}{d\Omega}\right|^2 d\Omega &
    \end{align*}
\end{ejercicio}
    
    
\begin{ejercicio}
    Para cada una de las siguientes funciones
    \begin{align*}
        \inciso & \parbox{.3\textwidth}{\begin{tikzpicture}[scale=0.6,transform shape]
    \begin{axis}[
        x=0.04\textwidth,y=0.1\textwidth,
    	axis y line=center,
    	axis x line=middle,
    	xlabel=$n$,ylabel={\LARGE $x(n)$},
    	xmin=-4.9,xmax=8.9,
    	ymin=-0.3,ymax=2.9,
    	xticklabel style = {xshift=0},
    	yticklabel style = {yshift=5},
    	]
    	\discretedelta{-4}{0.1};
    	\discretedelta{-3}{0.1};
    	\discretedelta{-2}{0.1};
    	\discretedelta{-1}{0.5};
    	\discretedelta{0}{1};
    	\discretedelta{1}{1.5};
    	\discretedelta{2}{2};
    	\discretedelta{3}{1.5};
    	\discretedelta{4}{1};
    	\discretedelta{5}{0.5};
    	\discretedelta{6}{0.1};
    	\discretedelta{7}{0.1};
    	\discretedelta{8}{0.1};
    \end{axis}
\end{tikzpicture}} &
        \inciso & \parbox{.4\textwidth}{\begin{tikzpicture}[scale=0.6,transform shape]
    \begin{axis}[
        x=0.04\textwidth,y=0.1\textwidth,
    	axis y line=center,
    	axis x line=middle,
    	xlabel=$n$,ylabel={\LARGE $x(n)$},
    	xmin=-9.9,xmax=9.9,
    	ymin=-1.3,ymax=1.9,
    	xticklabel style = {xshift=0},
    	yticklabel style = {yshift=5},
    	]
    	\discretedelta{-8}{0.1};
    	\discretedelta{-7}{-1};
    	\discretedelta{-6}{0.1};
    	\discretedelta{-5}{1};
    	\discretedelta{-4}{0.1};
    	\discretedelta{-3}{-1};
    	\discretedelta{-2}{0.1};
    	\discretedelta{-1}{1};
    	\discretedelta{0}{0.1};
    	\discretedelta{1}{-1};
    	\discretedelta{2}{0.1};
    	\discretedelta{3}{1};
    	\discretedelta{4}{0.1};
    	\discretedelta{5}{-1};
    	\discretedelta{6}{0.1};
    	\discretedelta{7}{1};
    	\discretedelta{8}{0.1};
    	\node at (-9,0.5) {\Large $\cdots$};
    	\node at (9,0.5) {\Large $\cdots$};
    \end{axis}
\end{tikzpicture}} \\
        \inciso & \parbox{.3\textwidth}{\begin{tikzpicture}[scale=0.6,transform shape]
    \begin{axis}[
        x=0.03\textwidth,y=0.1\textwidth,
    	axis y line=center,
    	axis x line=middle,
    	xlabel=$n$,ylabel={\LARGE $x(n)$},
    	xmin=-9.9,xmax=9.9,
    	ymin=-1.3,ymax=2.9,
    	xticklabel style = {xshift=0},
    	yticklabel style = {yshift=5},
    	]
    	\discretedelta{-9}{0.1};
    	\discretedelta{-8}{0.1};
    	\discretedelta{-7}{0.1};
    	\discretedelta{-6}{0.1};
    	\discretedelta{-5}{0.1};
    	\discretedelta{-4}{-1};
    	\discretedelta{-3}{-1};
    	\discretedelta{-2}{0.1};
    	\discretedelta{-1}{0.1};
    	\discretedelta{0}{2};
    	\discretedelta{1}{-1};
    	\discretedelta{2}{0.1};
    	\discretedelta{3}{0.1};
    	\discretedelta{4}{1};
    	\discretedelta{5}{0.1};
    	\discretedelta{6}{0.1};
    	\discretedelta{7}{0.1};
    	\discretedelta{8}{0.1};
    \end{axis}
\end{tikzpicture}} &
        \inciso & \parbox{.4\textwidth}{\begin{tikzpicture}[scale=0.6,transform shape]
    \begin{axis}[
        x=0.04\textwidth,y=0.1\textwidth,
    	axis y line=center,
    	axis x line=middle,
    	xlabel=$n$,ylabel={\LARGE $x(n)$},
    	xmin=-9.9,xmax=9.9,
    	ymin=-1.3,ymax=2.9,
    	xticklabel style = {xshift=0},
    	yticklabel style = {yshift=5},
    	]
    	\discretedelta{-9}{0.1};
    	\discretedelta{-8}{0.1};
    	\discretedelta{-7}{0.1};
    	\discretedelta{-6}{2};
    	\discretedelta{-5}{0.1};
    	\discretedelta{-4}{-1};
    	\discretedelta{-3}{-1};
    	\discretedelta{-2}{0.1};
    	\discretedelta{-1}{1};
    	\discretedelta{0}{0.1};
    	\discretedelta{1}{1};
    	\discretedelta{2}{0.1};
    	\discretedelta{3}{-1};
    	\discretedelta{4}{-1};
    	\discretedelta{5}{0.1};
    	\discretedelta{6}{2};
    	\discretedelta{7}{0.1};
    	\discretedelta{8}{0.1};
    	\discretedelta{9}{0.1};
    \end{axis}
\end{tikzpicture}} 
    \end{align*}
    indicar si cumplen algunas de estas condiciones:
    \begin{align*}
        \subinciso & \Realpart{X(e^{j\Omega})} = 0 &
        \subinciso & \Impart{X(e^{j\Omega})} = 0 &
        \subinciso & \exists \alpha \in \mathbb{R}\; \mbox{tal que } e^{j\Omega\alpha}X(e^{j\Omega})\; \mbox{es una función real} \\
        \subinciso & \int_{-\pi}^{\pi} X(e^{j\Omega}) d\Omega = 0 &
        \subinciso & X(e^{j\Omega})\; \mbox{es periódica} &
        \subinciso & X(e^{j\Omega})|_{\Omega=0} = 0
    \end{align*}
\end{ejercicio}

\begin{ejercicio}
    Sea un sistema LTI discreto cuya respuesta al impulso es $h(n) = \left(\frac{1}{2}\right)^n u(n)$. Usando la transformada de Fourier de tiempo discreto hallar la salida $y(n)$ del sistema para las siguientes entradas:
    \begin{align*}
        \inciso & x(n) = \left(\frac{3}{4}\right)^n u(n) &
        \inciso & x(n) = (-1)^n u(n) \\
        \inciso & x(n) = A e^{j\frac{\pi}{2}n}, \; \mbox{con } A \in \mathbb{R} &
        \inciso & x(n) = 10 - 5 \sin\left(\frac{\pi}{2}n\right) + 20 \cos\left(\frac\pi n\right)
    \end{align*}
\end{ejercicio}

\begin{ejercicio}
    Utilizando la transformada de Fourier, decidir si existe un sistema LTI que cumpla con que, si $x(n)$ es la entrada, entonces $y(n)$ es la salida:
    
    \inciso $x(n)= e^{j\pi n/3}$ e $y(n) = \cos(\pi n/3)$

    \inciso $x(n) = \cos(\pi n/3)$ e $y(n) = \cos(\pi n/3) + \sqrt{3} \sin(\pi n/3)$
\end{ejercicio}

\begin{ejercicio}
    Sea el siguiente sistema, donde $h_{pb}(n)$ es LTI:
    \begin{center}
    \parbox{.7\textwidth}{\begin{tikzpicture}[scale=0.8, transform shape]
    \node[circle,draw,thick,inner sep=0.03cm] (plus) at (0,0) {\Large +} ;
    \node[circle,draw,thick,inner sep=0.03cm, xshift=-2cm, yshift=1cm] (mult_cos_after) at (plus) {\Large $\times$} ;
    \node[circle,draw,thick,inner sep=0.03cm, xshift=-2cm, yshift=-1cm] (mult_sin_after) at (plus) {\Large $\times$} ;
    \node[rectangle,draw,thick,inner sep=0.3cm, xshift=-2.5cm] (hpb_cos) at (mult_cos_after) {$h_{pb}(n)$} ;
    \node[rectangle,draw,thick,inner sep=0.3cm, xshift=-2.5cm] (hpb_sin) at (mult_sin_after) {$h_{pb}(n)$} ;
    \node[circle,draw,thick,inner sep=0.03cm, xshift=-2.5cm] (mult_cos_before) at (hpb_cos) {\Large $\times$} ;
    \node[circle,draw,thick,inner sep=0.03cm, xshift=-2.5cm] (mult_sin_before) at (hpb_sin) {\Large $\times$} ;
    \node[xshift=-8cm] (arrow_split) at (plus) {} ;
    \node[xshift=-9cm] (x_n) at (plus) {$x(n)$} ;
    \node[xshift=2cm] (y_n) at (plus) {$y(n)$} ;

    \node[xshift=-2cm,yshift=1.5cm] (cos) at (mult_cos_before) {$\cos(\Omega_o n)$} ;
    \node[xshift=-2cm,yshift=-1.5cm] (sin) at (mult_sin_before) {$\sin(\Omega_o n)$} ;

    \draw[very thick] (x_n.east) -- (arrow_split.center) ;
    \draw[->, very thick] (arrow_split.center) |- (mult_cos_before) ;
    \draw[->, very thick] (arrow_split.center) |- (mult_sin_before) ;

    \draw[->, very thick] (cos.east) -| (mult_cos_before.north) ;
    \draw[->, very thick] (cos.east) -| (mult_cos_after.north) ;
    \draw[->, very thick] (sin.east) -| (mult_sin_before.south) ;
    \draw[->, very thick] (sin.east) -| (mult_sin_after.south) ;

    \draw[->, very thick] (mult_cos_before.east) -- (hpb_cos.west) ;
    \draw[->, very thick] (mult_sin_before.east) -- (hpb_sin.west) ;
    \draw[->, very thick] (hpb_cos.east) -- (mult_cos_after.west) ;
    \draw[->, very thick] (hpb_sin.east) -- (mult_sin_after.west) ;
    \draw[->, very thick] (mult_cos_after.east) -| (plus.north) ;
    \draw[->, very thick] (mult_sin_after.east) -| (plus.south) ;

    \draw[->, very thick] (plus.east) -- (y_n.west) ;

\end{tikzpicture}}
    \end{center}

    \inciso Determinar la respuesta al impulso del sistema.

    \inciso Demostrar que el sistema es lineal.

    \inciso Demostrar que el sistema es invariante en el tiempo. \emph{Ayuda:} $\cos(a)\cos(b) + \sin(a)\sin(b) = \cos(a-b)$.

    \inciso Si $\Omega_0 = \frac{\pi}{2}$ y el sistema $h_{pb}(n)$ es un filtro pasabajos ideal con frecuencia de corte $\Omega_c = \frac{\pi}{4}$, determinar la respuesta en frecuencia del sistema.
\end{ejercicio}

\begin{ejercicio}
    Obtener la respuesta en frecuencia del siguiente sistema y compararlo con la del ejercicio anterior:
    \begin{center}
    \parbox{.7\textwidth}{\begin{tikzpicture}[scale=0.8, transform shape]
    \node[rectangle, draw, thick, minimum height=3cm, minimum width=5cm] (hpb) at (0,0) {} ;
    \begin{axis}[
        x=0.04\textwidth,y=0.04\textwidth,
        axis y line=center,
        axis x line=middle,
        xlabel=$\omega$,ylabel={ $H(j\omega)$},
        xmin=-2.9,xmax=2.9,
        ymin=-0.8,ymax=2.7,
        ticks=none,
        at={(-2cm,-1.2cm)}
        ]
        \addplot[
        black,
        ultra thick
        ] coordinates {
            (-1.5,0) (-1.5,1) (1.5,1) (1.5,0)
        } ;
        \node at (-1.5,-.5) {$-\omega_0$};
        \node at (1.5,-.5) {$\omega_0$};
        \node at (0.2,1.3) {1} ;
    \end{axis} 
    \node[yshift=2cm] (hpb_label) at (hpb) {Pasa bajos ideal} ;

    \node[circle,draw,thick,inner sep=0.03cm, xshift=4cm] (mult_after) at (hpb) {\Large $\times$} ;
    \node[circle,draw,thick,inner sep=0.03cm, xshift=-4cm] (mult_before) at (hpb) {\Large $\times$} ;
    \node[yshift=2cm] (exp_after) at (mult_after) {$e^{-j\omega_c t}$} ;
    \node[yshift=2cm] (exp_before) at (mult_before) {$e^{j\omega_c t}$} ;
    \node[xshift=-2cm] (x_t) at (mult_before) {$x(t)$} ;
    \node[xshift=1cm,yshift=0.3cm] (y_t) at (mult_before) {$y(t)$} ;
    \node[xshift=2cm] (f_t) at (mult_after) {$f(t)$} ;
    \node[xshift=-1cm,yshift=0.3cm] (w_t) at (mult_after) {$w(t)$} ;

    \draw[->, very thick] (x_t.east) -- (mult_before.west) ;
    \draw[->, very thick] (mult_before.east) -- (hpb.west) ;
    \draw[->, very thick] (hpb.east) -- (mult_after.west) ;
    \draw[->, very thick] (mult_after.east) -- (f_t.west) ;
    \draw[->, very thick] (exp_before.south) -- (mult_before.north);
    \draw[->, very thick] (exp_after.south) -- (mult_after.north);
\end{tikzpicture}}
    \end{center}
\end{ejercicio}

\begin{ejercicio}
    Sea el sistema de la figura:
    \begin{center}
        \parbox{.7\textwidth}{\begin{tikzpicture}
    
\end{tikzpicture}}
    \end{center}
    donde
    \begin{align*}
        \parbox{.4\textwidth}{\begin{tikzpicture}[scale=0.8,transform shape]
    \begin{axis}[
        x=0.05\textwidth,y=0.05\textwidth,
        axis y line=center,
        axis x line=middle,
        xlabel=$\Omega$,ylabel={\large $X(e^{j\Omega})$},
        xmin=-2.4,xmax=2.9,
        ymin=-0.8,ymax=2.4,
        xtick = {-2, -1, 0, 1, 2},
        xticklabels = {$-2\pi$, $-\pi$, $0$, $\pi$, $2\pi$},
        ytick = {0, 1},
        yticklabels = {$0$,$1$},
        yticklabel style={yshift=0.3cm},
        ]
        \addplot[
        black,
        ultra thick
        ] coordinates {
            (-1,0) (0,1) (1,0)
        } ;
    \end{axis}
\end{tikzpicture}} &
        \parbox{.4\textwidth}{\begin{tikzpicture}[scale=0.8,transform shape]
    \begin{axis}[
        x=0.05\textwidth,y=0.05\textwidth,
        axis y line=center,
        axis x line=middle,
        xlabel=$\Omega$,ylabel={\large $H_{pb}(e^{j\Omega})$},
        xmin=-4.4,xmax=4.9,
        ymin=-0.8,ymax=2.4,
        xtick = {-4, -3, -2, -1, 0, 1, 2, 3, 4},
        xticklabels = {-$\pi$, $-\frac{3\pi}{4}$, $-\frac{\pi}{2}$, $-\frac{\pi}{4}$, $0$, $\frac{\pi}{4}$, $\frac{\pi}{2}$, $\frac{3\pi}{4}$, $\pi$},
        ytick = {0, 1},
        yticklabels = {$0$,$1$},
        yticklabel style={yshift=0.3cm},
        ]

        \addplot[
        black,
        ultra thick
        ] coordinates {
            (-1,0) (-1,1) (1,1) (1,0)
        } ;
    \end{axis}
\end{tikzpicture}}
    \end{align*}
    Graficar los espectros en frecuencias de las señales $g_1(n)$, $g_2(n)$, $g_3(n)$, $g_4(n)$ y $y(n)$.
\end{ejercicio}
        \newpage
        \myheader{Guía 5: Ecuaciones Diferenciales \\ y en Diferencias (Parte I)}

\begin{ejercicio}
    Sea un sistema LTI de tiempo discreto tal que la relación entre la salida y la entrada del mismo se puede escribir como:
    \begin{equation*}
        y(n) = \sum_{k=0}^{\infty} \alpha^{n-k} \left(x(k) + \beta x(k-1) + \gamma x(k-2)\right)
    \end{equation*}
    donde $|\alpha| < 1$ y $\beta, \gamma \in \R$.

    \inciso Obtener la respuesta al impulso del sistema y graficarla.
    
    \inciso Analizar la estabilidad del mismo.
    
    \inciso Determinar los valores de $\beta$ y $\gamma$ para que el sistema tenga las siguientes propiedades:
    \begin{itemize}
        \item Si $x(n) = (-1)^n$ para todo $n$ entonces $y(n) = 0$ para todo $n$.
        \item Si $x(n) = 1$ para todo $n$ entonces $y(n) = 1$ para todo $n$.
    \end{itemize}
\end{ejercicio}

\begin{ejercicio}
    Dado un sistema en tiempo discreto definido por la ecuación en diferencias 
    \begin{equation*}
        y(n) = x(n) + \frac{3}{4} y(n-1)
    \end{equation*}
    con condiciones de contorno
    \begin{align*}
    \inciso & y(-1) = 0 & \inciso & y(1) = 0 & \inciso & y(0) = 0 \\[.5em]
    \inciso & y(0) = 1 & \inciso & \lim_{n\rightarrow -\infty} y(n) = 0  & \inciso & \lim_{n\rightarrow +\infty} y(n) = 0
    \end{align*}
    se pide:
    \begin{itemize}
        \item Calcular el valor de la respuesta al impulso $h(n)$ del sistema en el intervalo $-5 \leq n \leq 5$
        \item Obtener una expresión cerrada de $h(n)$ para todo $n \in \mathbb{Z}$.
        \item Determinar si el sistema lineales, invariante ante desplazamientos, estables y causales.
    \end{itemize}
\end{ejercicio}
    
\begin{ejercicio}
    Dado un sistema en tiempo discreto definido por la ecuación en diferencias 
    \begin{equation*}
        y(n) = x(n) + \frac{1}{2} y(n-1)
    \end{equation*}
    con condiciones
    
    \inciso iniciales de reposo 

    \inciso finales de reposo

    \vspace*{1ex}

    \noindent se pide:
    \begin{itemize}
        \item Determinar si el sistema lineales, invariante ante desplazamientos, estables y causales. Establecer cómo se relacionan las condiciones de contorno con cada una de las propiedades del sistema.
        \item Calcular el valor de la respuesta a las señales $\delta(n-1)$ y $\delta(n+1)$ del sistema en el intervalo $-5 \leq n \leq 5$.
        \item Obtener una expresión cerrada de la respuesta a las señales $\delta(n-1)$ y $\delta(n+1)$ para todo $n \in \mathbb{Z}$.
    \end{itemize}
\end{ejercicio}
    
\begin{ejercicio}
    Demostrar que un sistema definido por ecuaciones en diferencias FIR siempre será lineal, invariante ante desplazamientos y estable. Determinar, además, qué condición debe cumplir un sistema de este tipo para ser causal.
\end{ejercicio}
    
\begin{ejercicio}
    Obtener la respuesta al impulso para el sistema definido por la ecuación diferencial 
    \begin{equation*}
        y(n) = x(n+1) + x(n) + x(n-1) + \frac{3}{4} y(n-1)
    \end{equation*}
    en condiciones inciales de reposo. 
\end{ejercicio}
    
\begin{ejercicio}
    Obtener la respuesta al impulso para el sistema definido por la ecuación diferencial 
    \begin{equation*}
        y(n) = x(n+1) + x(n) + x(n-1) + \frac{5}{4} y(n-1)
    \end{equation*}
    en condiciones finales de reposo. 
\end{ejercicio}
    
\begin{ejercicio}
    Implementar en \Keyboardsym una función que permita obtener la respuesta al impulso de una ecuación en diferencias para condiciones iniciales o finales de reposo.
\end{ejercicio}
    
\begin{ejercicio}
    Para el sistema descripto por la ecuación en diferencias $y(n) = \frac{1}{2} y(n-1) + x(n)$ con condiciones iniciales de reposo, se pide:
    
    \inciso Encontrar la expresión analítica de la salida $y(n)$ ante una entrada $x(n)$ aplicada a partir de $n = 0$.
    
    \inciso Encontrar a partir de la expresión anterior la salida que corresponde cuando la entrada es $x(n) = \delta(n)$. 
    
    \inciso Encontrar a partir de la expresión general de la salida del sistema, la salida que corresponde
    cuando la entrada es $x(n) = A e^{j\frac{\pi}{2}n}, n \geq 0$. A partir de esta expresión, y
    comparándola con la obtenida en el ejercicio anterior (c) discuta el significado de la respuesta permanente y de la transitoria. Grafique ambos tipos de respuesta en \Keyboard \hspace*{0.1em} y discuta cómo es mejor implementar la función que obtiene la salida del sistema, si mediante una convolución o un filtro recursivo.
\end{ejercicio}

\begin{ejercicio}
    Sea un sistema LTI y una señal $x(n)$ que satisface $x(n)= \delta(n) + \sum_{k=1}^M a_k x(n-k)$ donde $a_k \in \R$ para todo $k$. Sea $y(n)$ la salida del sistema cuando la entrada es $x(n)$.
    
    \inciso Determinar una ecuación en diferencias de $h(n)$ en función de $y(n)$ usando las propiedades básicas de un sistema LTI. 

    \inciso Si $y(n)$ es de duración finita, ¿qué se puede decir de la estabilidad del sistema?

    \inciso Si $y(n)$ es de duración infinita, obtener algún tipo de condición sobre $y(n)$ que asegure la estabilidad del sistema. 
\end{ejercicio}

\begin{ejercicio}
    Sea el sistema de tiempo continuo que, dada una entrada $x(t)$, la salida es $y(t) = \sum_{k=0}^{\infty} a_k x(t-kT)$ donde $a_k \in \R$ para todo $k$ y $T>0$. Este sistema puede modelar una situación donde existen una superposición de distintos ecos de la señal de entrada.

    \inciso Determinar si el sistema es LTI y, en caso afirmativo, determinar la respuesta al impulso.

    \inciso ¿Cuáles son las condiciones que en general deben cumplir los valores de $a_k$ (las amplitudes de
    los ecos) para que el sistema sea estable? Con este resultado, analizar el caso en que $a_k = \alpha^k$ con $\alpha \in \R$.

    \inciso Asumiendo que $a_k = \alpha^k$ con $|\alpha| < 1$, obtener un sistema que recupere la entrada $x(t)$ a partir de la salida $y(t)$ con ecos. Determinar, de ser posible, la ecuación en diferencias a coeficientes constantes que implementa el mismo. 
\end{ejercicio}

\begin{ejercicio}
    Sea la señal $x(t)$ cuya transformada de Fourier es $X(j\omega) = u(\omega - W_1) - u(\omega - W_2)$, con $W_1 < W_2$ y sea 
    \begin{equation*}
        y(t) = \frac{dx(t)}{dt} * x(t) + x(t)
    \end{equation*}
    Obtener $\int_{-\infty}^{\infty} |y(t-\alpha)|^2 dt$ donde $\alpha \in \R$.
\end{ejercicio}

\begin{ejercicio}
    Sea un sistema LTI cuya entrada es $x(n)$ y salida es $y(n)$. La relación entre la entrada y salida es:
    \begin{equation*}
        y(n) - \alpha y(n-1) = \sum_{k=-\infty}^{\infty} x(k) z(n-k) - x(n)
    \end{equation*}
    donde $z(n)$ es una secuencia cuya transformada de Fourier existe. 

    \inciso Encontrar la respuesta en frecuencia del sistema con $\alpha = \frac{1}{2}$. 

    \inciso Asumiendo que $z(n) = \beta^n u(n) + \delta(n)$ con $\beta = \frac{1}{3}$, encontrar la respuesta al impulso del sistema.
\end{ejercicio}

\begin{ejercicio}
    Sea un sistema LTI en tiempo discreto dado por:
    \begin{equation*}
        y(n) = \sum_{k=-\infty}^{n-1} \alpha^{n-k} \left( x(k - \beta_1) + x(k - \beta_2) \right),\; |\alpha| < 1, \; \beta_1, \beta_2 \in \mathbb{Z}\; \beta_2 > \beta_1
    \end{equation*}

    \inciso Determinar la respuesta al impulso $h(n)$ del sistema y la estabilidad del mismo. 

    \inciso Encontrar una ecuación en diferencias recursiva para el sistema.
\end{ejercicio}

\begin{ejercicio}
    Sea un sistema LTI en tiempo discreto causal con entrada $x(n)$ y salida $y(n)$ definido por el siguiente \emph{par} de ecuaciones diferenciales:
    \begin{align*}
        y(n) + \frac{1}{4} y(n-1) + w(n) + \frac{1}{2} w(n-1) &= \frac{2}{3} x(n) \\
        y(n) - \frac{5}{4} y(n-1) + 2w(n) - 2 w(n-1) &= -\frac{5}{3} x(n)
    \end{align*}
    \inciso Obtener la respuesta en frecuencia y la respuesta al impulso del sistema.
    
    \inciso Obtener una única ecuación en diferencias que relacione $x(n)$ con $y(n)$.
\end{ejercicio}

\begin{ejercicio}
    Un motor paso-a-paso puede modelarse a través del siguiente circuito:
    \begin{center}
        \parbox{0.5\textwidth}{
            \begin{circuitikz}[scale=0.8, transform shape, american voltages]
    \draw
    (0,0) to [american current source, l^=$i(t)$] (0,4)
    to (2,4)
    to [R, l^={\parbox{1.2cm}{
        \begin{center}
            $R_1$ \\ $100\;\Omega$    
        \end{center}
        }}] (2,0)
    to (0,0) ;
    \draw (2, 4) to [R, l_={$R_2$}] (6,4)
    to [L, l_={\parbox{1.2cm}{
        \begin{center}
            $L_1$ \\ $50\;\mu\mathrm{H}$    
        \end{center}
        }}, v^=$v_L(t)$] (6,0)
    to (2,0) ;

    \draw[->, thick] (3.3, 4.5) -- (4.6, 4.5);
    \node at (3.9, 4.8) {$i_1(t)$};
\end{circuitikz}
        }
    \end{center}
    
    En el mismo, el motor (representado por la inductancia $L_1$) se alimenta de una fuente de corriente $i(t)$. La ecuación de la corriente $i_1(t)$ que circula por el mismo se puede escribir como:
    \begin{equation*}
        i(t) = \frac{v_L(t)}{R_1} + \left(1 + \frac{R_2}{R_1}\right) i_1(t)
    \end{equation*}
    en donde la tensión $v_L(t) = L_1 \frac{d i_1(t)}{dt}$ es la tensión en los bornes de la inductancia. La fuente de corriente $i(t)$ entrega una corriente periódica de la siguiente forma:
    \begin{center}
        \parbox{0.5\textwidth}{
            \begin{tikzpicture}
    
\end{tikzpicture}
        }
    \end{center}
    \inciso Determinar la respuesta en frecuencia del sistema donde la entrada es $i(t)$ y la salida es $v_L(t)$.

    \inciso Determinar la representación en serie de Fourier de $i(t)$ y $v_L(t)$.

    \inciso Si se desea que la amplitud de la quinta armónica de la tensión en los bornes del inductor no supere los $10\;\mathrm{mV}$ dado que el motor debe funcionar cerca de un equipo sensible a la interferencia, determinar el valor de $R_2$ que garantiza eso. 
\end{ejercicio}


        \newpage
        \myheader{Guía 6: Muestreo e Interpolación}


\begin{ejercicio}
    A continuación se muestra el sistema global para filtrar una señal en tiempo continuo utilizando un filtro en tiempo discreto.
    \begin{center}
        \begin{tikzpicture}[scale=.8,transform shape]
    \node[circle,radius=1cm,draw,thick] (x) at (0,0) {$\times$};
    \node[above=1.5cm] (deltatrain) at (x) {$s(t) = \sum_{n=-\infty}^{\infty} \delta(t-nT)$};
    \node[left=3cm] (x_t) at (x) {$x_c(t)$} ;
    \node[above=5cm,right=.4cm] (x_p) at (x.north) {$x_s(t)$} ;
    \node[right=1.5cm,rectangle,draw,thick,inner sep=0.2cm] (ad) at (x) {\parbox{2cm}{\tiny Conversor de tren de impulsos a secuencia en tiempo discreto}} ;
    % \node[right=1.5cm,rectangle,draw,thick,inner sep=0.5cm] (ad) at (x) {$H(j\omega)$} ;
    \node[yshift=.3cm,right=1.8cm] at (ad) {$x(n)$} ;
    \node[right=3cm,rectangle,draw,thick,inner sep=0.5cm] (H_W) at (ad) {$H(e^{j\Omega})$} ; 
    \node[yshift=.3cm,right=1.2cm] at (H_W) {$y(n)$} ;
    
    \node[right=3cm,rectangle,draw,thick,inner sep=0.2cm] (da) at (H_W) {\parbox{1.5cm}{\tiny Conversor a tren de impulsos}} ;
    
    \node[right=2.5cm,rectangle,draw,thick,inner sep=0.2cm] (bp) at (da) {\parbox{1.5cm}{\tiny Filtro de reconstrucción ideal $H_r(j\omega)$}} ;
    
    \node[yshift=.5cm,right=1.2cm] (y_s) at (da) {$y_s(t)$} ;
    \node[right=2cm] (yr_t) at (bp) {$y_r(t)$} ;
    \node[dashed,thick,rectangle,draw,minimum width=7cm,minimum height=4cm,yshift=.5cm] (ad_box) at (x_p) {} ;
    \node[yshift=2.5cm] (ad_label) at (ad_box) {A/D} ;
    
    \node[dashed,thick,rectangle,draw,minimum width=6.5cm,minimum height=4cm,yshift=.4cm] (da_box) at (y_s) {} ;
    \node[yshift=2.5cm] (da_label) at (da_box) {D/A} ;
    
    \draw[->,very thick] (deltatrain) -- (x) ;
    \draw[->,very thick] (x_t) -- (x) ;
    \draw[->,very thick] (x) -- (ad) ;
    \draw[->,very thick] (ad) -- (H_W) ;
    \draw[->,very thick] (H_W) -- (da) ;
    \draw[->,very thick] (da) -- (bp) ;
    \draw[->,very thick] (bp) -- (yr_t) ;
    \end{tikzpicture}
    \end{center}
    Asumiendo que $H(e^{j\Omega})$ y $H_r(j\omega)$ tienen la forma
    \begin{center}
        \begin{tabular}{ccc}
    \begin{tikzpicture}[scale=0.9,transform shape]
    \begin{axis}[
        x=0.05\textwidth,y=0.05\textwidth,
        axis y line=center,
        axis x line=middle,
        xlabel=$\Omega$,ylabel={\large $H(e^{j\Omega})$},
        xmin=-2.9,xmax=2.9,
        ymin=-0.9,ymax=2.9,
        ticks=none
        ]
        \addplot[
        black,
        ultra thick
        ] coordinates {
            (-1.5,0) (-1.5,1) (1.5,1) (1.5,0)
        } ;
        \node at (-1.5,-.5) {$-\frac{\pi}{4}$};
        \node at (1.5,-.5) {$\frac{\pi}{4}$};
        \node at (0.2,1.2) {1} ;
    \end{axis} 
    \end{tikzpicture} & \hfill &
    \begin{tikzpicture}[scale=0.9,transform shape]
    \begin{axis}[
        x=0.05\textwidth,y=0.05\textwidth,
        axis y line=center,
        axis x line=middle,
        xlabel=$\omega$,ylabel={\Large $H_r(j\omega)$},
        xmin=-2.9,xmax=2.9,
        ymin=-0.9,ymax=2.9,
        ticks=none
        ]
        \addplot[
        black,
        ultra thick
        ] coordinates {
            (-1.5,0) (-1.5,1) (1.5,1) (1.5,0)
        } ;
        \node at (-1.5,-.5) {$-\frac{\pi}{T}$};
        \node at (1.5,-.5) {$\frac{\pi}{T}$};
        \node at (0.2,1.2) {T} ;
    \end{axis} 
    \end{tikzpicture}
\end{tabular}
    \end{center}
    se pide:
    
    \inciso Para el caso en que $\frac{1}{T}=20\;\mathrm{kHz}$ y la transformada $X_c(j\omega)$ de $x_c(t)$ es
    \begin{center}
        \begin{tikzpicture}[scale=0.9,transform shape]
    \begin{axis}[
        x=0.05\textwidth,y=0.05\textwidth,
        axis y line=center,
        axis x line=middle,
        xlabel=$\omega$,ylabel={\large $X(j\omega)$},
        xmin=-2.9,xmax=2.9,
        ymin=-0.8,ymax=1.9,
        ticks=none
        ]
        \addplot[
        black,
        ultra thick
        ] coordinates {
            (-1,0) (0,1) (1,0)
        } ;
        \node at (-1,-.3) {$-2\pi f_M$};
        \node at (1,-.3) {$2\pi f_M$};
        \node at (-.3,1) {1} ;
    \end{axis}
    \end{tikzpicture}
    \end{center}
    con $f_M=10\;\mathrm{kHz}$ graficar las transformadas $X_s(j\omega)$ y $X(e^{j\Omega})$ de $x_s(t)$ y $x(n)$ respectivamente.
    
    \inciso Determinar para qué rango de valores de $T$, el sistema completo con entrada $x_c(t)$ de banda limitada ($2\pi f_M=\mbox{frecuencia máxima de la señal}$) y salida $y_r(t)$ es equivalente al sistema LTI con respuesta en frecuencia $H_{eff}(j\omega)=\begin{cases}1 & \mbox{si $\omega \in (-\omega_c,\omega_c)$}\\ 0 & \mbox{en otro caso}\end{cases}$. Determinar el valor de $\omega_c$ en función de $T$.
    
\end{ejercicio}
    
    
\begin{ejercicio}
    La señal discreta $x_d(n)=\cos\left(\frac{\pi}{4}n\right)$ con $n\in\mathbb{Z}$ se obtuvo del muestreo de la señal continua $x_c(n)=\cos\left(\omega_0 t\right)$ con $t\in\mathbb{R}$ a una frecuencia de muestreo $F_S = 1000\;\mathrm{Hz}$. ¿Qué valores de $\omega_0$ positivos resultarían en la secuencia $x_d(n)$?
\end{ejercicio}

\begin{ejercicio}
    Sea el siguiente sistema de muestreo
    \begin{center}
    \begin{tikzpicture}
    \node[circle,inner sep=0.05cm, draw, very thick] (x) at (0,0) {\large $\times$};
    \node[above=1.5cm] (deltatrain) at (x) {$p(t) = \sum_{n=-\infty}^{\infty} \delta(t-nT)$};
    \node[left=1.5cm] (x_t) at (x) {$x(t)$} ;
    \node[right=1.5cm,rectangle,draw,very thick,inner sep=0.5cm] (H_jw) at (x) {$H_c(j\omega)$} ;
    \node[right=2cm] (x_r) at (H_jw) {$y_c(t)$} ;
    \node[above=5cm,right=.4cm] at (x.north) {$x_p(t)$} ;
    
    \draw[->,very thick] (deltatrain) -- (x) ;
    \draw[->,very thick] (x_t) -- (x) ;
    \draw[->,very thick] (x) -- (H_jw) ;
    \draw[->,very thick] (H_jw) -- (x_r) ;
\end{tikzpicture}
    \end{center}
    donde $H_c(j\omega)$ es un pasabajos ideal con frecuencia de corte $\frac{\omega_s}{2}$ y ganancia $T$ y donde $\omega_s = \frac{2\pi}{T}$. En este sistema, cuando la entrada vale $x_c(t) = \cos(\omega_0 t)$ la salida es $y_c(t) = \cos(\omega_0 t)$. Por otro lado, cuando $x_c(t) = \cos(10\omega_0 t)$ la salida es $y_c(t) = \cos(2\omega_0 t)$. Determinar un valor de $\omega_s$ compatible con esta situación y graficar los espectros de $x_p(t)$ para los dos casos, explicando clararmente la situación en cada uno de ellos.
\end{ejercicio}
    
\begin{ejercicio}
    Dado el siguiente sistema de muestreo
    \begin{center}
    \begin{tabular}{ccc}
    \begin{tikzpicture}
    \node[circle,radius=1cm,draw,thick] (x) at (0,0) {$\times$};
    \node[above=1.5cm] (deltatrain) at (x) {$p(t) = \sum_{n=-\infty}^{\infty} \delta(t-nT)$};
    \node[left=1.5cm] (x_t) at (x) {$x(t)$} ;
    \node[right=1.5cm,rectangle,draw,thick,inner sep=0.5cm] (H_jw) at (x) {$H(j\omega)$} ;
    \node[right=2cm] (x_r) at (H_jw) {$x_r(t)$} ;
    \node[above=5cm,right=.4cm] at (x.north) {$x_p(t)$} ;
    
    \draw[->,very thick] (deltatrain) -- (x) ;
    \draw[->,very thick] (x_t) -- (x) ;
    \draw[->,very thick] (x) -- (H_jw) ;
    \draw[->,very thick] (H_jw) -- (x_r) ;
    \end{tikzpicture} & \hfill &
    \begin{tikzpicture}[scale=0.9,transform shape]
    \begin{axis}[
        x=0.05\textwidth,y=0.05\textwidth,
        axis y line=center,
        axis x line=middle,
        xlabel=$\omega$,ylabel={\Large $H(j\omega)$},
        xmin=-4.9,xmax=4.9,
        ymin=-0.9,ymax=2.9,
        ticks=none
        ]
        \addplot[
        black,
        ultra thick
        ] coordinates {
            (-4,0) (-4,1) (-2.5,1) (-2.5,0) 
            (2.5,0) (2.5,1) (4,1)  (4,0)
        } ;
        \node at (-4,-.5) {$-\omega_b$};
        \node at (-2.5,-.5) {$-\omega_a$};
        \node at (4,-.5) {$\omega_b$};
        \node at (2.5,-.5) {$\omega_a$};
        \node at (0.15,1) {$\mathbf{-}A$} ;
    \end{axis}
    \end{tikzpicture}
    \end{tabular}
    \end{center}
    y considerando que $\omega_1 > \omega_2-\omega_1$, encontrar el máximo valor de $T$ y los valores de las constantes $A$, $\omega_a$, $\omega_b$ tales que $x(t)=x_r(t)$ cuando la señal $x(t)$ tiene un espectro como se muetra a continuación
    \begin{center}
    \begin{tikzpicture}[scale=0.9,transform shape]
    \begin{axis}[
        x=0.05\textwidth,y=0.05\textwidth,
        axis y line=center,
        axis x line=middle,
        xlabel=$\omega$,ylabel={\Large $X(j\omega)$},
        xmin=-4.9,xmax=4.9,
        ymin=-0.9,ymax=2.9,
        ticks=none
        ]
        \addplot[
        black,
        ultra thick
        ] coordinates {
            (-4,0) (-2.5,1.5) (-2.5,0) 
            (2.5,0) (2.5,1.5) (4,0)
        } ;
        \node at (-4,-.5) {$-\omega_b$};
        \node at (-2.5,-.5) {$-\omega_a$};
        \node at (4,-.5) {$\omega_b$};
        \node at (2.5,-.5) {$\omega_a$};
        \node at (0.15,1.5) {$\mathbf{-}1$} ;
    \end{axis}
    \end{tikzpicture}
    \end{center}
\end{ejercicio}
    
\begin{ejercicio}
    Sea el sistema de muestreo como se muestra a continuación:
    \begin{center}
    \begin{tabular}{ccc}
    \begin{tikzpicture}
    \node[circle,radius=1cm,draw,thick] (x) at (0,0) {$\times$};
    \node[above=1.5cm] (deltatrain) at (x) {$p(t) = \sum_{n=-\infty}^{\infty} \delta(t-n(T+\Delta))$};
    \node[left=1.5cm] (x_t) at (x) {$x(t)$} ;
    \node[right=1.5cm,rectangle,draw,thick,inner sep=0.5cm] (H_jw) at (x) {$H(j\omega)$} ;
    \node[right=2cm] (x_r) at (H_jw) {$y(t)$} ;
    \node[above=5cm,right=.4cm] at (x.north) {$x_p(t)$} ;
    
    \draw[->,very thick] (deltatrain) -- (x) ;
    \draw[->,very thick] (x_t) -- (x) ;
    \draw[->,very thick] (x) -- (H_jw) ;
    \draw[->,very thick] (H_jw) -- (x_r) ;
    \end{tikzpicture} & \hfill &
    \begin{tikzpicture}[scale=0.9,transform shape]
    \begin{axis}[
        x=0.05\textwidth,y=0.05\textwidth,
        axis y line=center,
        axis x line=middle,
        xlabel=$\omega$,ylabel={\Large $H(j\omega)$},
        xmin=-4.9,xmax=4.9,
        ymin=-0.9,ymax=2.9,
        ticks=none
        ]
        \addplot[
        black,
        ultra thick
        ] coordinates {
            (-2,0) (-2,1) (2,1) (2,0)
        } ;
        \node at (-2.2,-.5) {$-\frac{1}{2(T+\Delta)}$};
        \node at (2,-.5) {$\frac{1}{2(T+\Delta)}$};
        \node at (0.2,1.2) {1} ;
    \end{axis}
    \end{tikzpicture}
    \end{tabular}
    \end{center}
    Para $x(t)=\cos\left(\frac{2\pi}{T}t\right)$ encontrar el intervalo de valores de $\Delta$ de manera que $y(t)$ sea proporcional a $x(at)$ para algún $a\in(0,1)$. Determinar el valor de $a$ en términos de $T$ y de $\Delta$.
\end{ejercicio}
    
\begin{ejercicio}
    Sea el siguiente sistema de muestreo
    \begin{center}
    \begin{tabular}{ccc}
    \begin{tikzpicture}
    \node[circle,radius=1cm,draw,thick] (x) at (0,0) {$\times$};
    \node[above=1.5cm] (deltatrain) at (x) {$p(t) = \sum_{n=-\infty}^{\infty} (-1)^{n} \delta(t-nT)$};
    \node[left=1.5cm] (x_t) at (x) {$x(t)$} ;
    \node[right=1.5cm,rectangle,draw,thick,inner sep=0.5cm] (H_jw) at (x) {$H(j\omega)$} ;
    \node[right=2cm] (x_r) at (H_jw) {$y(t)$} ;
    \node[above=5cm,right=.4cm] at (x.north) {$x_p(t)$} ;
    
    \draw[->,very thick] (deltatrain) -- (x) ;
    \draw[->,very thick] (x_t) -- (x) ;
    \draw[->,very thick] (x) -- (H_jw) ;
    \draw[->,very thick] (H_jw) -- (x_r) ;
    \end{tikzpicture} & \hfill &
    \begin{tikzpicture}[scale=0.9,transform shape]
    \begin{axis}[
        x=0.05\textwidth,y=0.05\textwidth,
        axis y line=center,
        axis x line=middle,
        xlabel=$\omega$,ylabel={\Large $H(j\omega)$},
        xmin=-4.9,xmax=4.9,
        ymin=-0.9,ymax=2.9,
        ticks=none
        ]
        \addplot[
        black,
        ultra thick
        ] coordinates {
            (-4,0) (-4,1) (-2.5,1) (-2.5,0) 
            (2.5,0) (2.5,1) (4,1)  (4,0)
        } ;
        \node at (-4,-.5) {$-\frac{3\pi}{T}$};
        \node at (-2.5,-.5) {$-\frac{\pi}{T}$};
        \node at (4,-.5) {$\frac{3\pi}{T}$};
        \node at (2.5,-.5) {$\frac{\pi}{T}$};
        \node at (0.15,1) {$\mathbf{-}1$} ;
    \end{axis}
    \end{tikzpicture}
    \end{tabular}
    \end{center}
    Dada una señal $x(t)$ cuya transformada de Fourier es como se muestra a continuación:
    \begin{center}
    \begin{tikzpicture}[scale=0.9,transform shape]
    \begin{axis}[
        x=0.05\textwidth,y=0.05\textwidth,
        axis y line=center,
        axis x line=middle,
        xlabel=$\omega$,ylabel={\Large $X(j\omega)$},
        xmin=-4.9,xmax=4.9,
        ymin=-0.9,ymax=2.9,
        ticks=none
        ]
        \addplot [
        black, ultra thick,
        domain=-2.5:2.5, smooth
        ] {cos(deg(2*pi/5*x))*.5+1} ;
        \addplot [
        black, ultra thick
        ] coordinates {(-2.5,0) (-2.5,.5)} ;
        \addplot [
        black, ultra thick
        ] coordinates {(2.5,0) (2.5,.5)} ;
        \node at (-2.5,-.5) {$-\omega_M$};
        \node at (2.5,-.5) {$\omega_M$};
        \node at (0.15,1.5) {$\mathbf{-}1$} ;
    \end{axis}
    \end{tikzpicture}
    \end{center}
    se pide:
    
    \inciso Para $T=\frac{\pi}{2\omega_M}$ dibujar la transformada de Fourier de $x_p(t)$ e $y(t)$.
    
    \inciso Para $T=\frac{\pi}{2\omega_M}$ determinar un sistema con el cual se pueda recuperar $x(t)$ a partir de $x_p(t)$.
    
    \inciso Para $T=\frac{\pi}{2\omega_M}$ determinar un sistema con el cual se pueda recuperar $x(t)$ a partir de $y(t)$.
    
    \inciso Determinar el valor \emph{máximo} de $T$ en relación a $\omega_M$ para el cual $x(t)$ puede recuperarse a partir de $x_p(t)$ o de $y(t)$.
    
\end{ejercicio}
    
\begin{ejercicio}
    En la siguiente figura se muestra un sistema en el cual la señal de entrada es multiplicada por una onda cuadrada periódica $s(t)$ de período $T$. 
    \begin{center}
    \begin{tabular}{ccc}
    \begin{tikzpicture}
    \node[circle,radius=1cm,draw,thick] (x) at (0,0) {$\times$};
    \node[above=2cm] (deltatrain) at (x) {$s(t)$};
    \node[left=2cm] (x_t) at (x) {$x(t)$} ;
    % \node[right=1.5cm,rectangle,draw,thick,inner sep=0.5cm] (H_jw) at (x) {$H(j\omega)$} ;
    % \node[right=2cm] (x_r) at (H_jw) {$y(t)$} ;
    \node[right=2cm] (w) at (x) {$w(t)$} ;
    
    \draw[->,very thick] (deltatrain) -- (x) ;
    \draw[->,very thick] (x_t) -- (x) ;
    \draw[->,very thick] (x) -- (w) ;
    % \draw[->,very thick] (x) -- (H_jw) ;
    % \draw[->,very thick] (H_jw) -- (x_r) ;
    \end{tikzpicture} & \hfill &
    \begin{tikzpicture}[scale=0.9,transform shape]
    \begin{axis}[
        x=0.05\textwidth,y=0.05\textwidth,
        axis y line=center,
        axis x line=middle,
        xlabel=$t$,ylabel={\large $s(t)$},
        xmin=-4.9,xmax=4.9,
        ymin=-1.9,ymax=2.3,
        ticks=none
        ]
        \addplot[
        black,
        ultra thick
        ] coordinates {
            (-5.5,1) (-4.5 1) (-3.5, 1) (-3.5,-1) (-2.5,-1) (-2.5, 1) (-1.5, 1) (-1.5, -1) (-.5,-1) (-.5,1) (.5,1) (.5,-1) (1.5,-1) (1.5,1) (2.5,1) (2.5,-1) (3.5,-1) (3.5,1) (4.5,1)
        } ;
        \node at (0.2,1.3) {$1$} ;
        \node at (0.35,-1.3) {$-1$} ;
        \node at (-.85,-0.3) {$-\Delta$} ;
        \node at (.75,-0.3) {$\Delta$} ;
        \node at (2,-0.4) {$T$} ;
        \node at (2,0) {$|$} ;
    \end{axis}
    \end{tikzpicture}
    \end{tabular}
    \end{center}
    Suponiendo que la entrada es de banda limitada ($X(j\omega)=0\; \forall \omega>\omega_M$
    
    \inciso Para $\Delta=\frac{T}{3}$ determinar en términos de $\omega_M$ el valor máximo de $T$ para el cual no hay traslape entre las réplicas de $X(j\omega)$ en $W(j\omega)$.
    
    \inciso Para $\Delta=\frac{T}{4}$ determinar en términos de $\omega_M$ el valor máximo de $T$ para el cual no hay traslape entre las réplicas de $X(j\omega)$ en $W(j\omega)$
\end{ejercicio}
    
\begin{ejercicio}
    Escribir las ecuaciones en diferencias de un interpolador de orden cero y otro de orden uno, en su versión discreta. 
    
    \inciso Implementar las ecuaciones en \Keyboardsym
    
    \inciso Interpolar la señal $x_e(n)$ definida como
    \begin{equation*}
        x_e(n) = \begin{cases}
        \cos(\frac{2\pi100}{L}) & \mbox{para $n=kL$, $k\in \mathbb{Z}$} \\
        0 & \mbox{en otro caso}
        \end{cases}
    \end{equation*}
    
    \inciso ¿Es posible implementar realmente una interpolación ideal? ¿Qué aproximaciones se podrían hacer para obtenerlo y cómo se altera el espectro de la señal al hacerlas?
\end{ejercicio}

\begin{ejercicio}
    Sea el siguiente sistema:
    \begin{center}
        \begin{tikzpicture}[scale=1, transform shape]
    \node[rectangle, very thick, draw, minimum width=2cm, minimum height=1cm] (ad) at (0,0) {\large A/D};
    \node[rectangle, very thick, draw, minimum width=2cm, minimum height=1cm, xshift=-4cm] (hc) at (ad) {$H_c(j\omega)$};
    \node[rectangle, very thick, draw, minimum width=2cm, minimum height=1cm, xshift=4cm] (h) at (ad) {$H(e^{j\Omega})$};

    \node[xshift=-2cm] (r_c) at (hc) {$r_c(t)$};
    \node[xshift=3cm] (y_n) at (h) {$y(n)$};
    \node[xshift=-2cm,yshift=0.4cm] (x_c) at (ad) {$x_c(t)$};
    \node[xshift=2cm,yshift=0.4cm] (x_n) at (ad) {$x(n)$};
    \node[yshift=-1.2cm,xshift=0.4cm] (t1) at (ad) {$T_1$};

    \draw[->, very thick] (hc.east) -- (ad.west);
    \draw[->, very thick] (ad.east) -- (h.west);
    \draw[->, very thick] (r_c.east) -- (hc.west);
    \draw[->, very thick] (h.east) -- (y_n.west);
    \draw[->, very thick] (ad.south) ++ (0,-1cm) -- (ad.south) ;
\end{tikzpicture}
    \end{center}
    donde se sabe que el sistema $H_c(j\omega)$ es LTI, causal y se describe mediante:
    \begin{equation*}
        \frac{dx_c(t)}{dt} + \alpha x_c(t) = r_c(t), \; \alpha > 0,\; \mbox{con condición de reposo incial}
    \end{equation*}
    Sea, además, la señal $r_c(t) = u(t) - u(t-T_1)$, donde $T_1$ es también el período de muestreo del conversor A/D ideal de la figura.

    \inciso Obtener la señal $x_c(t)$.

    \inciso Encontrar el sistema LTI de tiempo discreto $H(e^{j\Omega})$ tal que $y(n) = \delta(n) - 2\delta(n-1) + \delta(n-2)$. De ser posible, obtener una ecuación en diferencias para dicho sistema.
\end{ejercicio}

\begin{ejercicio}
    Sea el siguiente sistema:
    \begin{center}
        \begin{tikzpicture}[scale=1, transform shape]
    \node[rectangle, very thick, draw, minimum width=2cm, minimum height=1cm] (ad) at (0,0) {\large A/D};
    \node[rectangle, very thick, draw, minimum width=2cm, minimum height=1cm, xshift=-4cm] (hc) at (ad) {$H_c(j\omega)$};
    \node[rectangle, very thick, draw, minimum width=2cm, minimum height=1cm, xshift=4cm] (h) at (ad) {$H(e^{j\Omega})$};
    \node[rectangle, very thick, draw, minimum width=2cm, minimum height=1cm,xshift=4cm] (da) at (h) {\large D/A};

    \node[xshift=-2cm] (r_c) at (hc) {$r_c(t)$};
    \node[xshift=2cm] (y_c) at (da) {$y_c(t)$};
    \node[yshift=0.4cm,xshift=2cm] (y_n) at (h) {$y(n)$};
    \node[xshift=-2cm,yshift=0.4cm] (x_c) at (ad) {$x_c(t)$};
    \node[xshift=2cm,yshift=0.4cm] (x_n) at (ad) {$x(n)$};
    \node[yshift=-1.2cm,xshift=0.4cm] (t1) at (ad) {$T_1$};
    \node[yshift=-1.2cm,xshift=0.4cm] (t2) at (da) {$T_2$};

    \draw[->, very thick] (hc.east) -- (ad.west);
    \draw[->, very thick] (ad.east) -- (h.west);
    \draw[->, very thick] (r_c.east) -- (hc.west);
    \draw[->, very thick] (h.east) -- (da.west);
    \draw[->, very thick] (da.east) -- (y_c.west);
    \draw[->, very thick] (ad.south) ++ (0,-1cm) -- (ad.south) ;
    \draw[->, very thick] (da.south) ++ (0,-1cm) -- (da.south) ;
\end{tikzpicture}
    \end{center}
    donde $r_c(t) = \cos^2(\omega_0t)\cos^2(2\omega_0t)$ y los conversores A/D y D/A son ideales. El filtro $H_c(j\omega)$ es un filtro pasabajos ideal con frecuencia de corte $3\omega_0$. Además, $T_1=\frac{2\pi}{3\omega_0}$.

    \inciso Asumiendo que $T_1=T_2$ y que $H(e^{j\Omega})=1$ para todo $\Omega \in [-\pi,\pi]$, calcular la salida $y_c(t)$ del sistema y dibujar los espectros en frecuencias de cada una de las señales presentes en la figura.

    \inciso Diseñar un filtro $H(e^{j\Omega})$ y obtener un valor de $T_2$ apropiado para que la salida del sistema sea $y_c(t)=\cos(2\omega_0 t)$.
\end{ejercicio}

\begin{ejercicio}
    Sea la señal $z(t) = x(t) + y(t)$ donde
    \begin{align*}
        x(t) = \frac{\sin^2\left(\frac{W}{2}t\right)}{\pi^2t^2} & \hspace*{1em} & y(t) = 2\frac{\sin\left(\frac{W}{2}t\right)}{\pi t} \cos\left(\frac{5W}{2}t\right)
    \end{align*}
    La señal $z(t)$ ingresa a un conversor A/D ideal con frecuencia de muestreo $\omega_s$.

    \inciso Calcular la frecuencia de Nyquist para esta señal.

    \inciso Diseñar un sistema que permita, con la mínima frecuencia de muestreo posible, recuperar la señal $y(t)$ usando las muestras $z(n)$ que se obtienen a la salida del conversor A/D. El sistema puede contener partes en tiempo discreto y tiempo continuo.
\end{ejercicio}


% \begin{ejercicio}
%     Dados los siguientes sistemas en cascada
%     \begin{center}
%         \begin{tikzpicture}
    \node[rectangle,draw,thick,inner sep=0.5cm] (H_W) at (0,0) {$H(e^{j\Omega})$} ;
    \node[xshift=3cm,rectangle,draw,thick,inner sep=0.4cm] (down2) at (H_W) {$\downarrow N$} ;
    \node[xshift=-3cm,rectangle,draw,thick,inner sep=0.4cm] (up2) at (H_W) {$\uparrow N$} ;
    \node[xshift=-3cm] (x_n) at (up2) {$x(n)$} ;
    \node[xshift=3cm] (y_n) at (down2) {$y(n)$} ;
    
    \node[rectangle,minimum width=8cm,minimum height=3cm,dashed,thick,draw] (box) at (H_W) {};
    \node[yshift=.5cm] (box_label) at (box.north) {$H_{eff}(e^{j\Omega})$};
    
    \draw[->] (x_n) -- (up2) ;
    \draw[->] (up2) -- (H_W) ;
    \draw[->] (H_W) -- (down2) ;
    \draw[->] (down2) -- (y_n) ;
    \end{tikzpicture}
%     \end{center}
%     determinar la respuesta en frecuencia del sistema equivalente $H_{eff}(e^{j\Omega})$.
% \end{ejercicio}
   
\begin{ejercicio}
    Sea $x_c(t)$ una señal real de tiempo continuo cuya frecuencia superior es $\omega_M = 2\pi 250\mathrm\;{Hz}$, y la señal $y_c(t)$ definida como $y_c(t) = x_c(t - \frac{1}{1000})$.
    
    \inciso Determinar si es posible recuperar $x_c(t)$ a partir de $x(n)=x_c(\frac{n}{500})$.
    
    \inciso Determinar si es posible recuperar $y_c(t)$ a partir de $y(n)=y_c(\frac{n}{500})$.
    
    \inciso Determinar si es posible obtener un sistema $H(e^{j\Omega})$ de manera que si se implementa en la siguiente estructura en cascada
    \begin{center}
        \begin{tikzpicture}
    \node[rectangle,draw,thick,inner sep=0.5cm] (H_W) at (0,0) {$H(e^{j\Omega})$} ;
    \node[xshift=3cm,rectangle,draw,thick,inner sep=0.4cm] (down2) at (H_W) {$\downarrow 2$} ;
    \node[xshift=-3cm,rectangle,draw,thick,inner sep=0.4cm] (up2) at (H_W) {$\uparrow 2$} ;
    \node[xshift=-3cm] (x_n) at (up2) {$x(n)$} ;
    \node[xshift=3cm] (y_n) at (down2) {$y(n)$} ;
    
    % \node[rectangle,minimum width=8cm,minimum height=3cm,dashed,thick,draw] (box) at (H_W) {};
    % \node[yshift=.5cm] (box_label) at (box.north) {$H_{eff}(e^{j\Omega})$};
    
    \draw[->] (x_n) -- (up2) ;
    \draw[->] (up2) -- (H_W) ;
    \draw[->] (H_W) -- (down2) ;
    \draw[->] (down2) -- (y_n) ;
\end{tikzpicture}
    \end{center}
    es posible obtener $y(n)$ a partir de $x(n)$.
    
    \inciso Determinar si es posible obtener $y(n)$ a partir de $x(n)$ utilizando un único sistema LTI con respuesta en frecuencia $H_{eff}(e^{j\Omega})$. En caso de que así sea, obtener $H_{eff}(e^{j\Omega})$.
    
\end{ejercicio}

\begin{ejercicio}
    Sea la señal
    \begin{equation*}
        x_c(t) = \sum_{k=-\infty}^{\infty} \alpha(k) g(t - kT_0)
    \end{equation*}
    donde $\alpha(k)$ es una secuencia de tiempo discreto tal que $\sum_{k=-\infty}^{\infty} |\alpha(k)| < \infty$ y $g(t)$ es una señal de banda limitada con ancho de banda $W = \frac{2 \pi}{T_0}$. La señal $x_c(t)$ ingresa al siguiente sistema:
    \begin{center}
        \begin{tikzpicture}[scale=1, transform shape]
    \node[rectangle, draw, very thick, minimum width=1cm, minimum height=1cm] (h) at (0,0) {$H(e^{j\Omega})$} ;
    \node[rectangle, draw, very thick, minimum width=1cm, minimum height=1cm, xshift=-2.5cm] (ad) at (h) {\large A/D} ;
    \node[xshift=-2cm] (x_c) at (ad) {$x_c(t)$} ;
    \node[xshift=-1.3cm,yshift=0.4cm] (x_n) at (h) {$x(n)$} ;
    \node[xshift=2cm] (y_n) at (h) {$y(n)$} ;
    \node[yshift=-1.5cm] (t) at (ad) {} ;
    \node[yshift=-1.1cm,xshift=0.4cm] (t_label) at (ad) {$T$} ;

    \draw[->, very thick] (ad.east) -- (h.west) ;
    \draw[->, very thick] (h.east) -- (y_n) ;
    \draw[->, very thick] (x_c) -- (ad.west) ;
    \draw[->, very thick] (t) -- (ad.south) ;
\end{tikzpicture}
    \end{center}
    en donde el conversor A/D tiene frecuencia de muestreo igual a la de Nyquist y el sistema $H(e^{j\Omega})$ es un filtro LTI de tiempo discreto.

    \inciso Justificar que $x_c(t)$ es de banda limitada y determinar $Y(e^{j\Omega})$.

    \inciso Asumiendo que $H(e^{j\Omega}) = \frac{T}{G(j\frac{\Omega}{T})}$ con $T$ el período de muestreo, ¿es posible recuperar la secuencia $\alpha(k)$ a partir de la salida $y(n)$? En caso afirmativo, diseñar un sistema de tiempo discreto que permita recuperar la señal mencionada.
\end{ejercicio}

\begin{ejercicio}
    Sea el siguiente sistema:
    \begin{center}
        \begin{tikzpicture}[scale=1, transform shape]
    \node[rectangle, very thick, draw, minimum width=1cm, minimum height=1cm] (ad) at (0,0) {\large A/D};
    \node[rectangle, very thick, draw, minimum width=1cm, minimum height=1cm, xshift=2.5cm] (up) at (ad) {\large $\uparrow 3$};
    \node[rectangle, very thick, draw, minimum width=1cm, minimum height=1cm, xshift=2.5cm] (h) at (up) {$H(e^{j\Omega})$};
    \node[rectangle, very thick, draw, minimum width=1cm, minimum height=1cm, xshift=2.5cm] (down) at (h) {\large $\downarrow 2$};
    \node[rectangle, very thick, draw, minimum width=1cm, minimum height=1cm, xshift=2.5cm] (da) at (down) {\large D/A};

    \node[xshift=-2cm] (r_c) at (ad) {$r_c(t)$};
    \node[xshift=1.25cm,yshift=0.4cm] (r_n) at (ad) {$r(n)$};
    \node[xshift=1.1cm,yshift=0.4cm] (x_n) at (up) {$x(n)$};
    \node[xshift=1.4cm,yshift=0.4cm] (y_n) at (h) {$y(n)$};
    \node[xshift=1.2cm,yshift=0.4cm] (s_n) at (down) {$s(n)$};
    \node[xshift=2cm] (s_c) at (da) {$s_c(t)$};
    
    \node[yshift=-1.2cm,xshift=0.4cm] (t1) at (ad) {$T_1$};
    \node[yshift=-1.2cm,xshift=0.4cm] (t2) at (da) {$T_3$};

    \draw[->, very thick] (r_c.east) -- (ad.west);
    \draw[->, very thick] (ad.east) -- (up.west);
    \draw[->, very thick] (up.east) -- (h.west);
    \draw[->, very thick] (h.east) -- (down.west);
    \draw[->, very thick] (down.east) -- (da.west);
    \draw[->, very thick] (da.east) -- (s_c.west);
    \draw[->, very thick] (ad.south) ++ (0,-1cm) -- (ad.south) ;
    \draw[->, very thick] (da.south) ++ (0,-1cm) -- (da.south) ;
\end{tikzpicture}
    \end{center}
    donde 
    \begin{align*}
        \parbox{0.5\textwidth}{
                \begin{tikzpicture}[scale=1,transform shape]
    \begin{axis}[
        x=0.05\textwidth,y=0.05\textwidth,
        axis y line=center,
        axis x line=middle,
        xlabel=$\Omega$,ylabel={\large $R_c({j\omega})$},
        xmin=-2.4,xmax=2.9,
        ymin=-0.8,ymax=2.4,
        xtick = {-1, 0, 1},
        xticklabels = {$-W$, $0$, $W$},
        ytick = {0, 1},
        yticklabels = {$0$,$1$},
        yticklabel style={yshift=0.3cm},
        ]
        \addplot[
        black,
        ultra thick
        ] coordinates {
            (-1,0) (0,1) (1,0)
        } ;
    \end{axis}
\end{tikzpicture}
        } & H(e^{e^{j\Omega}}) = \begin{cases}
            1 & |\Omega| \leq \Omega_0 \\
            0 & \text{en otro caso}
            \end{cases}
    \end{align*}
    y los conversores A/D y D/A son ideales. 

    \inciso Obtener $R(e^{j\Omega})$ y $X(e^{j\Omega})$.

    \inciso Determinar $\Omega_0$, $T_2$ y $\alpha$ tales que $y(n) = \alpha r_c(nT_2)$.

    \inciso Con el valor de $\Omega_0$ hallado en el punto anterior, obtener $T_3$ y $\beta$ tales que $s_c(t) = \beta r_c(t)$.
\end{ejercicio}

\begin{ejercicio}
    Sea el sistema de la figura:
    \begin{center}
        \begin{tikzpicture}[scale=1, transform shape]
    \node[rectangle, very thick, draw, minimum width=1cm, minimum height=1cm] (ad) at (0,0) {\large A/D};
    \node[rectangle, very thick, draw, minimum width=1cm, minimum height=1cm, xshift=2.5cm] (up2) at (ad) {\large $\uparrow 2$};
    \node[rectangle, very thick, draw, minimum width=1cm, minimum height=1cm, xshift=2.5cm] (h) at (up2) {$H(e^{j\Omega})$};
    \node[circle, very thick, draw, inner sep=0.05cm, xshift=2.5cm] (mult) at (h) {$\times$};
    \node[rectangle, very thick, draw, minimum width=1cm, minimum height=1cm, xshift=2.5cm] (da) at (mult) {\large D/A};

    \node[xshift=-2cm] (x_c) at (ad) {$x(t)$};
    \node[xshift=1.25cm,yshift=0.4cm] (x_n) at (ad) {$x(n)$};
    \node[xshift=1.1cm,yshift=0.4cm] (x_1) at (up2) {$x_1(n)$};
    \node[xshift=1.4cm,yshift=0.4cm] (x_2) at (h) {$x_2(n)$};
    \node[xshift=1cm,yshift=0.4cm] (y_n) at (mult) {$y(n)$};
    \node[xshift=2cm] (y_c) at (da) {$y(t)$};
    
    \node[yshift=-1.7cm] (cos) at (mult) {$\cos(\pi n)$};
    \node[yshift=-1.2cm,xshift=0.4cm] (t1) at (ad) {$T_1$};
    \node[yshift=-1.2cm,xshift=0.4cm] (t2) at (da) {$\frac{T}{2}$};

    \draw[->, very thick] (x_c.east) -- (ad.west);
    \draw[->, very thick] (ad.east) -- (up2.west);
    \draw[->, very thick] (up2.east) -- (h.west);
    \draw[->, very thick] (h.east) -- (mult.west);
    \draw[->, very thick] (mult.east) -- (da.west);
    \draw[->, very thick] (da.east) -- (y_c.west);
    \draw[->, very thick] (mult.south) ++ (0,-1cm) -- (mult.south) ;
    \draw[->, very thick] (ad.south) ++ (0,-1cm) -- (ad.south) ;
    \draw[->, very thick] (da.south) ++ (0,-1cm) -- (da.south) ;
\end{tikzpicture}
    \end{center}
    Tanto el conversor A/D como el D/A son ideales ($T = \frac{4}{3}\;\mathrm{s}$). Por otro lado, el espectro en frecuencias de $x(t)$ y la respuesta en frecuencia del sistema $H(e^{j\Omega})$ son los siguientes:
    \begin{align*}
        \parbox{0.5\textwidth}{
            \begin{center}
                \begin{tikzpicture}[scale=1,transform shape]
    \begin{axis}[
        x=0.05\textwidth,y=0.05\textwidth,
        axis y line=center,
        axis x line=middle,
        xlabel=$\Omega$,ylabel={\large $X(e^{j\Omega})$},
        xmin=-2.4,xmax=2.9,
        ymin=-0.8,ymax=2.4,
        xtick = {-2, -1, 0, 1, 2},
        xticklabels = {$-2\pi$, $-\pi$, $0$, $\pi$, $2\pi$},
        ytick = {0, 1},
        yticklabels = {$0$,$1$},
        yticklabel style={yshift=0.3cm},
        ]
        \addplot[
        black,
        ultra thick
        ] coordinates {
            (-1,0) (0,1) (1,0)
        } ;
    \end{axis}
\end{tikzpicture}
            \end{center}
        } & \hspace*{1em} & \parbox{0.5\textwidth}{
            \begin{center}
                \begin{tikzpicture}[scale=1,transform shape]
    \begin{axis}[
        x=0.05\textwidth,y=0.05\textwidth,
        axis y line=center,
        axis x line=middle,
        xlabel=$\Omega$,ylabel={\large $H(e^{j\Omega})$},
        xmin=-4.4,xmax=4.9,
        ymin=-0.8,ymax=2.4,
        xtick = {-4, -3, -2, -1, 0, 1, 2, 3, 4},
        xticklabels = {-$\pi$, $-\frac{3\pi}{4}$, $-\frac{\pi}{2}$, $-\frac{\pi}{4}$, $0$, $\frac{\pi}{4}$, $\frac{\pi}{2}$, $\frac{3\pi}{4}$, $\pi$},
        ytick = {0, 1},
        yticklabels = {$0$,$1$},
        yticklabel style={yshift=0.3cm},
        ]

        \addplot[
        black,
        ultra thick
        ] coordinates {
            (-2,0) (-2,1) (2,1) (2,0)
        } ;
    \end{axis}
\end{tikzpicture}
            \end{center}
        }
    \end{align*}        
    Dibujar los espectros de $x(n)$, $x_1(n)$, $x_2(n)$, $y(n)$ y $y(t)$.
\end{ejercicio}

        \newpage
        \myheader{Guía 7: Transformada Discreta de Fourier}

\begin{ejercicio}
\end{ejercicio}

\begin{ejercicio}
\end{ejercicio}

\begin{ejercicio}
\end{ejercicio}

\begin{ejercicio}
\end{ejercicio}

\begin{ejercicio}
\end{ejercicio}

\begin{ejercicio}
\end{ejercicio}

\begin{ejercicio}
\end{ejercicio}

\begin{ejercicio}
\end{ejercicio}

\begin{ejercicio}
\end{ejercicio}

\begin{ejercicio}
\end{ejercicio}

\begin{ejercicio}
\end{ejercicio}

\begin{ejercicio}
\end{ejercicio}

\begin{ejercicio}
\end{ejercicio}

\begin{ejercicio}
\end{ejercicio}

\begin{ejercicio}
\end{ejercicio}

\begin{ejercicio}
\end{ejercicio}

\begin{ejercicio}
\end{ejercicio}

\begin{ejercicio}
\end{ejercicio}

\begin{ejercicio}
\end{ejercicio}

\begin{ejercicio}
\end{ejercicio}

\begin{ejercicio}
\end{ejercicio}

\begin{ejercicio}
\end{ejercicio}

\begin{ejercicio}
\end{ejercicio}

\begin{ejercicio}
\end{ejercicio}

\begin{ejercicio}
\end{ejercicio}

\begin{ejercicio}
\end{ejercicio}

\begin{ejercicio}
\end{ejercicio}

\begin{ejercicio}
\end{ejercicio}

\begin{ejercicio}
\end{ejercicio}

\begin{ejercicio}
\end{ejercicio}
        \newpage
        \myheader{Guía 8: Transformada de Laplace y Z, Ecuaciones Diferenciales y en Diferencias (Parte II)}

\begin{ejercicio}
\end{ejercicio}

\begin{ejercicio}
\end{ejercicio}

\begin{ejercicio}
\end{ejercicio}

\begin{ejercicio}
\end{ejercicio}

\begin{ejercicio}
\end{ejercicio}

\begin{ejercicio}
\end{ejercicio}

\begin{ejercicio}
\end{ejercicio}

\begin{ejercicio}
\end{ejercicio}

\begin{ejercicio}
\end{ejercicio}

\begin{ejercicio}
\end{ejercicio}

\begin{ejercicio}
\end{ejercicio}

\begin{ejercicio}
\end{ejercicio}

\begin{ejercicio}
\end{ejercicio}

\begin{ejercicio}
\end{ejercicio}

\begin{ejercicio}
\end{ejercicio}

\begin{ejercicio}
\end{ejercicio}

\begin{ejercicio}
\end{ejercicio}

\begin{ejercicio}
\end{ejercicio}

\begin{ejercicio}
\end{ejercicio}

\begin{ejercicio}
\end{ejercicio}

\begin{ejercicio}
\end{ejercicio}

\begin{ejercicio}
\end{ejercicio}

\begin{ejercicio}
\end{ejercicio}

\begin{ejercicio}
\end{ejercicio}

\begin{ejercicio}
\end{ejercicio}

\begin{ejercicio}
\end{ejercicio}

\begin{ejercicio}
\end{ejercicio}

\begin{ejercicio}
\end{ejercicio}

\begin{ejercicio}
\end{ejercicio}

\begin{ejercicio}
\end{ejercicio}

        % \newpage
        % \myheader{Guía 9: Filtros Digitales, Sistemas de Fase Mínima y Pasa-Todo, Cuantificación}


\begin{ejercicio}
\end{ejercicio}

\begin{ejercicio}
\end{ejercicio}

\begin{ejercicio}
\end{ejercicio}

\begin{ejercicio}
\end{ejercicio}

\begin{ejercicio}
\end{ejercicio}

\begin{ejercicio}
\end{ejercicio}

\begin{ejercicio}
\end{ejercicio}

\begin{ejercicio}
\end{ejercicio}

\begin{ejercicio}
\end{ejercicio}

\begin{ejercicio}
\end{ejercicio}

\begin{ejercicio}
\end{ejercicio}

\begin{ejercicio}
\end{ejercicio}

\begin{ejercicio}
\end{ejercicio}

\begin{ejercicio}
\end{ejercicio}

\begin{ejercicio}
\end{ejercicio}

\begin{ejercicio}
\end{ejercicio}

\begin{ejercicio}
\end{ejercicio}

\begin{ejercicio}
\end{ejercicio}

\begin{ejercicio}
\end{ejercicio}

\begin{ejercicio}
\end{ejercicio}

\begin{ejercicio}
\end{ejercicio}

\begin{ejercicio}
\end{ejercicio}

\begin{ejercicio}
\end{ejercicio}

\begin{ejercicio}
\end{ejercicio}

\begin{ejercicio}
\end{ejercicio}

\begin{ejercicio}
\end{ejercicio}

\begin{ejercicio}
\end{ejercicio}

\begin{ejercicio}
\end{ejercicio}

\begin{ejercicio}
\end{ejercicio}

\begin{ejercicio}
\end{ejercicio}
    }
    \or \myheader{Guía 1: Señales en el tiempo}

\begin{ejercicio}
Graficar las siguientes señales en el intervalo $-10 \leq n \leq 10$. Implementar en \Keyboardsym una función para graficar cualquiera de ellas.

\inciso $-2\delta(n+2)$

\inciso $2^n u(n)$

\inciso $2^{-n} u(-n+2)$

\inciso $\cos\left(\frac{\pi}{3} n \right) u(n-2)$

\inciso $\sum_{k=-\infty}^{\infty} \delta(n-2k) - \delta(n-1-2k)$

\inciso $u(n+2) - u(n+2-N)$ con $N=3,5,7$


\end{ejercicio}

\begin{ejercicio}
Evaluar, si es posible, las siguientes sumas y expresar su respuesta en forma cartesiana y polar. Implementar una función en \Keyboardsym que obtenga la suma geométrica para cualquiera de ellas. 

\begin{align*}
\inciso & \sum_{n=0}^9 e^{j\frac{\pi}{2}n} &
\inciso & \sum_{n=-2}^7 e^{j\frac{\pi}{2}n} & \inciso & \sum_{n=0}^9 \cos\left(\frac{\pi}{2}n\right) & \hfill \\[.5em]
\inciso & \sum_{n=0}^9 \left(\frac{1}{2}\right)^n e^{j\frac{\pi}{2}n} & 
\inciso & \sum_{n=0}^\infty a^n e^{j\frac{\pi}{2}n},\; |a| < 1 & 
\inciso & \sum_{n=-4}^\infty \left(\frac{4}{3}\right)^n e^{j\frac{\pi}{2}n}
\end{align*}
\end{ejercicio}

\begin{ejercicio}
Sea $d_T(t)=t-T$ la función desplazamiento a derecha ($T>0$) y $e_a(t)=\frac{t}{a}$ la función expansión ($a > 1$). Para la señal $x(t)$

\begin{center}
    \begin{tikzpicture}[scale=0.6,transform shape]
    \begin{axis}[
    	axis y line=center,
    	axis x line=middle,
    	xlabel=$t$,ylabel={\Large $x(t)$},
    	xmin=-6,xmax=24,
    	ymin=-0.3,ymax=1.9,
    	xticklabel style = {xshift=5},
    	yticklabel style = {yshift=5},
    	]
    	\addplot[
    	black,
    	ultra thick
    	] coordinates {
    	    (0,0) (0,1) (10,1) (20,0)
    	};
    \end{axis}
\end{tikzpicture}

\end{center}

graficar:

\inciso $x(d_1(e_2(t)))$

\inciso $x(e_2(d_1(t)))$

\inciso $x\left(\frac{3}{2}t + 1\right)$

\inciso $x\left(-2t -1\right)$

\inciso $x\left(\frac{t}{2} - \frac{1}{2} \right)$

\end{ejercicio}

\begin{ejercicio}
Sea la señal discreta $x(n)$
\begin{center}
    \begin{tikzpicture}[scale=0.6,transform shape]
    \begin{axis}[
        x=0.04\textwidth,y=0.1\textwidth,
    	axis y line=center,
    	axis x line=middle,
    	xlabel=$n$,ylabel={\LARGE $x(n)$},
    	xmin=-6.9,xmax=6.9,
    	ymin=-0.3,ymax=3.5,
    	xticklabel style = {xshift=0},
    	yticklabel style = {yshift=5},
    	]
    	\discretedelta{-6}{0.1};
    	\discretedelta{-5}{0.1};
    	\discretedelta{-4}{0.1};
    	\discretedelta{-3}{0.1};
    	\discretedelta{-2}{1};
    	\discretedelta{-1}{2};
    	\discretedelta{0}{3};
    	\discretedelta{1}{2};
    	\discretedelta{2}{2};
    	\discretedelta{3}{1};
    	\discretedelta{4}{0.1};
    	\discretedelta{5}{0.1};
    	\discretedelta{6}{0.1};
    \end{axis}
\end{tikzpicture}
\end{center}

Graficar cada una de las siguientes señales. Implementar en \Keyboardsym una función que grafique cada una de ellas a partir de una $x(n)$ genérica.

\inciso $x(n) u(2-n)$

\inciso $x(n-1) \delta(n-3)$

\inciso $x(n-1) \delta(n-3)$

\inciso $\frac{1}{2}x(n) + \frac{1}{2}(-1)^n x(n)$

\end{ejercicio}

\begin{ejercicio}
Graficar las partes par e impar de las siguientes señales. Implementar en \Keyboardsym una función que obtenga la parte par e impar de una señal cualquiera. 

\begin{align*}
\inciso \parbox{.3\textwidth}{\begin{tikzpicture}[scale=0.6,transform shape]
    \begin{axis}[
        x=0.12\textwidth,y=0.12\textwidth,
    	axis y line=center,
    	axis x line=middle,
    	xlabel=$t$,ylabel={\Large $x(t)$},
    	xmin=-2.3,xmax=2.3,
    	ymin=-.2,ymax=1.3,
    	xticklabel style = {xshift=5},
    	yticklabel style = {yshift=5},
    	]
    	\addplot[
    	black,
    	ultra thick
    	] coordinates {
    	    (-2,0) (0,0) (1,1) (2,0)
    	} ;
    \end{axis}
\end{tikzpicture}} & \hfill & \inciso \parbox{.3\textwidth}{\begin{tikzpicture}[scale=0.6,transform shape]
    \begin{axis}[
        x=0.12\textwidth,y=0.12\textwidth,
    	axis y line=center,
    	axis x line=middle,
    	xlabel=$t$,ylabel={\Large $x(t)$},
    	xmin=-2.3,xmax=2.3,
    	ymin=-.2,ymax=1.3,
    	xticklabel style = {xshift=5},
    	yticklabel style = {yshift=5},
    	]
    	\addplot[
    	black,
    	ultra thick
    	] coordinates {
    	    (-2,0) (-1,1) (0,0) (1,1) 
    	    (2,1) (2,0)
    	} ;
    \end{axis}
\end{tikzpicture}}
\end{align*}
\end{ejercicio}

\begin{ejercicio}
Sea $x(t)$ una señal de tiempo continuo, y sean $y_1(t) = x(2 t)$ y $y_2(t) = x(t / 2)$. Determinar si las siguientes afirmaciones son verdaderas o falsas y justificar con demostración o contraejemplo según corresponda. Obtener el período fundamental de $y_1(t)$ y $y_2(t)$.

\inciso Si $x(t)$ es periódica, entonces $y_1(t)$ es periódica.

\inciso Si $y_1(t)$ es periódica, entonces $x(t)$ es periódica.

\inciso Si $x(t)$ es periódica, entonces $y_2(t)$ es periódica.

\inciso Si $y_2(t)$ es periódica, entonces $x(t)$ es periódica.

\end{ejercicio}

\begin{ejercicio}
Sea $x(n)$ una señal de tiempo discreta, y sean $y_1(n) = x(2 n)$ y 
\begin{equation*}
y_2(n) = \begin{cases}
x(n/2) & \mbox{si $n$ es par} \\
0 & \mbox{en otro caso}
\end{cases}
\end{equation*}
Determinar si las siguientes afirmaciones son verdaderas o falsas y justificar con demostración o contraejemplo según corresponda. Obtener el período fundamental de $y_1(n)$ y $y_2(n)$.

\inciso Si $x(n)$ es periódica, entonces $y_1(n)$ es periódica.

\inciso Si $y_1(n)$ es periódica, entonces $x(n)$ es periódica.

\inciso Si $x(n)$ es periódica, entonces $y_2(n)$ es periódica.

\inciso Si $y_2(n)$ es periódica, entonces $x(n)$ es periódica.

\end{ejercicio}

\begin{ejercicio}
Implementar en \Keyboardsym \emph{sin utilizar bucles} una función que, dada una señal $x(n)$ y un $N_0\in \mathbb{Z}$, obtenga $y_1(n)=x(N_0n)$ y 
\begin{equation*}
y_2(n) = \begin{cases}
x(n/N_0) & \mbox{si $n$ es m\`{u}ltiplo de $N_0$ } \\
0 & \mbox{en otro caso}
\end{cases}
\end{equation*}
\end{ejercicio}

\begin{ejercicio}
Sea $x(t)=e^{j\omega_0t}$ la señal exponencial compleja en tiempo continuo con período fundamental $T_0=\frac{2\pi}{\omega_0}$. Considere la señal discreta obtenida al tomar muestras de $x(t)$ equiespaciadas:
\begin{equation*}
    x_d(n) = x(nT_s) = e^{j\omega_0t}\big|_{t=nT_s} = e^{j\omega_0nT_s}
\end{equation*}

\inciso Demostrar que $x_d(n)$ es periódica si y sólo si $\frac{T_0}{T_s}$ es un número racional.

\inciso Determinar el período y la frecuencia fundamental de $x_d(n)$ cuando ésta es periódica. Expresar la frecuencia fundamental como una fracción de $T_0$.

\inciso ¿Cuántos periodos de $T_0$ se necesitan para obtener las muestras que forman un solo período de $x_d(n)$ en el caso en que ésta última es periódica?
\end{ejercicio}

\begin{ejercicio}
Indicar si las siguientes afirmaciones son verdaderas o falsas:

\inciso La suma de dos señales senoidales de tiempo continuo de frecuencias $f_1$ y $f_2$ es siempre una señal periódica. 

\inciso La suma de dos señales senoidales de tiempo discreto de frecuencias $f_1$ y $f_2$ es siempre una señal periódica. 

\end{ejercicio}

\begin{ejercicio}
Graficar en \Keyboardsym las siguientes funciones y determinar si son periódicas o no:
\begin{align*}
    \inciso & \cos\left(\frac{2\pi}{12} n\right) & \inciso & \cos\left(\frac{8\pi}{31} n\right) \\ 
    \inciso & \cos\left(\frac{1}{6} n\right) & \inciso & \parbox{.5\textwidth}{$x(n)$ definida como las muestras de una senoidal de frecuencia 10Hz muestreada con $T_s=\frac{1}{1000\mathrm{Hz}}$} \\
    \inciso & x(n) = \sum_{k=0}^{50}\cos\left(\frac{\pi k}{32}n\right) & \inciso & x(n) = \sum_{k=0}^{50} \Realpart{a_ke^{\frac{\pi k}{32}n}}\; \mathrm{con}\; a_k = \frac{1}{\pi k} \sin\left(\frac{\pi}{4} k\right)  \\ 
    \inciso & x(n) = \sum_{k=0}^{50}\cos\left(\frac{\pi f(k)}{2}n\right)\; \mathrm{con} & & \hspace{-2.5em} f(k) = 10 \tan\left(\frac{3\pi}{400} k\right)
\end{align*}
\end{ejercicio}

    \or \myheader{Guía 2: Caracterización de un Sistema}

\begin{ejercicio}
    Para el sistema en tiempo continuo definido por 
    \begin{align*}
        \inciso y(t) = e^{-t} x(t) & \hfill & \inciso y(t) = 2x(t) + \cos(2t)
    \end{align*}
    se pide:

    \subinciso Determinar si el sistema es lineal.

    \subinciso Encontrar la respuesta del sistema a $x(t)=\delta(t-t_0)$ para cualquier $t_0\in \mathbb{R}$. ¿El sistema es LTI?
    
    \subinciso Decidir si es posible obtener cualquier respuesta del sistema a partir de la respuesta a $x(t)=\delta(t-t_0)$.
\end{ejercicio}
    
\begin{ejercicio}
    Dar, si es posible, un ejemplo de un sistema en tiempo continuo en donde conocer la respuesta a $x(n)=\delta(n-n_0)$ para todo $n_0\in \Z$ no sea suficiente para encontrar la salida del sistema para cualquier $x(n)$.
\end{ejercicio}
    
\begin{ejercicio}
    Para los siguientes sistemas de tiempo continuo
    \begin{align*}
        \inciso y(t) = x(t/3) & \hfill & \inciso y(t) = \int_{-\infty}^{t} x(\tau + 2) d\tau & \hfill & \inciso y(t) = 2x(t) + 1
    \end{align*}
    determinar si cada uno de ellos es lineal, invariante ante desplazamientos, estable, causal y si tiene memoria.
\end{ejercicio}
    
\begin{ejercicio}
    Para los siguientes sistemas de tiempo discreto
    \begin{align*}
        \inciso & y(n) = x(-n) & \hfill \inciso & y(n) = \begin{cases}
        1 & \mbox{para $n = 0$} \\
        \pi x(n) & \mbox{en otro caso}
        \end{cases} \\[.5em]
        \inciso & y(n) = n x(n) & \hfill \inciso & y(n) = \Realpart{x(n)}
    \end{align*}
    determinar si cada uno de ellos es lineal, invariante ante desplazamientos, estable, causal y si tiene memoria.
\end{ejercicio}
    
\begin{ejercicio}
    Calcular $x(n) * x(n)$ con $x(n)=u(n+3) - u(n-4)$.
\end{ejercicio}
    
\begin{ejercicio}
    Dado un sistema LTI cuya respuesta impulsiva está dada por un pulso triangular $h(n)=\delta(n+2)+2\delta(n+1)+3\delta(t)+2\delta(t-1)+\delta(t-2)$ al cual se le aplica una entrada $x(n)$ que consiste en un tren de impulsos de período $N$, calcular y graficar la salida $y(n)$ para los siguientes casos:
    \begin{align*}
    \inciso N=6 & \hfill & \inciso N=4 & \hfill & \inciso N=2
    \end{align*}
\end{ejercicio}
    
\begin{ejercicio}
    A un sistema LTI cuya respuesta al impulso es $h(t) = u(t) - u(t - 1)$ se le aplica una entrada $x(t) = h(t/\alpha)$.
    
    \inciso Calcular la salida del sistema
    
    \inciso Si se sabe que la derivada de la salida tiene sólo 3 discontinuidades, obtener el valor de $\alpha$.
\end{ejercicio}

\begin{ejercicio}
    Dado un sistema LTI en tiempo discreto con una respuesta al impulso $h(n) = \alpha^n u(n)$ con $\alpha<1$, encontrar la salida del sistema para cada una de las siguientes entradas:
    
    \inciso $x(n) = \delta(n) - \delta(n-1)$
    
    \inciso $x(n) = u(n) - u(n-5)$
\end{ejercicio}
    
\begin{ejercicio}
    Dado un sistema LTI en tiempo continuo, encontrar la salida del sistema cuando la entrada es $x(t) = u(t-1)\sin(t)$ si la respuesta al impulso es:
    
    \inciso $h(t) = u(n)$ 
    
    \inciso $h(t) = u(t) - 2u(t-2) + u(t-5)$
\end{ejercicio}

\begin{ejercicio}
    Sobre un único sistema invariante en el tiempo se conocen los siguientes pares entrada-salida:
    \begin{align*}
        x_1(n) = \delta(n) + 2\delta(n-2) & \hspace{1em}\longrightarrow\hspace{1em} y_1(n) = \delta(n-1) + 2\delta(n-2) \\[.5em]
        x_2(n) = 3 \delta(n-2) & \hspace{1em}\longrightarrow\hspace{1em} y_2(n) = \delta(n-1) + 2 \delta(n-3) \\[.5em]
        x_3(n) = \delta(n-3) & \hspace{1em}\longrightarrow\hspace{1em} y_3(n) = \delta(n+1) + 2 \delta(n) + \delta(n-1)
    \end{align*}
    
    \inciso ¿Se puede afirmar algo sobre la linearidad del sistema?
    
    \inciso ¿Es posible hallar la respuesta del sistema $y_4(n)$ cuando la entrada es $x_4(n)=\delta(n)$ con los datos disponibles? En tal caso, obtenerla.
    
    \inciso ¿Es posible hallar la respuesta del sistema $y_5(n)$ cuando la entrada es $x_5(n)=5\delta(n-2)$ con los datos disponibles? En tal caso, obtenerla.
\end{ejercicio}

\begin{ejercicio}
    Determinar si los siguientes enunciados son verdaderos o falsos.
    
    \inciso La conexión en cascada de sistemas LTI resulta en un sistema total que también es LTI.
    
    \inciso La conexión en cascada de sistemas no lineales es un sistema no lineal.
    
    \inciso La conexión en cascada de sistemas no invariantes en el tiempo es un sistema no invariante en el tiempo.
    
    \inciso La conexión en cascada de sistemas causales con sistemas causales es siempre no causal.
    
    \inciso El orden de conexión de sistemas no invariantes en el tiempo no altera la salida para una misma entrada.
    
    \inciso En un sistema LTI si la entrada es periódica entonces la salida también lo es.
\end{ejercicio}
    
\begin{ejercicio}
    Dos sistemas LTI en tiempo discreto con respuesta al impulso $h_1(n)$ y $h_2(n)$ son conectados en cascada en ese orden. La entrada no se conoce pero la salida $y(n)$ es como se muestra en la siguiente figura:
    \begin{center}
        \begin{tikzpicture}[scale=0.6,transform shape]
    \begin{axis}[
        x=0.1\textwidth,y=0.1\textwidth,
    	axis y line=center,
    	axis x line=middle,
    	xlabel=$n$,ylabel={\LARGE $x(n)$},
    	xmin=-7.5,xmax=7.5,
    	ymin=-1.3,ymax=2.9,
    	xticklabel style = {xshift=0},
    	yticklabel style = {yshift=5},
    	]
    	\discretedelta{-7}{0.1};
    	\discretedelta{-6}{0.1};
    	\discretedelta{-5}{0.1};
    	\discretedelta{-4}{0.1};
    	\discretedelta{-3}{0.1};
    	\discretedelta{-2}{0.1};
    	\discretedelta{-1}{1};
    	\discretedelta{0}{2};
    	\discretedelta{1}{-1};
    	\discretedelta{2}{1};
    	\discretedelta{3}{0.1};
    	\discretedelta{4}{0.1};
    	\discretedelta{5}{0.1};
    	\discretedelta{6}{0.1};
    	\discretedelta{7}{0.1};
    \end{axis}
\end{tikzpicture}
    \end{center}
    
    \inciso Si los dos sistemas son causales, ¿qué se puede decir acerca del momento en que la entrada podría haber empezado? ¿Se puede establecer el momento exacto de comienzo?
    
    \inciso La entrada $x(n)$ que produjo la salida $y(n)$ anterior es aplicada a un nuevo par de sistemas conectados en cascada donde el primero tiene una respuesta impulsiva $h_a(n) = h_1(n + 1)$ y el segundo $h_b(n) = 2h_2(n)$. Graficar la salida.
\end{ejercicio}
    
\begin{ejercicio}
    En el sistema de la figura se sabe que $\gamma \in \Z_{\geq 0}$ y $\beta, \alpha_1, \alpha_2 \in \R$ con $\alpha_1 \neq \alpha_2$.
    \begin{center}
        \begin{tikzpicture}
    \node[circle,draw,thick,inner sep=0.05cm] (plus) at (0,0) {\Large +} ;
    \node[rectangle,draw,thick,inner sep=0.4cm] (h1) at (-4,1) {$h_1(n)=\alpha_1^n u(n)$} ;
    \node[rectangle,draw,thick,inner sep=0.4cm] (h2) at (-4,-1) {$h_2(n)=\beta \delta(n-\gamma)$} ;
    \node[rectangle,draw,thick,inner sep=0.4cm] (h3) at (4,0) {$h_3(n)=\alpha^n u(n)$} ;
    \node[xshift=-4cm] (x_n) at (h1) {$x(n)$} ;
    \node[xshift=-3cm,inner sep=-0.1] (x_n_arrow_cross) at (h1) {} ;

    \node[xshift=3cm,yshift=.3cm] (y_1) at (h1) {$y_1(n)$} ;
    \node[xshift=3cm,yshift=.3cm] (y_2) at (h2) {$y_2(n)$} ;
    \node[xshift=-2.5cm,yshift=-.4cm] (y_3) at (h3) {$y_3(n)$} ;
    \node[xshift=3cm,yshift=.3cm] (y_n) at (h3) {$y(n)$} ;

    \draw[->, very thick] (x_n) -- (h1.west) ;
    \draw[->, very thick] (x_n_arrow_cross) |- (h2.west) ;
    \draw[->, very thick] (h1.east) -| (plus.north) ;
    \draw[->, very thick] (h2.east) -| (plus.south) ;
    \draw[->, very thick] (plus.east) -- (h3.west) ;
    \draw[->, very thick] (h3.east) -- ++(2,0) ;
\end{tikzpicture}
    \end{center}

    \inciso Determinar $h(n)$ tal que $y(n) = h(n) * x(n)$.

    \inciso Determinar las condiciones que deben cumplir los parámetros $\alpha_1, \alpha_2, \beta, \gamma$ para que el sistema sea estable.
\end{ejercicio}

\begin{ejercicio}
    Dado un sistema LTI con $h(t) = u(t + T/2) - u(t - T/2)$ donde $T>0$, se desea analizar cómo el sistema opera sobre señales periódicas.
    
    \inciso Demostrar que para cualquier señal periódica de período $T$ la salida es constante para todo $t$.

    \inciso Determinar el valor de la constante si se sabe que la señal periódica es impar.

    \inciso Determinar si existen señales periódicas de período $T'\neq T$ para las cuales los resultados anteriores siguen siendo ciertos.
\end{ejercicio}

\begin{ejercicio}
    Sea un sistema LTI y una señal $x(n)$ que satisface $x(n)=\delta(n) + \sum_{k=1}^M a_k x(n-k)$ donde $a_k$ son valores reales. Sea $y(n)$ la salida del sistema para la entrada $x(n)$.

    \inciso Determinar la ecuación en diferencias de $h(n)$ en función de $y(n)$ usando las propiedades básicas de los sitemas LTI.
    
    \inciso Si $y(n)$ es de duración finita, ¿qué se puede decir sobre la estabilidad del sistema?

    \inciso Si $y(n)$ es de duración infinita, obtener algún tipo de condición sobre $y(n)$ que asegure la estabilidad del sistema.
\end{ejercicio}

\begin{ejercicio}
    Sea un sistema LTI de tiempo discreto tal que la relación entre la salida y la entrada del mismo se puede escribir como:
    \begin{equation*}
        y(n) = \sum_{k=0}^{\infty} \alpha^{n-k} \left(x(k) + \beta x(k-1) + \gamma x(k-2)\right)
    \end{equation*}
    donde $|\alpha| < 1$ y $\beta, \gamma \in \R$.

    \inciso Obtener la respuesta al impulso del sistema y graficarla.
    \inciso Analizar la estabilidad del mismo.
    \inciso Determinar los valores de $\beta$ y $\gamma$ para que el sistema tenga las siguientes propiedades:
    \begin{itemize}
        \item Si $x(n) = (-1)^n$ para todo $n$ entonces $y(n) = 0$ para todo $n$.
        \item Si $x(n) = 1$ para todo $n$ entonces $y(n) = 1$ para todo $n$.
    \end{itemize}
\end{ejercicio}

\begin{ejercicio}
    Dado un sistema en tiempo discreto definido por la ecuación en diferencias 
    \begin{equation*}
        y(n) = x(n) + \frac{3}{4} y(n-1)
    \end{equation*}
    con condiciones de contorno
    \begin{align*}
    \inciso & y(-1) = 0 & \inciso & y(1) = 0 & \inciso & y(0) = 0 \\[.5em]
    \inciso & y(0) = 1 & \inciso & \lim_{n\rightarrow -\infty} y(n) = 0  & \inciso & \lim_{n\rightarrow +\infty} y(n) = 0
    \end{align*}
    se pide:
    \begin{itemize}
        \item Calcular el valor de la respuesta al impulso $h(n)$ del sistema en el intervalo $-5 \leq n \leq 5$
        \item Obtener una expresión cerrada de $h(n)$ para todo $n \in \mathbb{Z}$.
        \item Determinar si el sistema lineales, invariante ante desplazamientos, estables y causales.
    \end{itemize}
\end{ejercicio}
    
\begin{ejercicio}

    Dado un sistema en tiempo discreto definido por la ecuación en diferencias 
    \begin{equation*}
        y(n) = x(n) + \frac{1}{2} y(n-1)
    \end{equation*}
    con condiciones
    
    \inciso iniciales de reposo

    \inciso finales de reposo

    \noindent se pide:
    \begin{itemize}
        \item Determinar si el sistema lineales, invariante ante desplazamientos, estables y causales. Establecer cómo se relacionan las condiciones de contorno con cada una de las propiedades del sistema.
        \item Calcular el valor de la respuesta a las señales $\delta(n-1)$ y $\delta(n+1)$ del sistema en el intervalo $-5 \leq n \leq 5$.
        \item Obtener una expresión cerrada de la respuesta a las señales $\delta(n-1)$ y $\delta(n+1)$ para todo $n \in \mathbb{Z}$.
    \end{itemize}
\end{ejercicio}
    
\begin{ejercicio}
    Demostrar que un sistema definido por ecuaciones en diferencias FIR siempre será lineal, invariante ante desplazamientos y estable. Determinar, además, qué condición debe cumplir un sistema de este tipo para ser causal.
\end{ejercicio}
    
\begin{ejercicio}
    Obtener la respuesta al impulso para el sistema definido por la ecuación diferencial 
    \begin{equation*}
        y(n) = x(n+1) + x(n) + x(n-1) + \frac{3}{4} y(n-1)
    \end{equation*}
    en condiciones inciales de reposo. 
\end{ejercicio}
    
\begin{ejercicio}
    Obtener la respuesta al impulso para el sistema definido por la ecuación diferencial 
    \begin{equation*}
        y(n) = x(n+1) + x(n) + x(n-1) + \frac{5}{4} y(n-1)
    \end{equation*}
    en condiciones finales de reposo. 
\end{ejercicio}
    
\begin{ejercicio}
    Implementar en \Keyboardsym una función que permita obtener la respuesta al impulso de una ecuación en diferencias para condiciones iniciales o finales de reposo.
\end{ejercicio}
    
    
    
    \or \myheader{Guía 3: Series de Fourier}


\begin{ejercicio}
Sea $x(t)$ un tren de pulsos de período $T$ definido en el intervalo $[-\frac{T}{2},\frac{T}{2})$ como 
\begin{equation*}
    x(t) = \begin{cases}
    1 & \mbox{si } t \in (-\frac{T_0}{2},\frac{T_0}{2}) \\
    0 & \mbox{si } t \in [-\frac{T}{2},-\frac{T_0}{2}) \cup (\frac{T_0}{2},\frac{T}{2})
    \end{cases}
\end{equation*}
\end{ejercicio}
con $T > T_0 > 0$. 

\inciso Obtenga los coeficientes $\left\{a_k\right\}_{k\in\mathbb{Z}}$ de la Serie de Fourier de tiempo continuo de $x(t)$ y escriba los 6 primeros términos de la serie.

\inciso Grafique el espectro de aplitudes y de fase.

\inciso Obtenga los coeficientes $\left\{b_k\right\}_{k\in\mathbb{Z}}$ de la Serie de Fourier de tiempo continuo de $x_1(t) = x(t-\frac{\pi}{6})$.

\inciso Obtenga los coeficientes $\left\{c_k\right\}_{k\in\mathbb{Z}}$ de la Serie de Fourier de tiempo continuo de $x_2(t) = x(t)+1$.

\inciso Obtenga los coeficientes $\left\{d_k\right\}_{k\in\mathbb{Z}}$ de la Serie de Fourier de tiempo continuo de $x_3(t) = \sum_{n=-\infty}^{\infty} (-1)^{n-1} \delta(t-(2n+1)\frac{T_0}{2})$.

\begin{ejercicio}
Utilizando las propiedades de la serie de Fourier de tiempo continuo, calcular los coeficientes $\left\{b_k\right\}_{k\in\mathbb{Z}}$ de la señal periódica $x(t)$:
\begin{align*}
    \inciso & \parbox{.4\textwidth}{\begin{tikzpicture}[scale=0.6]
    \begin{axis}[
        x=.05\textwidth,y=0.08\textwidth,
    	axis y line=center,
    	axis x line=middle,
    	xlabel=$t$,ylabel={\Large $x(t)$},
    	xmin=-6.9,xmax=6.9,
    	ymin=-1.2,ymax=1.5,
    	xtick={-6,...,6},
    	xticklabel style = {xshift=0},
    	yticklabel style = {yshift=5}
	] 
	\addplot[
    	black,
    	ultra thick
    	] coordinates {
	(-5,-1) (-3,1) (-3,-1) (-1,1) (-1,-1) (1,1) (1,-1) (3,1) (3,-1) (5,1)
	} ;
	\node at (axis cs:6.5,0.5) {\Large $\cdots$} ;
	\node at (axis cs:-5.9,0.5) {\Large $\cdots$} ;
    \end{axis}
\end{tikzpicture}} & \hspace{\fill} 
    \inciso & \parbox{.4\textwidth}{\begin{tikzpicture}[scale=0.6]
    \begin{axis}[
        x=.05\textwidth,y=0.1\textwidth,
    	axis y line=center,
    	axis x line=middle,
    	xlabel=$t$,ylabel={\Large $x(t)$},
    	xmin=-6.9,xmax=6.9,
    	ymin=-0.2,ymax=1.4,
    	xtick={-6,...,6},
    	xticklabel style = {xshift=0},
    	yticklabel style = {yshift=5}
	] 
	\addplot[
    	black,
    	ultra thick
    	] coordinates {
	(-5, 1) (-4, 0) (-2,0) (-1,1) (1,1) (2,0) (4,0) (5,1)
	} ;
	\node at (axis cs:6.5,0.5) {\Large $\cdots$} ;
	\node at (axis cs:-5.9,0.5) {\Large $\cdots$} ;
    \end{axis}
\end{tikzpicture}} \\
    \inciso & \parbox{.4\textwidth}{\begin{tikzpicture}[scale=0.6]
    \begin{axis}[
        x=.05\textwidth,y=0.06\textwidth,
    	axis y line=center,
    	axis x line=middle,
    	xlabel=$t$,ylabel={\Large $x(t)$},
    	xmin=-6.9,xmax=6.9,
    	ymin=-0.2,ymax=2.5,
    	xtick={-6,...,6},
    	xticklabel style = {xshift=0},
    	yticklabel style = {yshift=5}
	] 
	\addplot[
    	black,
    	ultra thick
    	] coordinates {
	(-5,0) (-3,2) (-2,0) (0,2) (1,0) (3,2)
	(4,0) (5,1)
	} ;
	\node at (axis cs:6.5,1) {\Large $\cdots$} ;
	\node at (axis cs:-5.9,1) {\Large $\cdots$} ;
    \end{axis}
\end{tikzpicture}} & \hspace{\fill} 
    \inciso & \parbox{.4\textwidth}{\begin{tikzpicture}[scale=0.6]
    \begin{axis}[
        x=.05\textwidth,y=0.06\textwidth,
    	axis y line=center,
    	axis x line=middle,
    	xlabel=$t$,ylabel={\Large $x(t)$},
    	xmin=-6.9,xmax=6.9,
    	ymin=-2.2,ymax=1.9,
    	xtick={-6,...,6},
    	xticklabel style = {xshift=0},
    	yticklabel style = {yshift=5}
	] 
	\diracdelta{-5}{-2};
	\diracdelta{-4}{1};
	\diracdelta{-3}{-2};
	\diracdelta{-2}{1};
	\diracdelta{-1}{-2};
	\diracdelta{0}{1};
	\diracdelta{1}{-2};
	\diracdelta{2}{1};
	\diracdelta{3}{-2};
	\diracdelta{4}{1};
	\diracdelta{5}{-2};
	\node at (axis cs:6.5,.5) {\Large $\cdots$} ;
	\node at (axis cs:-5.9,.5) {\Large $\cdots$} ;
    \end{axis}
\end{tikzpicture}} \\
    \inciso & \parbox{.4\textwidth}{\begin{tikzpicture}[scale=0.6]
    \begin{axis}[
        x=.05\textwidth,y=0.06\textwidth,
    	axis y line=center,
    	axis x line=middle,
    	xlabel=$t$,ylabel={\Large $x(t)$},
    	xmin=-6.9,xmax=6.9,
    	ymin=-1.4,ymax=1.4,
    	xtick={-6,...,6},
    	xticklabel style = {xshift=0},
    	yticklabel style = {yshift=5}
	] 
	\addplot[
    	black,
    	ultra thick
    	] coordinates {
	(-5,0) (-5,-1) (-4,-1) (-4,0) (-2,0) (-2,1) (-1,1) (-1,0) (1,0) (1,-1) (2,-1) (2,0) (4,0) (4,1) (5,1) (5,0)
	} ;
	\node at (axis cs:6.5,0.5) {\Large $\cdots$} ;
	\node at (axis cs:-5.9,0.5) {\Large $\cdots$} ;
    \end{axis}
\end{tikzpicture}} & \hspace{\fill} 
    \inciso & \parbox{.4\textwidth}{\begin{tikzpicture}[scale=0.6]
    \begin{axis}[
        x=.05\textwidth,y=0.06\textwidth,
    	axis y line=center,
    	axis x line=middle,
    	xlabel=$t$,ylabel={\Large $x(t)$},
    	xmin=-6.9,xmax=6.9,
    	ymin=-0.2,ymax=2.9,
    	xtick={-6,...,6},
    	xticklabel style = {xshift=0},
    	yticklabel style = {yshift=5}
	] 
	\addplot[
    	black,
    	ultra thick
    	] coordinates {
	(-5,1) (-4,1) (-4,0) (-3,0) (-3,2) (-2,2) (-2,1) (-1,1) (-1,0) (0,0) (0,2) (1,2) (1,1) (2,1) (2,0) (3,0) (3,2) (4,2) (4,1) (5,1) (5,0)
	} ;
	\node at (axis cs:6.5,1) {\Large $\cdots$} ;
	\node at (axis cs:-5.9,1) {\Large $\cdots$} ;
    \end{axis}
\end{tikzpicture}} \\
    \inciso & \parbox{.4\textwidth}{\begin{tikzpicture}[scale=0.6,transform shape]
    \begin{axis}[
        x=0.036\textwidth,y=0.12\textwidth,
    	axis y line=center,
    	axis x line=middle,
    	xlabel=$t$,ylabel={\Large $x(t)$},
    	xmin=-21,xmax=21,
    	ymin=-0.4,ymax=1.4,
    	xticklabel style = {xshift=5},
    	yticklabel style = {yshift=5},
    	]
    	\addplot[
    	black,
    	ultra thick
    	] coordinates {
    	    (-15,0) (-12,1) 
    	    (-9,0) (-6,1)
    	    (-3,0) (0,1)
    	    (3,0) (6,1)
    	    (9,0) (12,1)
    	    (15,0)
    	};
    	\node at (18,0.5) {\Huge $\cdots$} ;
    	\node at (-18,0.5) {\Huge $\cdots$} ;
    \end{axis}
\end{tikzpicture}} & \hspace{\fill} & \\
\end{align*}
\end{ejercicio}

\begin{ejercicio}
Calcular los coeficientes de la serie de Fourier de tiempo continuo de la señal $x(t)$ de período $T=3$ sabiendo que en el intervalo $[-1.5,1.5)$ se define como

\inciso $x(t) = \cos(20\pi t)w_1(t)$ con \begin{equation*}
    w_1(t) = \begin{cases}
    1 & \mbox{si } t \in (-1,1) \\
    0 & \mbox{si } t \in [-1.5,\,-1) \cup (1,1.5)
    \end{cases}
\end{equation*}

\inciso $x(t) = \cos(20\pi t)w_2(t)$ con \begin{equation*}
    w_2(t) = \begin{cases}
    1 & \mbox{si } t \in (-1.5,-1) \\
    0 & \mbox{si } t \in [-1,0) \\
    2 & \mbox{si } t \in (0,1) \\
    1 & \mbox{si } t \in (1,1.5) \\
    \end{cases}
\end{equation*}

\inciso $x(t) = \cos(20\pi t)w_3(t)$ con \begin{equation*}
    w_3(t) = \begin{cases}
    e^{-|t|} & \mbox{si } t \in (-1,1) \\
    0 & \mbox{si } t \in [-1.5,-1) \cup (1,1.5)
    \end{cases}
\end{equation*}
\end{ejercicio}


\begin{ejercicio}
Utilizando las propiedades de la serie de Fourier de tiempo discreto, calcular los coeficientes $\left\{a_k\right\}_{k=0}^{N}$ de la señal periódica $x(n)$:
\begin{align*}
    \inciso & \parbox{.4\textwidth}{\begin{tikzpicture}[scale=0.6,transform shape]
    \begin{axis}[
        x=0.035\textwidth,y=0.1\textwidth,
    	axis y line=center,
    	axis x line=middle,
    	xlabel=$n$,ylabel={\LARGE $x(n)$},
    	xmin=-9.9,xmax=9.9,
    	ymin=-0.3,ymax=1.9,
    	xticklabel style = {xshift=0},
    	yticklabel style = {yshift=5},
    	]
    	\discretedelta{-8}{0.1};
    	\discretedelta{-7}{1};
    	\discretedelta{-6}{1};
    	\discretedelta{-5}{1};
    	\discretedelta{-4}{1};
    	\discretedelta{-3}{1};
    	\discretedelta{-2}{0.1};
    	\discretedelta{-1}{0.1};
    	\discretedelta{0}{1};
    	\discretedelta{1}{1};
    	\discretedelta{2}{1};
    	\discretedelta{3}{1};
    	\discretedelta{4}{1};
    	\discretedelta{5}{0.1};
    	\discretedelta{6}{0.1};
    	\discretedelta{7}{1};
    	\discretedelta{8}{1};
    	\node at (-9,0.5) {\Large $\cdots$};
    	\node at (9,0.5) {\Large $\cdots$};
    \end{axis}
\end{tikzpicture}} & \hspace{\fill} 
    \inciso & \parbox{.4\textwidth}{\begin{tikzpicture}[scale=0.6,transform shape]
    \begin{axis}[
        x=0.035\textwidth,y=0.1\textwidth,
    	axis y line=center,
    	axis x line=middle,
    	xlabel=$n$,ylabel={\LARGE $x(n)$},
    	xmin=-9.9,xmax=9.9,
    	ymin=-0.3,ymax=1.9,
    	xticklabel style = {xshift=0},
    	yticklabel style = {yshift=5},
    	]
    	\discretedelta{-8}{0.1};
    	\discretedelta{-7}{0.1};
    	\discretedelta{-6}{1};
    	\discretedelta{-5}{1};
    	\discretedelta{-4}{1};
    	\discretedelta{-3}{1};
    	\discretedelta{-2}{0.1};
    	\discretedelta{-1}{0.1};
    	\discretedelta{0}{1};
    	\discretedelta{1}{1};
    	\discretedelta{2}{1};
    	\discretedelta{3}{1};
    	\discretedelta{4}{0.1};
    	\discretedelta{5}{0.1};
    	\discretedelta{6}{1};
    	\discretedelta{7}{1};
    	\discretedelta{8}{1};
    	\node at (-9,0.5) {\Large $\cdots$};
    	\node at (9,0.5) {\Large $\cdots$};
    \end{axis}
\end{tikzpicture}} \\
    \inciso & \parbox{.4\textwidth}{\begin{tikzpicture}[scale=0.6,transform shape]
    \begin{axis}[
        x=0.035\textwidth,y=0.08\textwidth,
    	axis y line=center,
    	axis x line=middle,
    	xlabel=$n$,ylabel={\LARGE $x(n)$},
    	xmin=-9.9,xmax=9.9,
    	ymin=-1.4,ymax=2.9,
    	xticklabel style = {xshift=0},
    	yticklabel style = {yshift=5},
    	]
    	\discretedelta{-8}{-1};
    	\discretedelta{-7}{2};
    	\discretedelta{-6}{1};
    	\discretedelta{-5}{2};
    	\discretedelta{-4}{-1};
    	\discretedelta{-3}{0.1};
    	\discretedelta{-2}{-1};
    	\discretedelta{-1}{2};
    	\discretedelta{0}{1};
    	\discretedelta{1}{2};
    	\discretedelta{2}{-1};
    	\discretedelta{3}{0.1};
    	\discretedelta{4}{-1};
    	\discretedelta{5}{2};
    	\discretedelta{6}{1};
    	\discretedelta{7}{2};
    	\discretedelta{8}{-1};
    	\node at (-9,0.5) {\Large $\cdots$};
    	\node at (9,0.5) {\Large $\cdots$};
    \end{axis}
\end{tikzpicture}} & \hspace{\fill} 
    \inciso & x(n) = \sin(2\pi n / 3) \cos(\pi n/2)
\end{align*}

\noindent \hspace*{0.6em} \inciso $x(n)$ periódica con período 4, siendo $x(n) = 1-\sin(\pi n/4)$ para $0 \leq n \leq 3$ 

\noindent \hspace*{0.6em} \inciso $x(n)$ periódica con período 12, siendo $x(n) = 1-\sin(\pi n/4)$ para $0 \leq n \leq 11$
\end{ejercicio}

\begin{ejercicio}
Sea $x(n)$ una señal periódica de período $N$, hallar:
\begin{align*}
    & \inciso x(n-n_0),\; n_0 \in\mathbb{Z} & \hfill & \inciso x(n) - x(n-1) & \hfill & \inciso x(n) - x\left(n-\frac{N}{2}\right),\; \mbox{$N$ par} \\
    & \inciso x(n) + x\left(n-\frac{N}{2}\right),\; \mbox{$N$ par} & \hfill & \inciso x^*(-n) & \hfill & \inciso (-1)^n x(n),\; \mbox{$N$ par} \\ & \inciso (-1)^n x(n),\; \mbox{$N$ impar}
    & \hfill & \inciso y(n) = \begin{cases} x(n) & \mbox{si $n$ es par} \\ 0 & \mbox{en otro caso}
    \end{cases}
\end{align*}
\end{ejercicio}

\begin{ejercicio}
Sea un sistema LTI de tiempo continuo con respuesta al impulso $h(t) = e^{-4|t|}$. Hallar la salida $y(t)$ del sistema para las siguientes entradas:

\inciso $x(t)=e^{j\omega_0 t}$ con $\omega_0\in\mathbb{R}$ ¿Es $x(t)$ una autofunción del sistema?

\inciso $x(t)=e^{j\omega_0 t}u(t)$ con $\omega_0\in\mathbb{R}$. ¿Es $x(t)$ una autofunción del sistema?

\inciso $x(t)=\sum_{k=-\infty}^{\infty} (-1)^k \delta(t-k)$

\inciso $x(t)$ de período 1 tal que $x(t) = \begin{cases} 1 & \mbox{si $-1/4 \leq t < 1/4$} \\ 0 & \mbox{en otro caso} \end{cases}$ \hspace{1em} para $-1/2\leq t < 1/2$.
\end{ejercicio}

\begin{ejercicio}
Para cada uno de los siguientes pares de señales $x(n)$ e $y(n)$, determinar si existe un sistema LTI discreto cuya salida $y(n)$ pueda corresponder efectivamente a la entrada $x(n)$. En el caso de que exista, determinar si es único y qué codiciones debe cumplir la respuesta al impulso.

\inciso $x(n) = \left(\frac{1}{2}\right)^n$ e $y(n) = \left(\frac{1}{4}\right)^n$

\inciso $x(n)=e^{jn/8}$ e $y(n) = e^{j2n/8}$

\inciso $x(n)= e^{j\pi n/3}$ e $y(n) = \cos(\pi n/3)$

\inciso $x(n) = \cos(\pi n/3)$ e $y(n) = \cos(\pi n/3) + \sqrt{3} \sin(\pi n/3)$
\end{ejercicio}

\begin{ejercicio}
Decidir si existe un sistema LTI discreto que cumpla con que si $x(n)$ es la entrada, entonces $y(n)$ es una salida posible.
\begin{align*}
    \inciso & \parbox{.4\textwidth}{
        \begin{tikzpicture}[scale=0.6,transform shape]
    \begin{axis}[
        x=0.022\textwidth,y=0.1\textwidth,
    	axis y line=center,
    	axis x line=middle,
    	xlabel=$n$,ylabel={\LARGE $x(n)$},
    	xmin=-15.9,xmax=15.9,
    	ymin=-0.3,ymax=1.9,
    	xticklabel style = {xshift=0},
    	yticklabel style = {yshift=5}
    	]
    	\discretedelta{-14}{0.1};
    	\discretedelta{-13}{1};
    	\discretedelta{-12}{1};
    	\discretedelta{-11}{1};
    	\discretedelta{-10}{0.1};
    	\discretedelta{-9}{0.1};
    	\discretedelta{-8}{0.1};
    	\discretedelta{-7}{0.1};
    	\discretedelta{-6}{0.1};
    	\discretedelta{-5}{0.1};
    	\discretedelta{-4}{0.1};
    	\discretedelta{-3}{0.1};
    	\discretedelta{-2}{0.1};
    	\discretedelta{-1}{1};
    	\discretedelta{0}{1};
    	\discretedelta{1}{1};
    	\discretedelta{2}{0.1};
    	\discretedelta{3}{0.1};
    	\discretedelta{4}{0.1};
    	\discretedelta{5}{0.1};
    	\discretedelta{6}{0.1};
    	\discretedelta{7}{0.1};
    	\discretedelta{8}{0.1};
    	\discretedelta{9}{0.1};
    	\discretedelta{10}{0.1};
    	\discretedelta{11}{1};
    	\discretedelta{12}{1};
    	\discretedelta{13}{1};
    	\discretedelta{14}{0.1};
    	\node at (-15,0.5) {\Large $\cdots$};
    	\node at (15,0.5) {\Large $\cdots$};
    \end{axis}
\end{tikzpicture}
    } 
    & \hspace{\fill} 
    & \parbox{.4\textwidth}{
        \begin{tikzpicture}[scale=0.6,transform shape]
    \begin{axis}[
        x=0.022\textwidth,y=0.1\textwidth,
    	axis y line=center,
    	axis x line=middle,
    	xlabel=$n$,ylabel={\LARGE $y(n)$},
    	xmin=-15.9,xmax=15.9,
    	ymin=-0.3,ymax=1.9,
    	xticklabel style = {xshift=0},
    	yticklabel style = {yshift=5}
    	]
    	\discretedelta{-14}{1};
    	\discretedelta{-13}{0.1};
    	\discretedelta{-12}{0.1};
    	\discretedelta{-11}{0.1};
    	\discretedelta{-10}{1};
    	\discretedelta{-9}{1};
    	\discretedelta{-8}{1};
    	\discretedelta{-7}{0.1};
    	\discretedelta{-6}{0.1};
    	\discretedelta{-5}{0.1};
    	\discretedelta{-4}{1};
    	\discretedelta{-3}{1};
    	\discretedelta{-2}{1};
    	\discretedelta{-1}{0.1};
    	\discretedelta{0}{0.1};
    	\discretedelta{1}{0.1};
    	\discretedelta{2}{1};
    	\discretedelta{3}{1};
    	\discretedelta{4}{1};
    	\discretedelta{5}{0.1};
    	\discretedelta{6}{0.1};
    	\discretedelta{7}{0.1};
    	\discretedelta{8}{1};
    	\discretedelta{9}{1};
    	\discretedelta{10}{1};
    	\discretedelta{11}{0.1};
    	\discretedelta{12}{0.1};
    	\discretedelta{13}{0.1};
    	\discretedelta{14}{1};
    	\node at (-16,0.5) {\Large $\cdots$};
    	\node at (16,0.5) {\Large $\cdots$};
    \end{axis}
\end{tikzpicture}
    } \\
    \inciso & \parbox{.4\textwidth}{
        \begin{tikzpicture}[scale=0.6,transform shape]
    \begin{axis}[
        x=0.022\textwidth,y=0.1\textwidth,
    	axis y line=center,
    	axis x line=middle,
    	xlabel=$n$,ylabel={\LARGE $x(n)$},
    	xmin=-15.9,xmax=15.9,
    	ymin=-0.3,ymax=1.9,
    	xticklabel style = {xshift=0},
    	yticklabel style = {yshift=5}
    	]
    	\discretedelta{-14}{0.1};
    	\discretedelta{-13}{1};
    	\discretedelta{-12}{1};
    	\discretedelta{-11}{1};
    	\discretedelta{-10}{0.1};
    	\discretedelta{-9}{0.1};
    	\discretedelta{-8}{0.1};
    	\discretedelta{-7}{0.1};
    	\discretedelta{-6}{0.1};
    	\discretedelta{-5}{0.1};
    	\discretedelta{-4}{0.1};
    	\discretedelta{-3}{0.1};
    	\discretedelta{-2}{0.1};
    	\discretedelta{-1}{1};
    	\discretedelta{0}{1};
    	\discretedelta{1}{1};
    	\discretedelta{2}{0.1};
    	\discretedelta{3}{0.1};
    	\discretedelta{4}{0.1};
    	\discretedelta{5}{0.1};
    	\discretedelta{6}{0.1};
    	\discretedelta{7}{0.1};
    	\discretedelta{8}{0.1};
    	\discretedelta{9}{0.1};
    	\discretedelta{10}{0.1};
    	\discretedelta{11}{1};
    	\discretedelta{12}{1};
    	\discretedelta{13}{1};
    	\discretedelta{14}{0.1};
    	\node at (-15,0.5) {\Large $\cdots$};
    	\node at (15,0.5) {\Large $\cdots$};
    \end{axis}
\end{tikzpicture}
    } & \hspace{\fill} 
    & \parbox{.4\textwidth}{
        \begin{tikzpicture}[scale=0.6,transform shape]
    \begin{axis}[
        x=0.022\textwidth,y=0.1\textwidth,
    	axis y line=center,
    	axis x line=middle,
    	xlabel=$n$,ylabel={\LARGE $y(n)$},
    	xmin=-15.9,xmax=15.9,
    	ymin=-0.3,ymax=1.9,
    	xticklabel style = {xshift=0},
    	yticklabel style = {yshift=5}
    	]
    	\discretedelta{-14}{0.1};
    	\discretedelta{-13}{0.1};
    	\discretedelta{-12}{0.1};
    	\discretedelta{-11}{0.1};
    	\discretedelta{-10}{1};
    	\discretedelta{-9}{1};
    	\discretedelta{-8}{1};
    	\discretedelta{-7}{0.1};
    	\discretedelta{-6}{0.1};
    	\discretedelta{-5}{0.1};
    	\discretedelta{-4}{0.1};
    	\discretedelta{-3}{0.1};
    	\discretedelta{-2}{0.1};
    	\discretedelta{-1}{1};
    	\discretedelta{0}{1};
    	\discretedelta{1}{1};
    	\discretedelta{2}{0.1};
    	\discretedelta{3}{0.1};
    	\discretedelta{4}{0.1};
    	\discretedelta{5}{0.1};
    	\discretedelta{6}{0.1};
    	\discretedelta{7}{0.1};
    	\discretedelta{8}{1};
    	\discretedelta{9}{1};
    	\discretedelta{10}{1};
    	\discretedelta{11}{0.1};
    	\discretedelta{12}{0.1};
    	\discretedelta{13}{0.1};
    	\discretedelta{14}{0.1};
    	\node at (-15,0.5) {\Large $\cdots$};
    	\node at (15,0.5) {\Large $\cdots$};
    \end{axis}
\end{tikzpicture}
    }
\end{align*}
En caso de que se cumpla, decidir si es posible asegurar que todos los sistemas que cumplan con esta condición son realmente LTI.
\end{ejercicio}
    \or \myheader{Guía 4: Transformada de Fourier}


\begin{ejercicio}
    Hallar, utilizando propiedades, la transformada de Fourier de las siguientes señales:
    
    \begin{align*}
        \inciso & x(t) = 
        \begin{cases} 
            1 + \cos(\pi t) & |t| \leq 1 \\
            0 & |t| > 1 
        \end{cases} & \hspace{\fill} 
        \inciso & x(t) = \frac{\sin(\pi t)}{\pi t} \frac{\sin(2\pi (t-1))}{\pi (t-1)} \\
        \inciso & x(t) = 
        \begin{cases} 
            e^{-t} & 0 \leq t < 1 \\
            0 & \mbox{en otro caso}
        \end{cases} & \hspace{\fill} 
        \inciso & \parbox{.3\textwidth}{\begin{tikzpicture}[scale=0.6,transform shape]
    \begin{axis}[
    	axis y line=center,
    	axis x line=middle,
    	xlabel=$t$,ylabel=$x(t)$,
    	xmin=-1.9,xmax=3.9,
    	ymin=-1.9,ymax=2.9,
    	xticklabel style = {xshift=5},
    	yticklabel style = {yshift=5},
    	]
    	\addplot[
    	black,
    	ultra thick
    	] coordinates {
    	    (-1,0) (-1,1) (0,1) (0,2)
    	    (1,2) (1,-1) (3,-1) (3,0)
    	};
    \end{axis}
\end{tikzpicture}
} \\
        \inciso & \parbox{.3\textwidth}{\begin{tikzpicture}[scale=0.6,transform shape]
    \begin{axis}[
    	axis y line=center,
    	axis x line=middle,
    	xlabel=$t$,ylabel=$x(t)$,
    	xmin=-2.9,xmax=2.9,
    	ymin=-0.9,ymax=2.9,
    	xticklabel style = {xshift=5},
    	yticklabel style = {yshift=5},
    	]
    	\addplot[
    	black,
    	ultra thick
    	] coordinates {
    	    (-2,0) (0,2) (2,0)
    	};
    \end{axis}
\end{tikzpicture}} & \hspace{\fill} 
        \inciso & \parbox{.3\textwidth}{\begin{tikzpicture}[scale=0.6,transform shape]
    \begin{axis}[
    	axis y line=center,
    	axis x line=middle,
    	xlabel=$t$,ylabel=$x(t)$,
    	xmin=-2.9,xmax=2.9,
    	ymin=-1.9,ymax=1.9,
    	xticklabel style = {xshift=5},
    	yticklabel style = {yshift=5},
    	]
    	\addplot[
    	black,
    	ultra thick
    	] coordinates {
    	    (-2,0) (-2,-1) (-1,-1) (1,1) 
    	    (2,1) (2,0)
    	} ;
    \end{axis}
\end{tikzpicture}} \\
        \inciso & \parbox{.3\textwidth}{\begin{tikzpicture}[scale=0.6]
    \begin{axis}[
        x=.06\textwidth,y=0.1\textwidth,
    	axis y line=center,
    	axis x line=middle,
    	xlabel=$t$,ylabel=$x(t)$,
    	xmin=-7.9,xmax=7.5,
    	ymin=-0.3,ymax=2.3,
    	xticklabel style = {xshift=0},
    	yticklabel style = {yshift=5}
	]
	\diracdelta{-6}{2};
	\diracdelta{-5}{1};
	\diracdelta{-4}{2};
	\diracdelta{-3}{1};
	\diracdelta{-2}{2};
	\diracdelta{-1}{1};
	\diracdelta{0}{2};
	\diracdelta{1}{1};
	\diracdelta{2}{2};
	\diracdelta{3}{1};
	\diracdelta{4}{2};
	\diracdelta{5}{1};
	\node at (axis cs:6.5,1) {\Large $\cdots$} ;
	\node at (axis cs:-7,1) {\Large $\cdots$} ;
    \end{axis}
\end{tikzpicture}} & \hspace{\fill} & \\
    \end{align*}
    \end{ejercicio}
    
    \begin{ejercicio}
    Sea $x(t)$ la siguiente función:
    \begin{center}
        \begin{tikzpicture}[scale=0.6,transform shape]
    \begin{axis}[
        x=0.03\textwidth,y=0.2\textwidth,
    	axis y line=center,
    	axis x line=middle,
    	xlabel=$t$,ylabel=$x(t)$,
    	xmin=-21,xmax=21,
    	ymin=-0.4,ymax=1.4,
    	xticklabel style = {xshift=5},
    	yticklabel style = {yshift=5},
    	]
    	\addplot[
    	black,
    	ultra thick
    	] coordinates {
    	    (-15,0) (-12,1) 
    	    (-9,0) (-6,1)
    	    (-3,0) (0,1)
    	    (3,0) (6,1)
    	    (9,0) (12,1)
    	    (15,0)
    	};
    	\node at (18,0.5) {\Large $\cdots$} ;
    	\node at (-18,0.5) {\Large $\cdots$} ;
    \end{axis}
\end{tikzpicture}
    \end{center}
    
    \inciso Calcular la transformada de $x(t)$ utilizando la transformada $y(t) = \frac{\sin(\pi t)}{\pi t} \frac{\sin(2\pi (t-1))}{\pi (t-1)}$ calculada en el punto anterior. 
    
    \inciso ¿Qué relación existe entre la transformada de Fourier de una señal periódica y los coeficientes de su serie de Fourier? 
    
    \inciso ¿Qué característica distintiva tiene una Transformada de Fourier de una señal periódica?
    
    \end{ejercicio}
    
    
    \begin{ejercicio}
    Hallar, utilizando propiedades, la antitransformada de Fourier de las siguientes funciones:
    \begin{align*}
        \inciso & X(\omega) = \cos(4\omega + \pi/3) 
        \hspace*{10em} \inciso \hspace*{0.1em} X(\omega) = \frac{2\sin(3 (\omega-2\pi))}{\omega - 2\pi} \\
        \inciso & 
        X(\omega) \; \mathrm{tal\, que} \; |X(\omega)| = \begin{cases}
            |\omega| & \mbox{si } \omega \in [-1, 1) \\ 
            0 & \mathrm{en\, otro\, caso}
        \end{cases}
        \hspace*{1em} \mathrm{y} \hspace*{1em} \arg(X(\omega)) = -3\omega
    \end{align*}
    \end{ejercicio}
    
    \begin{ejercicio}
    Sea $X(\omega)$ la antitransformada de $x(t)$:
    \begin{center}
        \begin{tikzpicture}[scale=0.6,transform shape]
    \begin{axis}[
        x=0.1\textwidth,y=0.1\textwidth,
    	axis y line=center,
    	axis x line=middle,
    	xlabel={\Large $t$},ylabel={\Large $x(t)$},
    	xmin=-1.9,xmax=3.9,
    	ymin=-0.2,ymax=2.5,
    	xtick distance=1,
    	ytick distance=1,
    	xticklabel style = {xshift=5},
    	yticklabel style = {yshift=5},
    	]
    	\addplot[
    	black,
    	ultra thick
    	] coordinates {
    	    (-1,0) (-1,2) (0,2) (1,1)
    	    (2,2) (3,2) (3,0)
    	};
    \end{axis}
\end{tikzpicture}

    \end{center}
    
    Hallar los siguientes valores sin obtener en forma explícita la función $X(\omega)$:
    \begin{align*}
        \inciso & \angle X(\omega) & \inciso & X(0) & \inciso & \int_{-\infty}^{\infty} X(\omega) d\omega \\[.5em]
        \inciso & \int_{-\infty}^{\infty} X(\omega) \frac{2\sin(\omega)}{\omega} e^{j2\omega} d\omega & \inciso & \int_{-\infty}^{\infty} |X(\omega)|^2 d\omega
    \end{align*}
    
    \end{ejercicio}
    
    \begin{ejercicio}
    Sea $x(t) = e^{-t} (u(t) - u(t-1))$. Graficar las siguientes funciones y hallar la transformada de Fourier de todas ellas:
    \begin{align*}
        \inciso & x_1 = x(-t) + x(t) & 
        \inciso & x_2 = -x(-t) + x(t) &
        \inciso & x_3 = x(t+1) + x(t) &
        \inciso & x_4 = t x(t)
    \end{align*}
\end{ejercicio}
    
\begin{ejercicio}
    Para cada una de las siguientes funciones
    \begin{align*}
        \inciso & \parbox{.5\textwidth}{\begin{tikzpicture}[scale=0.6,transform shape]
    \begin{axis}[
        x=0.04\textwidth,y=0.1\textwidth,
    	axis y line=center,
    	axis x line=middle,
    	xlabel=$t$,ylabel={\LARGE $x(t)$},
    	xmin=-9,xmax=10,
    	ymin=-1.9,ymax=1.9,
    	xticklabel style = {xshift=5},
    	yticklabel style = {yshift=5},
    	]
    	\addplot[
    	black,
    	ultra thick
    	] coordinates {
    	    (-7,1) (-5,-1) 
    	    (-3,1) (-1,-1)
    	    (1,1) (3,-1)
    	    (5,1) (7,-1)
    	};
    	\node at (8,0.5) {\Huge $\cdots$} ;
    	\node at (-8,0.5) {\Huge $\cdots$} ;
    \end{axis}
\end{tikzpicture}} & \hspace{\fill} &
        \inciso & \parbox{.3\textwidth}{\begin{tikzpicture}[scale=0.6,transform shape]
    \begin{axis}[
    	axis y line=center,
    	axis x line=middle,
    	xlabel=$t$,ylabel=$x(t)$,
    	xmin=-2.9,xmax=2.9,
    	ymin=-0.9,ymax=2.9,
    	xticklabel style = {xshift=5},
    	yticklabel style = {yshift=5},
    	]
    	\diracdelta{1}{2};
    \end{axis}
\end{tikzpicture}} \\
        \inciso & \parbox{.5\textwidth}{\begin{tikzpicture}[scale=0.6,transform shape]
    \begin{axis}[
        x=0.08\textwidth,y=0.1\textwidth,
    	axis y line=center,
    	axis x line=middle,
    	xlabel=$t$,ylabel={\Large $x(t)$},
    	xmin=-0.9,xmax=8.9,
    	ymin=-1.3,ymax=1.3,
    	xticklabel style = {xshift=5},
    	yticklabel style = {yshift=5},
    	]
    	\addplot [
    	black, ultra thick,
    	domain=2:8, smooth
    	] {sin(deg(2*pi*x))} ;
    	\addplot[
    	black, ultra thick
    	] coordinates {(-1,0) (2,0)} ;
    	\addplot[
    	black, ultra thick
    	] coordinates {(8,0) (8.5,0)} ;
    \end{axis}
\end{tikzpicture}} & \hspace{\fill} &
        \inciso & \parbox{.3\textwidth}{\begin{tikzpicture}[scale=0.6,transform shape]
    \begin{axis}[
    	axis y line=center,
    	axis x line=middle,
    	xlabel=$t$,ylabel={\Large $x(t)$},
    	xmin=-2.9,xmax=2.9,
    	ymin=-1.9,ymax=1.9,
    	xticklabel style = {xshift=5},
    	yticklabel style = {yshift=5},
    	]
    	\addplot[
    	black,
    	ultra thick
    	] coordinates {(-2,0) (-2,-1) (2,1) (2,0)} ;
    \end{axis}
\end{tikzpicture}} \\
        \inciso & \parbox{.5\textwidth}{\pgfmathdeclarefunction{ejcuatrotresa}{1}{%
  \pgfmathparse{#1^2 * exp(-abs(#1))}%
}

\begin{tikzpicture}[scale=0.6,transform shape]
    \begin{axis}[
        x=0.04\textwidth,y=0.2\textwidth,
    	axis y line=center,
    	axis x line=middle,
    	xlabel=$t$,ylabel={\Large $x(t)=t^2 e^{-|t|}$},
    	xmin=-8.9,xmax=8.9,
    	ymin=-.3,ymax=1.3,
    	xticklabel style = {xshift=5},
    	yticklabel style = {yshift=5},
    	samples=100
    	]
    	\addplot [
    	black, ultra thick,
    	domain=-10:10, smooth
    	] {ejcuatrotresa(x)} ;
    \end{axis}
\end{tikzpicture}} & \hspace{\fill} &
        \inciso & \parbox{.3\textwidth}{\pgfmathdeclarefunction{gauss}{3}{%
  \pgfmathparse{exp(-((#1-#2)^2)/(2*#3^2))}%
}

\begin{tikzpicture}[scale=0.6,transform shape]
    \begin{axis}[
        x=0.04\textwidth,y=0.2\textwidth,
    	axis y line=center,
    	axis x line=middle,
    	xlabel=$t$,ylabel={\Large $x(t)=e^{-t^2/2}$},
    	xmin=-4.9,xmax=4.9,
    	ymin=-.3,ymax=1.3,
    	xticklabel style = {xshift=5},
    	yticklabel style = {yshift=5},
    	]
    	\addplot [
    	black, ultra thick,
    	domain=-10:10, smooth, samples=100
    	] {gauss(x,0,1)} ;
    \end{axis}
\end{tikzpicture}} \\
    \end{align*}
    indicar si cumplen algunas de estas condiciones:
    \begin{align*}
        \subinciso & \Realpart{X(\omega)} = 0 &
        \subinciso & \Impart{X(\omega)} = 0 &
        \subinciso & \exists \alpha \in \mathbb{R} \; \mbox{tal que } e^{j\omega\alpha}X(\omega)\; \mbox{es una función real} \\
        \subinciso & \int_{-\infty}^{\infty} X(\omega) d\omega = 0 &
        \subinciso & \int_{-\infty}^{\infty} \omega X(\omega) d\omega = 0 &
        \subinciso & X(\omega)\; \mbox{es periódica}
    \end{align*}
\end{ejercicio}
    
\begin{ejercicio}
    Hallar y graficar la transformada de Fourier de tiempo discreto de las siguientes secuencias:
    \begin{align*}
        \inciso & x(n) = u(n) - u(n-20) &
        \inciso & x(n) = \delta(n) - \delta(n-1) \\
        \inciso & x(n) = \left( \frac{1}{2} \right)^{-n} u(-n-1) &
        \inciso & x(n) = \sin\left(\frac{\pi}{2} n \right) + \cos(n) 
    \end{align*}
\end{ejercicio}
        
\begin{ejercicio}
    Hallar la antitransformada de Fourier en tiempo discreto de las siguientes funciones:
    
    \inciso $X(e^{j\Omega}) = 1 + 3e^{-j2\Omega} - 4e^{-j10\Omega}$
    
    \inciso $X(e^{j\Omega}) = \sum_{k=-\infty}^{\infty} (-1)^k \delta(\Omega - \frac{\pi}{2}k)$
    
    \inciso $X(e^{j\Omega}) = \frac{e^{-j\Omega} - 1/5}{1-1/5 e^{-j\Omega}}$
    \end{ejercicio}
    
    \begin{ejercicio}
    Sea $X(e^{j\Omega})$ la antitransformada de Fourier de la secuencia $x(n)$:
    
    \begin{center}
    \parbox{.7\textwidth}{\begin{tikzpicture}[scale=0.6]
    \begin{axis}[
        x=.06\textwidth,y=0.1\textwidth,
    	axis y line=center,
    	axis x line=middle,
    	xlabel=$n$,ylabel=$x(n)$,
    	xmin=-7.9,xmax=11.9,
    	ymin=-1.3,ymax=2.3,
    	xticklabel style = {xshift=0},
    	yticklabel style = {yshift=5}
	]
	\discretedelta{-7}{0.1};
	\discretedelta{-6}{0.1};
	\discretedelta{-5}{0.1};
	\discretedelta{-4}{0.1};
	\discretedelta{-3}{-1};
	\discretedelta{-2}{0.1};
	\discretedelta{-1}{1};
	\discretedelta{0}{2};
	\discretedelta{1}{1};
	\discretedelta{2}{0.1};
	\discretedelta{3}{1};
	\discretedelta{4}{2};
	\discretedelta{5}{1};
	\discretedelta{6}{0.1};
	\discretedelta{7}{-1};
	\discretedelta{8}{0.1};
	\discretedelta{9}{0.1};
	\discretedelta{10}{0.1};
	\discretedelta{11}{0.1};
    \end{axis}
\end{tikzpicture}}
    \end{center}
    
    Hallar los siguientes valores sin obtener explícitamente la función $X(e^{j\Omega})$:
    \begin{align*}
    \inciso & X(0) & \inciso & \angle X(e^{j\Omega}) & \inciso & \int_{-\pi}^{\pi} X(e^{j\Omega}) d\Omega \\ 
    \inciso & X(e^{j\pi}) & \inciso & \int_{-\pi}^{\pi}  \left|\frac{X(e^{j\Omega})}{d\Omega}\right|^2 d\Omega &
    \end{align*}
\end{ejercicio}
    
    
\begin{ejercicio}
    Para cada una de las siguientes funciones
    \begin{align*}
        \inciso & \parbox{.3\textwidth}{\begin{tikzpicture}[scale=0.6,transform shape]
    \begin{axis}[
        x=0.04\textwidth,y=0.1\textwidth,
    	axis y line=center,
    	axis x line=middle,
    	xlabel=$n$,ylabel={\LARGE $x(n)$},
    	xmin=-4.9,xmax=8.9,
    	ymin=-0.3,ymax=2.9,
    	xticklabel style = {xshift=0},
    	yticklabel style = {yshift=5},
    	]
    	\discretedelta{-4}{0.1};
    	\discretedelta{-3}{0.1};
    	\discretedelta{-2}{0.1};
    	\discretedelta{-1}{0.5};
    	\discretedelta{0}{1};
    	\discretedelta{1}{1.5};
    	\discretedelta{2}{2};
    	\discretedelta{3}{1.5};
    	\discretedelta{4}{1};
    	\discretedelta{5}{0.5};
    	\discretedelta{6}{0.1};
    	\discretedelta{7}{0.1};
    	\discretedelta{8}{0.1};
    \end{axis}
\end{tikzpicture}} &
        \inciso & \parbox{.4\textwidth}{\begin{tikzpicture}[scale=0.6,transform shape]
    \begin{axis}[
        x=0.04\textwidth,y=0.1\textwidth,
    	axis y line=center,
    	axis x line=middle,
    	xlabel=$n$,ylabel={\LARGE $x(n)$},
    	xmin=-9.9,xmax=9.9,
    	ymin=-1.3,ymax=1.9,
    	xticklabel style = {xshift=0},
    	yticklabel style = {yshift=5},
    	]
    	\discretedelta{-8}{0.1};
    	\discretedelta{-7}{-1};
    	\discretedelta{-6}{0.1};
    	\discretedelta{-5}{1};
    	\discretedelta{-4}{0.1};
    	\discretedelta{-3}{-1};
    	\discretedelta{-2}{0.1};
    	\discretedelta{-1}{1};
    	\discretedelta{0}{0.1};
    	\discretedelta{1}{-1};
    	\discretedelta{2}{0.1};
    	\discretedelta{3}{1};
    	\discretedelta{4}{0.1};
    	\discretedelta{5}{-1};
    	\discretedelta{6}{0.1};
    	\discretedelta{7}{1};
    	\discretedelta{8}{0.1};
    	\node at (-9,0.5) {\Large $\cdots$};
    	\node at (9,0.5) {\Large $\cdots$};
    \end{axis}
\end{tikzpicture}} \\
        \inciso & \parbox{.3\textwidth}{\begin{tikzpicture}[scale=0.6,transform shape]
    \begin{axis}[
        x=0.03\textwidth,y=0.1\textwidth,
    	axis y line=center,
    	axis x line=middle,
    	xlabel=$n$,ylabel={\LARGE $x(n)$},
    	xmin=-9.9,xmax=9.9,
    	ymin=-1.3,ymax=2.9,
    	xticklabel style = {xshift=0},
    	yticklabel style = {yshift=5},
    	]
    	\discretedelta{-9}{0.1};
    	\discretedelta{-8}{0.1};
    	\discretedelta{-7}{0.1};
    	\discretedelta{-6}{0.1};
    	\discretedelta{-5}{0.1};
    	\discretedelta{-4}{-1};
    	\discretedelta{-3}{-1};
    	\discretedelta{-2}{0.1};
    	\discretedelta{-1}{0.1};
    	\discretedelta{0}{2};
    	\discretedelta{1}{-1};
    	\discretedelta{2}{0.1};
    	\discretedelta{3}{0.1};
    	\discretedelta{4}{1};
    	\discretedelta{5}{0.1};
    	\discretedelta{6}{0.1};
    	\discretedelta{7}{0.1};
    	\discretedelta{8}{0.1};
    \end{axis}
\end{tikzpicture}} &
        \inciso & \parbox{.4\textwidth}{\begin{tikzpicture}[scale=0.6,transform shape]
    \begin{axis}[
        x=0.04\textwidth,y=0.1\textwidth,
    	axis y line=center,
    	axis x line=middle,
    	xlabel=$n$,ylabel={\LARGE $x(n)$},
    	xmin=-9.9,xmax=9.9,
    	ymin=-1.3,ymax=2.9,
    	xticklabel style = {xshift=0},
    	yticklabel style = {yshift=5},
    	]
    	\discretedelta{-9}{0.1};
    	\discretedelta{-8}{0.1};
    	\discretedelta{-7}{0.1};
    	\discretedelta{-6}{2};
    	\discretedelta{-5}{0.1};
    	\discretedelta{-4}{-1};
    	\discretedelta{-3}{-1};
    	\discretedelta{-2}{0.1};
    	\discretedelta{-1}{1};
    	\discretedelta{0}{0.1};
    	\discretedelta{1}{1};
    	\discretedelta{2}{0.1};
    	\discretedelta{3}{-1};
    	\discretedelta{4}{-1};
    	\discretedelta{5}{0.1};
    	\discretedelta{6}{2};
    	\discretedelta{7}{0.1};
    	\discretedelta{8}{0.1};
    	\discretedelta{9}{0.1};
    \end{axis}
\end{tikzpicture}} 
    \end{align*}
    indicar si cumplen algunas de estas condiciones:
    \begin{align*}
        \subinciso & \Realpart{X(e^{j\Omega})} = 0 &
        \subinciso & \Impart{X(e^{j\Omega})} = 0 &
        \subinciso & \exists \alpha \in \mathbb{R}\; \mbox{tal que } e^{j\Omega\alpha}X(e^{j\Omega})\; \mbox{es una función real} \\
        \subinciso & \int_{-\pi}^{\pi} X(e^{j\Omega}) d\Omega = 0 &
        \subinciso & X(e^{j\Omega})\; \mbox{es periódica} &
        \subinciso & X(e^{j\Omega})|_{\Omega=0} = 0
    \end{align*}
\end{ejercicio}

\begin{ejercicio}
    Sea un sistema LTI discreto cuya respuesta al impulso es $h(n) = \left(\frac{1}{2}\right)^n u(n)$. Usando la transformada de Fourier de tiempo discreto hallar la salida $y(n)$ del sistema para las siguientes entradas:
    \begin{align*}
        \inciso & x(n) = \left(\frac{3}{4}\right)^n u(n) &
        \inciso & x(n) = (-1)^n u(n) \\
        \inciso & x(n) = A e^{j\frac{\pi}{2}n}, \; \mbox{con } A \in \mathbb{R} &
        \inciso & x(n) = 10 - 5 \sin\left(\frac{\pi}{2}n\right) + 20 \cos\left(\frac\pi n\right)
    \end{align*}
\end{ejercicio}

\begin{ejercicio}
    Utilizando la transformada de Fourier, decidir si existe un sistema LTI que cumpla con que, si $x(n)$ es la entrada, entonces $y(n)$ es la salida:
    
    \inciso $x(n)= e^{j\pi n/3}$ e $y(n) = \cos(\pi n/3)$

    \inciso $x(n) = \cos(\pi n/3)$ e $y(n) = \cos(\pi n/3) + \sqrt{3} \sin(\pi n/3)$
\end{ejercicio}

\begin{ejercicio}
    Sea el siguiente sistema, donde $h_{pb}(n)$ es LTI:
    \begin{center}
    \parbox{.7\textwidth}{\begin{tikzpicture}[scale=0.8, transform shape]
    \node[circle,draw,thick,inner sep=0.03cm] (plus) at (0,0) {\Large +} ;
    \node[circle,draw,thick,inner sep=0.03cm, xshift=-2cm, yshift=1cm] (mult_cos_after) at (plus) {\Large $\times$} ;
    \node[circle,draw,thick,inner sep=0.03cm, xshift=-2cm, yshift=-1cm] (mult_sin_after) at (plus) {\Large $\times$} ;
    \node[rectangle,draw,thick,inner sep=0.3cm, xshift=-2.5cm] (hpb_cos) at (mult_cos_after) {$h_{pb}(n)$} ;
    \node[rectangle,draw,thick,inner sep=0.3cm, xshift=-2.5cm] (hpb_sin) at (mult_sin_after) {$h_{pb}(n)$} ;
    \node[circle,draw,thick,inner sep=0.03cm, xshift=-2.5cm] (mult_cos_before) at (hpb_cos) {\Large $\times$} ;
    \node[circle,draw,thick,inner sep=0.03cm, xshift=-2.5cm] (mult_sin_before) at (hpb_sin) {\Large $\times$} ;
    \node[xshift=-8cm] (arrow_split) at (plus) {} ;
    \node[xshift=-9cm] (x_n) at (plus) {$x(n)$} ;
    \node[xshift=2cm] (y_n) at (plus) {$y(n)$} ;

    \node[xshift=-2cm,yshift=1.5cm] (cos) at (mult_cos_before) {$\cos(\Omega_o n)$} ;
    \node[xshift=-2cm,yshift=-1.5cm] (sin) at (mult_sin_before) {$\sin(\Omega_o n)$} ;

    \draw[very thick] (x_n.east) -- (arrow_split.center) ;
    \draw[->, very thick] (arrow_split.center) |- (mult_cos_before) ;
    \draw[->, very thick] (arrow_split.center) |- (mult_sin_before) ;

    \draw[->, very thick] (cos.east) -| (mult_cos_before.north) ;
    \draw[->, very thick] (cos.east) -| (mult_cos_after.north) ;
    \draw[->, very thick] (sin.east) -| (mult_sin_before.south) ;
    \draw[->, very thick] (sin.east) -| (mult_sin_after.south) ;

    \draw[->, very thick] (mult_cos_before.east) -- (hpb_cos.west) ;
    \draw[->, very thick] (mult_sin_before.east) -- (hpb_sin.west) ;
    \draw[->, very thick] (hpb_cos.east) -- (mult_cos_after.west) ;
    \draw[->, very thick] (hpb_sin.east) -- (mult_sin_after.west) ;
    \draw[->, very thick] (mult_cos_after.east) -| (plus.north) ;
    \draw[->, very thick] (mult_sin_after.east) -| (plus.south) ;

    \draw[->, very thick] (plus.east) -- (y_n.west) ;

\end{tikzpicture}}
    \end{center}

    \inciso Determinar la respuesta al impulso del sistema.

    \inciso Demostrar que el sistema es lineal.

    \inciso Demostrar que el sistema es invariante en el tiempo. \emph{Ayuda:} $\cos(a)\cos(b) + \sin(a)\sin(b) = \cos(a-b)$.

    \inciso Si $\Omega_0 = \frac{\pi}{2}$ y el sistema $h_{pb}(n)$ es un filtro pasabajos ideal con frecuencia de corte $\Omega_c = \frac{\pi}{4}$, determinar la respuesta en frecuencia del sistema.
\end{ejercicio}

\begin{ejercicio}
    Obtener la respuesta en frecuencia del siguiente sistema y compararlo con la del ejercicio anterior:
    \begin{center}
    \parbox{.7\textwidth}{\begin{tikzpicture}[scale=0.8, transform shape]
    \node[rectangle, draw, thick, minimum height=3cm, minimum width=5cm] (hpb) at (0,0) {} ;
    \begin{axis}[
        x=0.04\textwidth,y=0.04\textwidth,
        axis y line=center,
        axis x line=middle,
        xlabel=$\omega$,ylabel={ $H(j\omega)$},
        xmin=-2.9,xmax=2.9,
        ymin=-0.8,ymax=2.7,
        ticks=none,
        at={(-2cm,-1.2cm)}
        ]
        \addplot[
        black,
        ultra thick
        ] coordinates {
            (-1.5,0) (-1.5,1) (1.5,1) (1.5,0)
        } ;
        \node at (-1.5,-.5) {$-\omega_0$};
        \node at (1.5,-.5) {$\omega_0$};
        \node at (0.2,1.3) {1} ;
    \end{axis} 
    \node[yshift=2cm] (hpb_label) at (hpb) {Pasa bajos ideal} ;

    \node[circle,draw,thick,inner sep=0.03cm, xshift=4cm] (mult_after) at (hpb) {\Large $\times$} ;
    \node[circle,draw,thick,inner sep=0.03cm, xshift=-4cm] (mult_before) at (hpb) {\Large $\times$} ;
    \node[yshift=2cm] (exp_after) at (mult_after) {$e^{-j\omega_c t}$} ;
    \node[yshift=2cm] (exp_before) at (mult_before) {$e^{j\omega_c t}$} ;
    \node[xshift=-2cm] (x_t) at (mult_before) {$x(t)$} ;
    \node[xshift=1cm,yshift=0.3cm] (y_t) at (mult_before) {$y(t)$} ;
    \node[xshift=2cm] (f_t) at (mult_after) {$f(t)$} ;
    \node[xshift=-1cm,yshift=0.3cm] (w_t) at (mult_after) {$w(t)$} ;

    \draw[->, very thick] (x_t.east) -- (mult_before.west) ;
    \draw[->, very thick] (mult_before.east) -- (hpb.west) ;
    \draw[->, very thick] (hpb.east) -- (mult_after.west) ;
    \draw[->, very thick] (mult_after.east) -- (f_t.west) ;
    \draw[->, very thick] (exp_before.south) -- (mult_before.north);
    \draw[->, very thick] (exp_after.south) -- (mult_after.north);
\end{tikzpicture}}
    \end{center}
\end{ejercicio}

\begin{ejercicio}
    Sea el sistema de la figura:
    \begin{center}
        \parbox{.7\textwidth}{\begin{tikzpicture}
    
\end{tikzpicture}}
    \end{center}
    donde
    \begin{align*}
        \parbox{.4\textwidth}{\begin{tikzpicture}[scale=0.8,transform shape]
    \begin{axis}[
        x=0.05\textwidth,y=0.05\textwidth,
        axis y line=center,
        axis x line=middle,
        xlabel=$\Omega$,ylabel={\large $X(e^{j\Omega})$},
        xmin=-2.4,xmax=2.9,
        ymin=-0.8,ymax=2.4,
        xtick = {-2, -1, 0, 1, 2},
        xticklabels = {$-2\pi$, $-\pi$, $0$, $\pi$, $2\pi$},
        ytick = {0, 1},
        yticklabels = {$0$,$1$},
        yticklabel style={yshift=0.3cm},
        ]
        \addplot[
        black,
        ultra thick
        ] coordinates {
            (-1,0) (0,1) (1,0)
        } ;
    \end{axis}
\end{tikzpicture}} &
        \parbox{.4\textwidth}{\begin{tikzpicture}[scale=0.8,transform shape]
    \begin{axis}[
        x=0.05\textwidth,y=0.05\textwidth,
        axis y line=center,
        axis x line=middle,
        xlabel=$\Omega$,ylabel={\large $H_{pb}(e^{j\Omega})$},
        xmin=-4.4,xmax=4.9,
        ymin=-0.8,ymax=2.4,
        xtick = {-4, -3, -2, -1, 0, 1, 2, 3, 4},
        xticklabels = {-$\pi$, $-\frac{3\pi}{4}$, $-\frac{\pi}{2}$, $-\frac{\pi}{4}$, $0$, $\frac{\pi}{4}$, $\frac{\pi}{2}$, $\frac{3\pi}{4}$, $\pi$},
        ytick = {0, 1},
        yticklabels = {$0$,$1$},
        yticklabel style={yshift=0.3cm},
        ]

        \addplot[
        black,
        ultra thick
        ] coordinates {
            (-1,0) (-1,1) (1,1) (1,0)
        } ;
    \end{axis}
\end{tikzpicture}}
    \end{align*}
    Graficar los espectros en frecuencias de las señales $g_1(n)$, $g_2(n)$, $g_3(n)$, $g_4(n)$ y $y(n)$.
\end{ejercicio}
    \or \myheader{Guía 5: Ecuaciones Diferenciales \\ y en Diferencias (Parte I)}

\begin{ejercicio}
    Sea un sistema LTI de tiempo discreto tal que la relación entre la salida y la entrada del mismo se puede escribir como:
    \begin{equation*}
        y(n) = \sum_{k=0}^{\infty} \alpha^{n-k} \left(x(k) + \beta x(k-1) + \gamma x(k-2)\right)
    \end{equation*}
    donde $|\alpha| < 1$ y $\beta, \gamma \in \R$.

    \inciso Obtener la respuesta al impulso del sistema y graficarla.
    
    \inciso Analizar la estabilidad del mismo.
    
    \inciso Determinar los valores de $\beta$ y $\gamma$ para que el sistema tenga las siguientes propiedades:
    \begin{itemize}
        \item Si $x(n) = (-1)^n$ para todo $n$ entonces $y(n) = 0$ para todo $n$.
        \item Si $x(n) = 1$ para todo $n$ entonces $y(n) = 1$ para todo $n$.
    \end{itemize}
\end{ejercicio}

\begin{ejercicio}
    Dado un sistema en tiempo discreto definido por la ecuación en diferencias 
    \begin{equation*}
        y(n) = x(n) + \frac{3}{4} y(n-1)
    \end{equation*}
    con condiciones de contorno
    \begin{align*}
    \inciso & y(-1) = 0 & \inciso & y(1) = 0 & \inciso & y(0) = 0 \\[.5em]
    \inciso & y(0) = 1 & \inciso & \lim_{n\rightarrow -\infty} y(n) = 0  & \inciso & \lim_{n\rightarrow +\infty} y(n) = 0
    \end{align*}
    se pide:
    \begin{itemize}
        \item Calcular el valor de la respuesta al impulso $h(n)$ del sistema en el intervalo $-5 \leq n \leq 5$
        \item Obtener una expresión cerrada de $h(n)$ para todo $n \in \mathbb{Z}$.
        \item Determinar si el sistema lineales, invariante ante desplazamientos, estables y causales.
    \end{itemize}
\end{ejercicio}
    
\begin{ejercicio}
    Dado un sistema en tiempo discreto definido por la ecuación en diferencias 
    \begin{equation*}
        y(n) = x(n) + \frac{1}{2} y(n-1)
    \end{equation*}
    con condiciones
    
    \inciso iniciales de reposo 

    \inciso finales de reposo

    \vspace*{1ex}

    \noindent se pide:
    \begin{itemize}
        \item Determinar si el sistema lineales, invariante ante desplazamientos, estables y causales. Establecer cómo se relacionan las condiciones de contorno con cada una de las propiedades del sistema.
        \item Calcular el valor de la respuesta a las señales $\delta(n-1)$ y $\delta(n+1)$ del sistema en el intervalo $-5 \leq n \leq 5$.
        \item Obtener una expresión cerrada de la respuesta a las señales $\delta(n-1)$ y $\delta(n+1)$ para todo $n \in \mathbb{Z}$.
    \end{itemize}
\end{ejercicio}
    
\begin{ejercicio}
    Demostrar que un sistema definido por ecuaciones en diferencias FIR siempre será lineal, invariante ante desplazamientos y estable. Determinar, además, qué condición debe cumplir un sistema de este tipo para ser causal.
\end{ejercicio}
    
\begin{ejercicio}
    Obtener la respuesta al impulso para el sistema definido por la ecuación diferencial 
    \begin{equation*}
        y(n) = x(n+1) + x(n) + x(n-1) + \frac{3}{4} y(n-1)
    \end{equation*}
    en condiciones inciales de reposo. 
\end{ejercicio}
    
\begin{ejercicio}
    Obtener la respuesta al impulso para el sistema definido por la ecuación diferencial 
    \begin{equation*}
        y(n) = x(n+1) + x(n) + x(n-1) + \frac{5}{4} y(n-1)
    \end{equation*}
    en condiciones finales de reposo. 
\end{ejercicio}
    
\begin{ejercicio}
    Implementar en \Keyboardsym una función que permita obtener la respuesta al impulso de una ecuación en diferencias para condiciones iniciales o finales de reposo.
\end{ejercicio}
    
\begin{ejercicio}
    Para el sistema descripto por la ecuación en diferencias $y(n) = \frac{1}{2} y(n-1) + x(n)$ con condiciones iniciales de reposo, se pide:
    
    \inciso Encontrar la expresión analítica de la salida $y(n)$ ante una entrada $x(n)$ aplicada a partir de $n = 0$.
    
    \inciso Encontrar a partir de la expresión anterior la salida que corresponde cuando la entrada es $x(n) = \delta(n)$. 
    
    \inciso Encontrar a partir de la expresión general de la salida del sistema, la salida que corresponde
    cuando la entrada es $x(n) = A e^{j\frac{\pi}{2}n}, n \geq 0$. A partir de esta expresión, y
    comparándola con la obtenida en el ejercicio anterior (c) discuta el significado de la respuesta permanente y de la transitoria. Grafique ambos tipos de respuesta en \Keyboard \hspace*{0.1em} y discuta cómo es mejor implementar la función que obtiene la salida del sistema, si mediante una convolución o un filtro recursivo.
\end{ejercicio}

\begin{ejercicio}
    Sea un sistema LTI y una señal $x(n)$ que satisface $x(n)= \delta(n) + \sum_{k=1}^M a_k x(n-k)$ donde $a_k \in \R$ para todo $k$. Sea $y(n)$ la salida del sistema cuando la entrada es $x(n)$.
    
    \inciso Determinar una ecuación en diferencias de $h(n)$ en función de $y(n)$ usando las propiedades básicas de un sistema LTI. 

    \inciso Si $y(n)$ es de duración finita, ¿qué se puede decir de la estabilidad del sistema?

    \inciso Si $y(n)$ es de duración infinita, obtener algún tipo de condición sobre $y(n)$ que asegure la estabilidad del sistema. 
\end{ejercicio}

\begin{ejercicio}
    Sea el sistema de tiempo continuo que, dada una entrada $x(t)$, la salida es $y(t) = \sum_{k=0}^{\infty} a_k x(t-kT)$ donde $a_k \in \R$ para todo $k$ y $T>0$. Este sistema puede modelar una situación donde existen una superposición de distintos ecos de la señal de entrada.

    \inciso Determinar si el sistema es LTI y, en caso afirmativo, determinar la respuesta al impulso.

    \inciso ¿Cuáles son las condiciones que en general deben cumplir los valores de $a_k$ (las amplitudes de
    los ecos) para que el sistema sea estable? Con este resultado, analizar el caso en que $a_k = \alpha^k$ con $\alpha \in \R$.

    \inciso Asumiendo que $a_k = \alpha^k$ con $|\alpha| < 1$, obtener un sistema que recupere la entrada $x(t)$ a partir de la salida $y(t)$ con ecos. Determinar, de ser posible, la ecuación en diferencias a coeficientes constantes que implementa el mismo. 
\end{ejercicio}

\begin{ejercicio}
    Sea la señal $x(t)$ cuya transformada de Fourier es $X(j\omega) = u(\omega - W_1) - u(\omega - W_2)$, con $W_1 < W_2$ y sea 
    \begin{equation*}
        y(t) = \frac{dx(t)}{dt} * x(t) + x(t)
    \end{equation*}
    Obtener $\int_{-\infty}^{\infty} |y(t-\alpha)|^2 dt$ donde $\alpha \in \R$.
\end{ejercicio}

\begin{ejercicio}
    Sea un sistema LTI cuya entrada es $x(n)$ y salida es $y(n)$. La relación entre la entrada y salida es:
    \begin{equation*}
        y(n) - \alpha y(n-1) = \sum_{k=-\infty}^{\infty} x(k) z(n-k) - x(n)
    \end{equation*}
    donde $z(n)$ es una secuencia cuya transformada de Fourier existe. 

    \inciso Encontrar la respuesta en frecuencia del sistema con $\alpha = \frac{1}{2}$. 

    \inciso Asumiendo que $z(n) = \beta^n u(n) + \delta(n)$ con $\beta = \frac{1}{3}$, encontrar la respuesta al impulso del sistema.
\end{ejercicio}

\begin{ejercicio}
    Sea un sistema LTI en tiempo discreto dado por:
    \begin{equation*}
        y(n) = \sum_{k=-\infty}^{n-1} \alpha^{n-k} \left( x(k - \beta_1) + x(k - \beta_2) \right),\; |\alpha| < 1, \; \beta_1, \beta_2 \in \mathbb{Z}\; \beta_2 > \beta_1
    \end{equation*}

    \inciso Determinar la respuesta al impulso $h(n)$ del sistema y la estabilidad del mismo. 

    \inciso Encontrar una ecuación en diferencias recursiva para el sistema.
\end{ejercicio}

\begin{ejercicio}
    Sea un sistema LTI en tiempo discreto causal con entrada $x(n)$ y salida $y(n)$ definido por el siguiente \emph{par} de ecuaciones diferenciales:
    \begin{align*}
        y(n) + \frac{1}{4} y(n-1) + w(n) + \frac{1}{2} w(n-1) &= \frac{2}{3} x(n) \\
        y(n) - \frac{5}{4} y(n-1) + 2w(n) - 2 w(n-1) &= -\frac{5}{3} x(n)
    \end{align*}
    \inciso Obtener la respuesta en frecuencia y la respuesta al impulso del sistema.
    
    \inciso Obtener una única ecuación en diferencias que relacione $x(n)$ con $y(n)$.
\end{ejercicio}

\begin{ejercicio}
    Un motor paso-a-paso puede modelarse a través del siguiente circuito:
    \begin{center}
        \parbox{0.5\textwidth}{
            \begin{circuitikz}[scale=0.8, transform shape, american voltages]
    \draw
    (0,0) to [american current source, l^=$i(t)$] (0,4)
    to (2,4)
    to [R, l^={\parbox{1.2cm}{
        \begin{center}
            $R_1$ \\ $100\;\Omega$    
        \end{center}
        }}] (2,0)
    to (0,0) ;
    \draw (2, 4) to [R, l_={$R_2$}] (6,4)
    to [L, l_={\parbox{1.2cm}{
        \begin{center}
            $L_1$ \\ $50\;\mu\mathrm{H}$    
        \end{center}
        }}, v^=$v_L(t)$] (6,0)
    to (2,0) ;

    \draw[->, thick] (3.3, 4.5) -- (4.6, 4.5);
    \node at (3.9, 4.8) {$i_1(t)$};
\end{circuitikz}
        }
    \end{center}
    
    En el mismo, el motor (representado por la inductancia $L_1$) se alimenta de una fuente de corriente $i(t)$. La ecuación de la corriente $i_1(t)$ que circula por el mismo se puede escribir como:
    \begin{equation*}
        i(t) = \frac{v_L(t)}{R_1} + \left(1 + \frac{R_2}{R_1}\right) i_1(t)
    \end{equation*}
    en donde la tensión $v_L(t) = L_1 \frac{d i_1(t)}{dt}$ es la tensión en los bornes de la inductancia. La fuente de corriente $i(t)$ entrega una corriente periódica de la siguiente forma:
    \begin{center}
        \parbox{0.5\textwidth}{
            \begin{tikzpicture}
    
\end{tikzpicture}
        }
    \end{center}
    \inciso Determinar la respuesta en frecuencia del sistema donde la entrada es $i(t)$ y la salida es $v_L(t)$.

    \inciso Determinar la representación en serie de Fourier de $i(t)$ y $v_L(t)$.

    \inciso Si se desea que la amplitud de la quinta armónica de la tensión en los bornes del inductor no supere los $10\;\mathrm{mV}$ dado que el motor debe funcionar cerca de un equipo sensible a la interferencia, determinar el valor de $R_2$ que garantiza eso. 
\end{ejercicio}


    \or \myheader{Guía 6: Muestreo e Interpolación}


\begin{ejercicio}
    A continuación se muestra el sistema global para filtrar una señal en tiempo continuo utilizando un filtro en tiempo discreto.
    \begin{center}
        \begin{tikzpicture}[scale=.8,transform shape]
    \node[circle,radius=1cm,draw,thick] (x) at (0,0) {$\times$};
    \node[above=1.5cm] (deltatrain) at (x) {$s(t) = \sum_{n=-\infty}^{\infty} \delta(t-nT)$};
    \node[left=3cm] (x_t) at (x) {$x_c(t)$} ;
    \node[above=5cm,right=.4cm] (x_p) at (x.north) {$x_s(t)$} ;
    \node[right=1.5cm,rectangle,draw,thick,inner sep=0.2cm] (ad) at (x) {\parbox{2cm}{\tiny Conversor de tren de impulsos a secuencia en tiempo discreto}} ;
    % \node[right=1.5cm,rectangle,draw,thick,inner sep=0.5cm] (ad) at (x) {$H(j\omega)$} ;
    \node[yshift=.3cm,right=1.8cm] at (ad) {$x(n)$} ;
    \node[right=3cm,rectangle,draw,thick,inner sep=0.5cm] (H_W) at (ad) {$H(e^{j\Omega})$} ; 
    \node[yshift=.3cm,right=1.2cm] at (H_W) {$y(n)$} ;
    
    \node[right=3cm,rectangle,draw,thick,inner sep=0.2cm] (da) at (H_W) {\parbox{1.5cm}{\tiny Conversor a tren de impulsos}} ;
    
    \node[right=2.5cm,rectangle,draw,thick,inner sep=0.2cm] (bp) at (da) {\parbox{1.5cm}{\tiny Filtro de reconstrucción ideal $H_r(j\omega)$}} ;
    
    \node[yshift=.5cm,right=1.2cm] (y_s) at (da) {$y_s(t)$} ;
    \node[right=2cm] (yr_t) at (bp) {$y_r(t)$} ;
    \node[dashed,thick,rectangle,draw,minimum width=7cm,minimum height=4cm,yshift=.5cm] (ad_box) at (x_p) {} ;
    \node[yshift=2.5cm] (ad_label) at (ad_box) {A/D} ;
    
    \node[dashed,thick,rectangle,draw,minimum width=6.5cm,minimum height=4cm,yshift=.4cm] (da_box) at (y_s) {} ;
    \node[yshift=2.5cm] (da_label) at (da_box) {D/A} ;
    
    \draw[->,very thick] (deltatrain) -- (x) ;
    \draw[->,very thick] (x_t) -- (x) ;
    \draw[->,very thick] (x) -- (ad) ;
    \draw[->,very thick] (ad) -- (H_W) ;
    \draw[->,very thick] (H_W) -- (da) ;
    \draw[->,very thick] (da) -- (bp) ;
    \draw[->,very thick] (bp) -- (yr_t) ;
    \end{tikzpicture}
    \end{center}
    Asumiendo que $H(e^{j\Omega})$ y $H_r(j\omega)$ tienen la forma
    \begin{center}
        \begin{tabular}{ccc}
    \begin{tikzpicture}[scale=0.9,transform shape]
    \begin{axis}[
        x=0.05\textwidth,y=0.05\textwidth,
        axis y line=center,
        axis x line=middle,
        xlabel=$\Omega$,ylabel={\large $H(e^{j\Omega})$},
        xmin=-2.9,xmax=2.9,
        ymin=-0.9,ymax=2.9,
        ticks=none
        ]
        \addplot[
        black,
        ultra thick
        ] coordinates {
            (-1.5,0) (-1.5,1) (1.5,1) (1.5,0)
        } ;
        \node at (-1.5,-.5) {$-\frac{\pi}{4}$};
        \node at (1.5,-.5) {$\frac{\pi}{4}$};
        \node at (0.2,1.2) {1} ;
    \end{axis} 
    \end{tikzpicture} & \hfill &
    \begin{tikzpicture}[scale=0.9,transform shape]
    \begin{axis}[
        x=0.05\textwidth,y=0.05\textwidth,
        axis y line=center,
        axis x line=middle,
        xlabel=$\omega$,ylabel={\Large $H_r(j\omega)$},
        xmin=-2.9,xmax=2.9,
        ymin=-0.9,ymax=2.9,
        ticks=none
        ]
        \addplot[
        black,
        ultra thick
        ] coordinates {
            (-1.5,0) (-1.5,1) (1.5,1) (1.5,0)
        } ;
        \node at (-1.5,-.5) {$-\frac{\pi}{T}$};
        \node at (1.5,-.5) {$\frac{\pi}{T}$};
        \node at (0.2,1.2) {T} ;
    \end{axis} 
    \end{tikzpicture}
\end{tabular}
    \end{center}
    se pide:
    
    \inciso Para el caso en que $\frac{1}{T}=20\;\mathrm{kHz}$ y la transformada $X_c(j\omega)$ de $x_c(t)$ es
    \begin{center}
        \begin{tikzpicture}[scale=0.9,transform shape]
    \begin{axis}[
        x=0.05\textwidth,y=0.05\textwidth,
        axis y line=center,
        axis x line=middle,
        xlabel=$\omega$,ylabel={\large $X(j\omega)$},
        xmin=-2.9,xmax=2.9,
        ymin=-0.8,ymax=1.9,
        ticks=none
        ]
        \addplot[
        black,
        ultra thick
        ] coordinates {
            (-1,0) (0,1) (1,0)
        } ;
        \node at (-1,-.3) {$-2\pi f_M$};
        \node at (1,-.3) {$2\pi f_M$};
        \node at (-.3,1) {1} ;
    \end{axis}
    \end{tikzpicture}
    \end{center}
    con $f_M=10\;\mathrm{kHz}$ graficar las transformadas $X_s(j\omega)$ y $X(e^{j\Omega})$ de $x_s(t)$ y $x(n)$ respectivamente.
    
    \inciso Determinar para qué rango de valores de $T$, el sistema completo con entrada $x_c(t)$ de banda limitada ($2\pi f_M=\mbox{frecuencia máxima de la señal}$) y salida $y_r(t)$ es equivalente al sistema LTI con respuesta en frecuencia $H_{eff}(j\omega)=\begin{cases}1 & \mbox{si $\omega \in (-\omega_c,\omega_c)$}\\ 0 & \mbox{en otro caso}\end{cases}$. Determinar el valor de $\omega_c$ en función de $T$.
    
\end{ejercicio}
    
    
\begin{ejercicio}
    La señal discreta $x_d(n)=\cos\left(\frac{\pi}{4}n\right)$ con $n\in\mathbb{Z}$ se obtuvo del muestreo de la señal continua $x_c(n)=\cos\left(\omega_0 t\right)$ con $t\in\mathbb{R}$ a una frecuencia de muestreo $F_S = 1000\;\mathrm{Hz}$. ¿Qué valores de $\omega_0$ positivos resultarían en la secuencia $x_d(n)$?
\end{ejercicio}

\begin{ejercicio}
    Sea el siguiente sistema de muestreo
    \begin{center}
    \begin{tikzpicture}
    \node[circle,inner sep=0.05cm, draw, very thick] (x) at (0,0) {\large $\times$};
    \node[above=1.5cm] (deltatrain) at (x) {$p(t) = \sum_{n=-\infty}^{\infty} \delta(t-nT)$};
    \node[left=1.5cm] (x_t) at (x) {$x(t)$} ;
    \node[right=1.5cm,rectangle,draw,very thick,inner sep=0.5cm] (H_jw) at (x) {$H_c(j\omega)$} ;
    \node[right=2cm] (x_r) at (H_jw) {$y_c(t)$} ;
    \node[above=5cm,right=.4cm] at (x.north) {$x_p(t)$} ;
    
    \draw[->,very thick] (deltatrain) -- (x) ;
    \draw[->,very thick] (x_t) -- (x) ;
    \draw[->,very thick] (x) -- (H_jw) ;
    \draw[->,very thick] (H_jw) -- (x_r) ;
\end{tikzpicture}
    \end{center}
    donde $H_c(j\omega)$ es un pasabajos ideal con frecuencia de corte $\frac{\omega_s}{2}$ y ganancia $T$ y donde $\omega_s = \frac{2\pi}{T}$. En este sistema, cuando la entrada vale $x_c(t) = \cos(\omega_0 t)$ la salida es $y_c(t) = \cos(\omega_0 t)$. Por otro lado, cuando $x_c(t) = \cos(10\omega_0 t)$ la salida es $y_c(t) = \cos(2\omega_0 t)$. Determinar un valor de $\omega_s$ compatible con esta situación y graficar los espectros de $x_p(t)$ para los dos casos, explicando clararmente la situación en cada uno de ellos.
\end{ejercicio}
    
\begin{ejercicio}
    Dado el siguiente sistema de muestreo
    \begin{center}
    \begin{tabular}{ccc}
    \begin{tikzpicture}
    \node[circle,radius=1cm,draw,thick] (x) at (0,0) {$\times$};
    \node[above=1.5cm] (deltatrain) at (x) {$p(t) = \sum_{n=-\infty}^{\infty} \delta(t-nT)$};
    \node[left=1.5cm] (x_t) at (x) {$x(t)$} ;
    \node[right=1.5cm,rectangle,draw,thick,inner sep=0.5cm] (H_jw) at (x) {$H(j\omega)$} ;
    \node[right=2cm] (x_r) at (H_jw) {$x_r(t)$} ;
    \node[above=5cm,right=.4cm] at (x.north) {$x_p(t)$} ;
    
    \draw[->,very thick] (deltatrain) -- (x) ;
    \draw[->,very thick] (x_t) -- (x) ;
    \draw[->,very thick] (x) -- (H_jw) ;
    \draw[->,very thick] (H_jw) -- (x_r) ;
    \end{tikzpicture} & \hfill &
    \begin{tikzpicture}[scale=0.9,transform shape]
    \begin{axis}[
        x=0.05\textwidth,y=0.05\textwidth,
        axis y line=center,
        axis x line=middle,
        xlabel=$\omega$,ylabel={\Large $H(j\omega)$},
        xmin=-4.9,xmax=4.9,
        ymin=-0.9,ymax=2.9,
        ticks=none
        ]
        \addplot[
        black,
        ultra thick
        ] coordinates {
            (-4,0) (-4,1) (-2.5,1) (-2.5,0) 
            (2.5,0) (2.5,1) (4,1)  (4,0)
        } ;
        \node at (-4,-.5) {$-\omega_b$};
        \node at (-2.5,-.5) {$-\omega_a$};
        \node at (4,-.5) {$\omega_b$};
        \node at (2.5,-.5) {$\omega_a$};
        \node at (0.15,1) {$\mathbf{-}A$} ;
    \end{axis}
    \end{tikzpicture}
    \end{tabular}
    \end{center}
    y considerando que $\omega_1 > \omega_2-\omega_1$, encontrar el máximo valor de $T$ y los valores de las constantes $A$, $\omega_a$, $\omega_b$ tales que $x(t)=x_r(t)$ cuando la señal $x(t)$ tiene un espectro como se muetra a continuación
    \begin{center}
    \begin{tikzpicture}[scale=0.9,transform shape]
    \begin{axis}[
        x=0.05\textwidth,y=0.05\textwidth,
        axis y line=center,
        axis x line=middle,
        xlabel=$\omega$,ylabel={\Large $X(j\omega)$},
        xmin=-4.9,xmax=4.9,
        ymin=-0.9,ymax=2.9,
        ticks=none
        ]
        \addplot[
        black,
        ultra thick
        ] coordinates {
            (-4,0) (-2.5,1.5) (-2.5,0) 
            (2.5,0) (2.5,1.5) (4,0)
        } ;
        \node at (-4,-.5) {$-\omega_b$};
        \node at (-2.5,-.5) {$-\omega_a$};
        \node at (4,-.5) {$\omega_b$};
        \node at (2.5,-.5) {$\omega_a$};
        \node at (0.15,1.5) {$\mathbf{-}1$} ;
    \end{axis}
    \end{tikzpicture}
    \end{center}
\end{ejercicio}
    
\begin{ejercicio}
    Sea el sistema de muestreo como se muestra a continuación:
    \begin{center}
    \begin{tabular}{ccc}
    \begin{tikzpicture}
    \node[circle,radius=1cm,draw,thick] (x) at (0,0) {$\times$};
    \node[above=1.5cm] (deltatrain) at (x) {$p(t) = \sum_{n=-\infty}^{\infty} \delta(t-n(T+\Delta))$};
    \node[left=1.5cm] (x_t) at (x) {$x(t)$} ;
    \node[right=1.5cm,rectangle,draw,thick,inner sep=0.5cm] (H_jw) at (x) {$H(j\omega)$} ;
    \node[right=2cm] (x_r) at (H_jw) {$y(t)$} ;
    \node[above=5cm,right=.4cm] at (x.north) {$x_p(t)$} ;
    
    \draw[->,very thick] (deltatrain) -- (x) ;
    \draw[->,very thick] (x_t) -- (x) ;
    \draw[->,very thick] (x) -- (H_jw) ;
    \draw[->,very thick] (H_jw) -- (x_r) ;
    \end{tikzpicture} & \hfill &
    \begin{tikzpicture}[scale=0.9,transform shape]
    \begin{axis}[
        x=0.05\textwidth,y=0.05\textwidth,
        axis y line=center,
        axis x line=middle,
        xlabel=$\omega$,ylabel={\Large $H(j\omega)$},
        xmin=-4.9,xmax=4.9,
        ymin=-0.9,ymax=2.9,
        ticks=none
        ]
        \addplot[
        black,
        ultra thick
        ] coordinates {
            (-2,0) (-2,1) (2,1) (2,0)
        } ;
        \node at (-2.2,-.5) {$-\frac{1}{2(T+\Delta)}$};
        \node at (2,-.5) {$\frac{1}{2(T+\Delta)}$};
        \node at (0.2,1.2) {1} ;
    \end{axis}
    \end{tikzpicture}
    \end{tabular}
    \end{center}
    Para $x(t)=\cos\left(\frac{2\pi}{T}t\right)$ encontrar el intervalo de valores de $\Delta$ de manera que $y(t)$ sea proporcional a $x(at)$ para algún $a\in(0,1)$. Determinar el valor de $a$ en términos de $T$ y de $\Delta$.
\end{ejercicio}
    
\begin{ejercicio}
    Sea el siguiente sistema de muestreo
    \begin{center}
    \begin{tabular}{ccc}
    \begin{tikzpicture}
    \node[circle,radius=1cm,draw,thick] (x) at (0,0) {$\times$};
    \node[above=1.5cm] (deltatrain) at (x) {$p(t) = \sum_{n=-\infty}^{\infty} (-1)^{n} \delta(t-nT)$};
    \node[left=1.5cm] (x_t) at (x) {$x(t)$} ;
    \node[right=1.5cm,rectangle,draw,thick,inner sep=0.5cm] (H_jw) at (x) {$H(j\omega)$} ;
    \node[right=2cm] (x_r) at (H_jw) {$y(t)$} ;
    \node[above=5cm,right=.4cm] at (x.north) {$x_p(t)$} ;
    
    \draw[->,very thick] (deltatrain) -- (x) ;
    \draw[->,very thick] (x_t) -- (x) ;
    \draw[->,very thick] (x) -- (H_jw) ;
    \draw[->,very thick] (H_jw) -- (x_r) ;
    \end{tikzpicture} & \hfill &
    \begin{tikzpicture}[scale=0.9,transform shape]
    \begin{axis}[
        x=0.05\textwidth,y=0.05\textwidth,
        axis y line=center,
        axis x line=middle,
        xlabel=$\omega$,ylabel={\Large $H(j\omega)$},
        xmin=-4.9,xmax=4.9,
        ymin=-0.9,ymax=2.9,
        ticks=none
        ]
        \addplot[
        black,
        ultra thick
        ] coordinates {
            (-4,0) (-4,1) (-2.5,1) (-2.5,0) 
            (2.5,0) (2.5,1) (4,1)  (4,0)
        } ;
        \node at (-4,-.5) {$-\frac{3\pi}{T}$};
        \node at (-2.5,-.5) {$-\frac{\pi}{T}$};
        \node at (4,-.5) {$\frac{3\pi}{T}$};
        \node at (2.5,-.5) {$\frac{\pi}{T}$};
        \node at (0.15,1) {$\mathbf{-}1$} ;
    \end{axis}
    \end{tikzpicture}
    \end{tabular}
    \end{center}
    Dada una señal $x(t)$ cuya transformada de Fourier es como se muestra a continuación:
    \begin{center}
    \begin{tikzpicture}[scale=0.9,transform shape]
    \begin{axis}[
        x=0.05\textwidth,y=0.05\textwidth,
        axis y line=center,
        axis x line=middle,
        xlabel=$\omega$,ylabel={\Large $X(j\omega)$},
        xmin=-4.9,xmax=4.9,
        ymin=-0.9,ymax=2.9,
        ticks=none
        ]
        \addplot [
        black, ultra thick,
        domain=-2.5:2.5, smooth
        ] {cos(deg(2*pi/5*x))*.5+1} ;
        \addplot [
        black, ultra thick
        ] coordinates {(-2.5,0) (-2.5,.5)} ;
        \addplot [
        black, ultra thick
        ] coordinates {(2.5,0) (2.5,.5)} ;
        \node at (-2.5,-.5) {$-\omega_M$};
        \node at (2.5,-.5) {$\omega_M$};
        \node at (0.15,1.5) {$\mathbf{-}1$} ;
    \end{axis}
    \end{tikzpicture}
    \end{center}
    se pide:
    
    \inciso Para $T=\frac{\pi}{2\omega_M}$ dibujar la transformada de Fourier de $x_p(t)$ e $y(t)$.
    
    \inciso Para $T=\frac{\pi}{2\omega_M}$ determinar un sistema con el cual se pueda recuperar $x(t)$ a partir de $x_p(t)$.
    
    \inciso Para $T=\frac{\pi}{2\omega_M}$ determinar un sistema con el cual se pueda recuperar $x(t)$ a partir de $y(t)$.
    
    \inciso Determinar el valor \emph{máximo} de $T$ en relación a $\omega_M$ para el cual $x(t)$ puede recuperarse a partir de $x_p(t)$ o de $y(t)$.
    
\end{ejercicio}
    
\begin{ejercicio}
    En la siguiente figura se muestra un sistema en el cual la señal de entrada es multiplicada por una onda cuadrada periódica $s(t)$ de período $T$. 
    \begin{center}
    \begin{tabular}{ccc}
    \begin{tikzpicture}
    \node[circle,radius=1cm,draw,thick] (x) at (0,0) {$\times$};
    \node[above=2cm] (deltatrain) at (x) {$s(t)$};
    \node[left=2cm] (x_t) at (x) {$x(t)$} ;
    % \node[right=1.5cm,rectangle,draw,thick,inner sep=0.5cm] (H_jw) at (x) {$H(j\omega)$} ;
    % \node[right=2cm] (x_r) at (H_jw) {$y(t)$} ;
    \node[right=2cm] (w) at (x) {$w(t)$} ;
    
    \draw[->,very thick] (deltatrain) -- (x) ;
    \draw[->,very thick] (x_t) -- (x) ;
    \draw[->,very thick] (x) -- (w) ;
    % \draw[->,very thick] (x) -- (H_jw) ;
    % \draw[->,very thick] (H_jw) -- (x_r) ;
    \end{tikzpicture} & \hfill &
    \begin{tikzpicture}[scale=0.9,transform shape]
    \begin{axis}[
        x=0.05\textwidth,y=0.05\textwidth,
        axis y line=center,
        axis x line=middle,
        xlabel=$t$,ylabel={\large $s(t)$},
        xmin=-4.9,xmax=4.9,
        ymin=-1.9,ymax=2.3,
        ticks=none
        ]
        \addplot[
        black,
        ultra thick
        ] coordinates {
            (-5.5,1) (-4.5 1) (-3.5, 1) (-3.5,-1) (-2.5,-1) (-2.5, 1) (-1.5, 1) (-1.5, -1) (-.5,-1) (-.5,1) (.5,1) (.5,-1) (1.5,-1) (1.5,1) (2.5,1) (2.5,-1) (3.5,-1) (3.5,1) (4.5,1)
        } ;
        \node at (0.2,1.3) {$1$} ;
        \node at (0.35,-1.3) {$-1$} ;
        \node at (-.85,-0.3) {$-\Delta$} ;
        \node at (.75,-0.3) {$\Delta$} ;
        \node at (2,-0.4) {$T$} ;
        \node at (2,0) {$|$} ;
    \end{axis}
    \end{tikzpicture}
    \end{tabular}
    \end{center}
    Suponiendo que la entrada es de banda limitada ($X(j\omega)=0\; \forall \omega>\omega_M$
    
    \inciso Para $\Delta=\frac{T}{3}$ determinar en términos de $\omega_M$ el valor máximo de $T$ para el cual no hay traslape entre las réplicas de $X(j\omega)$ en $W(j\omega)$.
    
    \inciso Para $\Delta=\frac{T}{4}$ determinar en términos de $\omega_M$ el valor máximo de $T$ para el cual no hay traslape entre las réplicas de $X(j\omega)$ en $W(j\omega)$
\end{ejercicio}
    
\begin{ejercicio}
    Escribir las ecuaciones en diferencias de un interpolador de orden cero y otro de orden uno, en su versión discreta. 
    
    \inciso Implementar las ecuaciones en \Keyboardsym
    
    \inciso Interpolar la señal $x_e(n)$ definida como
    \begin{equation*}
        x_e(n) = \begin{cases}
        \cos(\frac{2\pi100}{L}) & \mbox{para $n=kL$, $k\in \mathbb{Z}$} \\
        0 & \mbox{en otro caso}
        \end{cases}
    \end{equation*}
    
    \inciso ¿Es posible implementar realmente una interpolación ideal? ¿Qué aproximaciones se podrían hacer para obtenerlo y cómo se altera el espectro de la señal al hacerlas?
\end{ejercicio}

\begin{ejercicio}
    Sea el siguiente sistema:
    \begin{center}
        \begin{tikzpicture}[scale=1, transform shape]
    \node[rectangle, very thick, draw, minimum width=2cm, minimum height=1cm] (ad) at (0,0) {\large A/D};
    \node[rectangle, very thick, draw, minimum width=2cm, minimum height=1cm, xshift=-4cm] (hc) at (ad) {$H_c(j\omega)$};
    \node[rectangle, very thick, draw, minimum width=2cm, minimum height=1cm, xshift=4cm] (h) at (ad) {$H(e^{j\Omega})$};

    \node[xshift=-2cm] (r_c) at (hc) {$r_c(t)$};
    \node[xshift=3cm] (y_n) at (h) {$y(n)$};
    \node[xshift=-2cm,yshift=0.4cm] (x_c) at (ad) {$x_c(t)$};
    \node[xshift=2cm,yshift=0.4cm] (x_n) at (ad) {$x(n)$};
    \node[yshift=-1.2cm,xshift=0.4cm] (t1) at (ad) {$T_1$};

    \draw[->, very thick] (hc.east) -- (ad.west);
    \draw[->, very thick] (ad.east) -- (h.west);
    \draw[->, very thick] (r_c.east) -- (hc.west);
    \draw[->, very thick] (h.east) -- (y_n.west);
    \draw[->, very thick] (ad.south) ++ (0,-1cm) -- (ad.south) ;
\end{tikzpicture}
    \end{center}
    donde se sabe que el sistema $H_c(j\omega)$ es LTI, causal y se describe mediante:
    \begin{equation*}
        \frac{dx_c(t)}{dt} + \alpha x_c(t) = r_c(t), \; \alpha > 0,\; \mbox{con condición de reposo incial}
    \end{equation*}
    Sea, además, la señal $r_c(t) = u(t) - u(t-T_1)$, donde $T_1$ es también el período de muestreo del conversor A/D ideal de la figura.

    \inciso Obtener la señal $x_c(t)$.

    \inciso Encontrar el sistema LTI de tiempo discreto $H(e^{j\Omega})$ tal que $y(n) = \delta(n) - 2\delta(n-1) + \delta(n-2)$. De ser posible, obtener una ecuación en diferencias para dicho sistema.
\end{ejercicio}

\begin{ejercicio}
    Sea el siguiente sistema:
    \begin{center}
        \begin{tikzpicture}[scale=1, transform shape]
    \node[rectangle, very thick, draw, minimum width=2cm, minimum height=1cm] (ad) at (0,0) {\large A/D};
    \node[rectangle, very thick, draw, minimum width=2cm, minimum height=1cm, xshift=-4cm] (hc) at (ad) {$H_c(j\omega)$};
    \node[rectangle, very thick, draw, minimum width=2cm, minimum height=1cm, xshift=4cm] (h) at (ad) {$H(e^{j\Omega})$};
    \node[rectangle, very thick, draw, minimum width=2cm, minimum height=1cm,xshift=4cm] (da) at (h) {\large D/A};

    \node[xshift=-2cm] (r_c) at (hc) {$r_c(t)$};
    \node[xshift=2cm] (y_c) at (da) {$y_c(t)$};
    \node[yshift=0.4cm,xshift=2cm] (y_n) at (h) {$y(n)$};
    \node[xshift=-2cm,yshift=0.4cm] (x_c) at (ad) {$x_c(t)$};
    \node[xshift=2cm,yshift=0.4cm] (x_n) at (ad) {$x(n)$};
    \node[yshift=-1.2cm,xshift=0.4cm] (t1) at (ad) {$T_1$};
    \node[yshift=-1.2cm,xshift=0.4cm] (t2) at (da) {$T_2$};

    \draw[->, very thick] (hc.east) -- (ad.west);
    \draw[->, very thick] (ad.east) -- (h.west);
    \draw[->, very thick] (r_c.east) -- (hc.west);
    \draw[->, very thick] (h.east) -- (da.west);
    \draw[->, very thick] (da.east) -- (y_c.west);
    \draw[->, very thick] (ad.south) ++ (0,-1cm) -- (ad.south) ;
    \draw[->, very thick] (da.south) ++ (0,-1cm) -- (da.south) ;
\end{tikzpicture}
    \end{center}
    donde $r_c(t) = \cos^2(\omega_0t)\cos^2(2\omega_0t)$ y los conversores A/D y D/A son ideales. El filtro $H_c(j\omega)$ es un filtro pasabajos ideal con frecuencia de corte $3\omega_0$. Además, $T_1=\frac{2\pi}{3\omega_0}$.

    \inciso Asumiendo que $T_1=T_2$ y que $H(e^{j\Omega})=1$ para todo $\Omega \in [-\pi,\pi]$, calcular la salida $y_c(t)$ del sistema y dibujar los espectros en frecuencias de cada una de las señales presentes en la figura.

    \inciso Diseñar un filtro $H(e^{j\Omega})$ y obtener un valor de $T_2$ apropiado para que la salida del sistema sea $y_c(t)=\cos(2\omega_0 t)$.
\end{ejercicio}

\begin{ejercicio}
    Sea la señal $z(t) = x(t) + y(t)$ donde
    \begin{align*}
        x(t) = \frac{\sin^2\left(\frac{W}{2}t\right)}{\pi^2t^2} & \hspace*{1em} & y(t) = 2\frac{\sin\left(\frac{W}{2}t\right)}{\pi t} \cos\left(\frac{5W}{2}t\right)
    \end{align*}
    La señal $z(t)$ ingresa a un conversor A/D ideal con frecuencia de muestreo $\omega_s$.

    \inciso Calcular la frecuencia de Nyquist para esta señal.

    \inciso Diseñar un sistema que permita, con la mínima frecuencia de muestreo posible, recuperar la señal $y(t)$ usando las muestras $z(n)$ que se obtienen a la salida del conversor A/D. El sistema puede contener partes en tiempo discreto y tiempo continuo.
\end{ejercicio}


% \begin{ejercicio}
%     Dados los siguientes sistemas en cascada
%     \begin{center}
%         \begin{tikzpicture}
    \node[rectangle,draw,thick,inner sep=0.5cm] (H_W) at (0,0) {$H(e^{j\Omega})$} ;
    \node[xshift=3cm,rectangle,draw,thick,inner sep=0.4cm] (down2) at (H_W) {$\downarrow N$} ;
    \node[xshift=-3cm,rectangle,draw,thick,inner sep=0.4cm] (up2) at (H_W) {$\uparrow N$} ;
    \node[xshift=-3cm] (x_n) at (up2) {$x(n)$} ;
    \node[xshift=3cm] (y_n) at (down2) {$y(n)$} ;
    
    \node[rectangle,minimum width=8cm,minimum height=3cm,dashed,thick,draw] (box) at (H_W) {};
    \node[yshift=.5cm] (box_label) at (box.north) {$H_{eff}(e^{j\Omega})$};
    
    \draw[->] (x_n) -- (up2) ;
    \draw[->] (up2) -- (H_W) ;
    \draw[->] (H_W) -- (down2) ;
    \draw[->] (down2) -- (y_n) ;
    \end{tikzpicture}
%     \end{center}
%     determinar la respuesta en frecuencia del sistema equivalente $H_{eff}(e^{j\Omega})$.
% \end{ejercicio}
   
\begin{ejercicio}
    Sea $x_c(t)$ una señal real de tiempo continuo cuya frecuencia superior es $\omega_M = 2\pi 250\mathrm\;{Hz}$, y la señal $y_c(t)$ definida como $y_c(t) = x_c(t - \frac{1}{1000})$.
    
    \inciso Determinar si es posible recuperar $x_c(t)$ a partir de $x(n)=x_c(\frac{n}{500})$.
    
    \inciso Determinar si es posible recuperar $y_c(t)$ a partir de $y(n)=y_c(\frac{n}{500})$.
    
    \inciso Determinar si es posible obtener un sistema $H(e^{j\Omega})$ de manera que si se implementa en la siguiente estructura en cascada
    \begin{center}
        \begin{tikzpicture}
    \node[rectangle,draw,thick,inner sep=0.5cm] (H_W) at (0,0) {$H(e^{j\Omega})$} ;
    \node[xshift=3cm,rectangle,draw,thick,inner sep=0.4cm] (down2) at (H_W) {$\downarrow 2$} ;
    \node[xshift=-3cm,rectangle,draw,thick,inner sep=0.4cm] (up2) at (H_W) {$\uparrow 2$} ;
    \node[xshift=-3cm] (x_n) at (up2) {$x(n)$} ;
    \node[xshift=3cm] (y_n) at (down2) {$y(n)$} ;
    
    % \node[rectangle,minimum width=8cm,minimum height=3cm,dashed,thick,draw] (box) at (H_W) {};
    % \node[yshift=.5cm] (box_label) at (box.north) {$H_{eff}(e^{j\Omega})$};
    
    \draw[->] (x_n) -- (up2) ;
    \draw[->] (up2) -- (H_W) ;
    \draw[->] (H_W) -- (down2) ;
    \draw[->] (down2) -- (y_n) ;
\end{tikzpicture}
    \end{center}
    es posible obtener $y(n)$ a partir de $x(n)$.
    
    \inciso Determinar si es posible obtener $y(n)$ a partir de $x(n)$ utilizando un único sistema LTI con respuesta en frecuencia $H_{eff}(e^{j\Omega})$. En caso de que así sea, obtener $H_{eff}(e^{j\Omega})$.
    
\end{ejercicio}

\begin{ejercicio}
    Sea la señal
    \begin{equation*}
        x_c(t) = \sum_{k=-\infty}^{\infty} \alpha(k) g(t - kT_0)
    \end{equation*}
    donde $\alpha(k)$ es una secuencia de tiempo discreto tal que $\sum_{k=-\infty}^{\infty} |\alpha(k)| < \infty$ y $g(t)$ es una señal de banda limitada con ancho de banda $W = \frac{2 \pi}{T_0}$. La señal $x_c(t)$ ingresa al siguiente sistema:
    \begin{center}
        \begin{tikzpicture}[scale=1, transform shape]
    \node[rectangle, draw, very thick, minimum width=1cm, minimum height=1cm] (h) at (0,0) {$H(e^{j\Omega})$} ;
    \node[rectangle, draw, very thick, minimum width=1cm, minimum height=1cm, xshift=-2.5cm] (ad) at (h) {\large A/D} ;
    \node[xshift=-2cm] (x_c) at (ad) {$x_c(t)$} ;
    \node[xshift=-1.3cm,yshift=0.4cm] (x_n) at (h) {$x(n)$} ;
    \node[xshift=2cm] (y_n) at (h) {$y(n)$} ;
    \node[yshift=-1.5cm] (t) at (ad) {} ;
    \node[yshift=-1.1cm,xshift=0.4cm] (t_label) at (ad) {$T$} ;

    \draw[->, very thick] (ad.east) -- (h.west) ;
    \draw[->, very thick] (h.east) -- (y_n) ;
    \draw[->, very thick] (x_c) -- (ad.west) ;
    \draw[->, very thick] (t) -- (ad.south) ;
\end{tikzpicture}
    \end{center}
    en donde el conversor A/D tiene frecuencia de muestreo igual a la de Nyquist y el sistema $H(e^{j\Omega})$ es un filtro LTI de tiempo discreto.

    \inciso Justificar que $x_c(t)$ es de banda limitada y determinar $Y(e^{j\Omega})$.

    \inciso Asumiendo que $H(e^{j\Omega}) = \frac{T}{G(j\frac{\Omega}{T})}$ con $T$ el período de muestreo, ¿es posible recuperar la secuencia $\alpha(k)$ a partir de la salida $y(n)$? En caso afirmativo, diseñar un sistema de tiempo discreto que permita recuperar la señal mencionada.
\end{ejercicio}

\begin{ejercicio}
    Sea el siguiente sistema:
    \begin{center}
        \begin{tikzpicture}[scale=1, transform shape]
    \node[rectangle, very thick, draw, minimum width=1cm, minimum height=1cm] (ad) at (0,0) {\large A/D};
    \node[rectangle, very thick, draw, minimum width=1cm, minimum height=1cm, xshift=2.5cm] (up) at (ad) {\large $\uparrow 3$};
    \node[rectangle, very thick, draw, minimum width=1cm, minimum height=1cm, xshift=2.5cm] (h) at (up) {$H(e^{j\Omega})$};
    \node[rectangle, very thick, draw, minimum width=1cm, minimum height=1cm, xshift=2.5cm] (down) at (h) {\large $\downarrow 2$};
    \node[rectangle, very thick, draw, minimum width=1cm, minimum height=1cm, xshift=2.5cm] (da) at (down) {\large D/A};

    \node[xshift=-2cm] (r_c) at (ad) {$r_c(t)$};
    \node[xshift=1.25cm,yshift=0.4cm] (r_n) at (ad) {$r(n)$};
    \node[xshift=1.1cm,yshift=0.4cm] (x_n) at (up) {$x(n)$};
    \node[xshift=1.4cm,yshift=0.4cm] (y_n) at (h) {$y(n)$};
    \node[xshift=1.2cm,yshift=0.4cm] (s_n) at (down) {$s(n)$};
    \node[xshift=2cm] (s_c) at (da) {$s_c(t)$};
    
    \node[yshift=-1.2cm,xshift=0.4cm] (t1) at (ad) {$T_1$};
    \node[yshift=-1.2cm,xshift=0.4cm] (t2) at (da) {$T_3$};

    \draw[->, very thick] (r_c.east) -- (ad.west);
    \draw[->, very thick] (ad.east) -- (up.west);
    \draw[->, very thick] (up.east) -- (h.west);
    \draw[->, very thick] (h.east) -- (down.west);
    \draw[->, very thick] (down.east) -- (da.west);
    \draw[->, very thick] (da.east) -- (s_c.west);
    \draw[->, very thick] (ad.south) ++ (0,-1cm) -- (ad.south) ;
    \draw[->, very thick] (da.south) ++ (0,-1cm) -- (da.south) ;
\end{tikzpicture}
    \end{center}
    donde 
    \begin{align*}
        \parbox{0.5\textwidth}{
                \begin{tikzpicture}[scale=1,transform shape]
    \begin{axis}[
        x=0.05\textwidth,y=0.05\textwidth,
        axis y line=center,
        axis x line=middle,
        xlabel=$\Omega$,ylabel={\large $R_c({j\omega})$},
        xmin=-2.4,xmax=2.9,
        ymin=-0.8,ymax=2.4,
        xtick = {-1, 0, 1},
        xticklabels = {$-W$, $0$, $W$},
        ytick = {0, 1},
        yticklabels = {$0$,$1$},
        yticklabel style={yshift=0.3cm},
        ]
        \addplot[
        black,
        ultra thick
        ] coordinates {
            (-1,0) (0,1) (1,0)
        } ;
    \end{axis}
\end{tikzpicture}
        } & H(e^{e^{j\Omega}}) = \begin{cases}
            1 & |\Omega| \leq \Omega_0 \\
            0 & \text{en otro caso}
            \end{cases}
    \end{align*}
    y los conversores A/D y D/A son ideales. 

    \inciso Obtener $R(e^{j\Omega})$ y $X(e^{j\Omega})$.

    \inciso Determinar $\Omega_0$, $T_2$ y $\alpha$ tales que $y(n) = \alpha r_c(nT_2)$.

    \inciso Con el valor de $\Omega_0$ hallado en el punto anterior, obtener $T_3$ y $\beta$ tales que $s_c(t) = \beta r_c(t)$.
\end{ejercicio}

\begin{ejercicio}
    Sea el sistema de la figura:
    \begin{center}
        \begin{tikzpicture}[scale=1, transform shape]
    \node[rectangle, very thick, draw, minimum width=1cm, minimum height=1cm] (ad) at (0,0) {\large A/D};
    \node[rectangle, very thick, draw, minimum width=1cm, minimum height=1cm, xshift=2.5cm] (up2) at (ad) {\large $\uparrow 2$};
    \node[rectangle, very thick, draw, minimum width=1cm, minimum height=1cm, xshift=2.5cm] (h) at (up2) {$H(e^{j\Omega})$};
    \node[circle, very thick, draw, inner sep=0.05cm, xshift=2.5cm] (mult) at (h) {$\times$};
    \node[rectangle, very thick, draw, minimum width=1cm, minimum height=1cm, xshift=2.5cm] (da) at (mult) {\large D/A};

    \node[xshift=-2cm] (x_c) at (ad) {$x(t)$};
    \node[xshift=1.25cm,yshift=0.4cm] (x_n) at (ad) {$x(n)$};
    \node[xshift=1.1cm,yshift=0.4cm] (x_1) at (up2) {$x_1(n)$};
    \node[xshift=1.4cm,yshift=0.4cm] (x_2) at (h) {$x_2(n)$};
    \node[xshift=1cm,yshift=0.4cm] (y_n) at (mult) {$y(n)$};
    \node[xshift=2cm] (y_c) at (da) {$y(t)$};
    
    \node[yshift=-1.7cm] (cos) at (mult) {$\cos(\pi n)$};
    \node[yshift=-1.2cm,xshift=0.4cm] (t1) at (ad) {$T_1$};
    \node[yshift=-1.2cm,xshift=0.4cm] (t2) at (da) {$\frac{T}{2}$};

    \draw[->, very thick] (x_c.east) -- (ad.west);
    \draw[->, very thick] (ad.east) -- (up2.west);
    \draw[->, very thick] (up2.east) -- (h.west);
    \draw[->, very thick] (h.east) -- (mult.west);
    \draw[->, very thick] (mult.east) -- (da.west);
    \draw[->, very thick] (da.east) -- (y_c.west);
    \draw[->, very thick] (mult.south) ++ (0,-1cm) -- (mult.south) ;
    \draw[->, very thick] (ad.south) ++ (0,-1cm) -- (ad.south) ;
    \draw[->, very thick] (da.south) ++ (0,-1cm) -- (da.south) ;
\end{tikzpicture}
    \end{center}
    Tanto el conversor A/D como el D/A son ideales ($T = \frac{4}{3}\;\mathrm{s}$). Por otro lado, el espectro en frecuencias de $x(t)$ y la respuesta en frecuencia del sistema $H(e^{j\Omega})$ son los siguientes:
    \begin{align*}
        \parbox{0.5\textwidth}{
            \begin{center}
                \begin{tikzpicture}[scale=1,transform shape]
    \begin{axis}[
        x=0.05\textwidth,y=0.05\textwidth,
        axis y line=center,
        axis x line=middle,
        xlabel=$\Omega$,ylabel={\large $X(e^{j\Omega})$},
        xmin=-2.4,xmax=2.9,
        ymin=-0.8,ymax=2.4,
        xtick = {-2, -1, 0, 1, 2},
        xticklabels = {$-2\pi$, $-\pi$, $0$, $\pi$, $2\pi$},
        ytick = {0, 1},
        yticklabels = {$0$,$1$},
        yticklabel style={yshift=0.3cm},
        ]
        \addplot[
        black,
        ultra thick
        ] coordinates {
            (-1,0) (0,1) (1,0)
        } ;
    \end{axis}
\end{tikzpicture}
            \end{center}
        } & \hspace*{1em} & \parbox{0.5\textwidth}{
            \begin{center}
                \begin{tikzpicture}[scale=1,transform shape]
    \begin{axis}[
        x=0.05\textwidth,y=0.05\textwidth,
        axis y line=center,
        axis x line=middle,
        xlabel=$\Omega$,ylabel={\large $H(e^{j\Omega})$},
        xmin=-4.4,xmax=4.9,
        ymin=-0.8,ymax=2.4,
        xtick = {-4, -3, -2, -1, 0, 1, 2, 3, 4},
        xticklabels = {-$\pi$, $-\frac{3\pi}{4}$, $-\frac{\pi}{2}$, $-\frac{\pi}{4}$, $0$, $\frac{\pi}{4}$, $\frac{\pi}{2}$, $\frac{3\pi}{4}$, $\pi$},
        ytick = {0, 1},
        yticklabels = {$0$,$1$},
        yticklabel style={yshift=0.3cm},
        ]

        \addplot[
        black,
        ultra thick
        ] coordinates {
            (-2,0) (-2,1) (2,1) (2,0)
        } ;
    \end{axis}
\end{tikzpicture}
            \end{center}
        }
    \end{align*}        
    Dibujar los espectros de $x(n)$, $x_1(n)$, $x_2(n)$, $y(n)$ y $y(t)$.
\end{ejercicio}

    \or \myheader{Guía 7: Transformada Discreta de Fourier}

\begin{ejercicio}
\end{ejercicio}

\begin{ejercicio}
\end{ejercicio}

\begin{ejercicio}
\end{ejercicio}

\begin{ejercicio}
\end{ejercicio}

\begin{ejercicio}
\end{ejercicio}

\begin{ejercicio}
\end{ejercicio}

\begin{ejercicio}
\end{ejercicio}

\begin{ejercicio}
\end{ejercicio}

\begin{ejercicio}
\end{ejercicio}

\begin{ejercicio}
\end{ejercicio}

\begin{ejercicio}
\end{ejercicio}

\begin{ejercicio}
\end{ejercicio}

\begin{ejercicio}
\end{ejercicio}

\begin{ejercicio}
\end{ejercicio}

\begin{ejercicio}
\end{ejercicio}

\begin{ejercicio}
\end{ejercicio}

\begin{ejercicio}
\end{ejercicio}

\begin{ejercicio}
\end{ejercicio}

\begin{ejercicio}
\end{ejercicio}

\begin{ejercicio}
\end{ejercicio}

\begin{ejercicio}
\end{ejercicio}

\begin{ejercicio}
\end{ejercicio}

\begin{ejercicio}
\end{ejercicio}

\begin{ejercicio}
\end{ejercicio}

\begin{ejercicio}
\end{ejercicio}

\begin{ejercicio}
\end{ejercicio}

\begin{ejercicio}
\end{ejercicio}

\begin{ejercicio}
\end{ejercicio}

\begin{ejercicio}
\end{ejercicio}

\begin{ejercicio}
\end{ejercicio}
    \or \myheader{Guía 8: Transformada de Laplace y Z, Ecuaciones Diferenciales y en Diferencias (Parte II)}

\begin{ejercicio}
\end{ejercicio}

\begin{ejercicio}
\end{ejercicio}

\begin{ejercicio}
\end{ejercicio}

\begin{ejercicio}
\end{ejercicio}

\begin{ejercicio}
\end{ejercicio}

\begin{ejercicio}
\end{ejercicio}

\begin{ejercicio}
\end{ejercicio}

\begin{ejercicio}
\end{ejercicio}

\begin{ejercicio}
\end{ejercicio}

\begin{ejercicio}
\end{ejercicio}

\begin{ejercicio}
\end{ejercicio}

\begin{ejercicio}
\end{ejercicio}

\begin{ejercicio}
\end{ejercicio}

\begin{ejercicio}
\end{ejercicio}

\begin{ejercicio}
\end{ejercicio}

\begin{ejercicio}
\end{ejercicio}

\begin{ejercicio}
\end{ejercicio}

\begin{ejercicio}
\end{ejercicio}

\begin{ejercicio}
\end{ejercicio}

\begin{ejercicio}
\end{ejercicio}

\begin{ejercicio}
\end{ejercicio}

\begin{ejercicio}
\end{ejercicio}

\begin{ejercicio}
\end{ejercicio}

\begin{ejercicio}
\end{ejercicio}

\begin{ejercicio}
\end{ejercicio}

\begin{ejercicio}
\end{ejercicio}

\begin{ejercicio}
\end{ejercicio}

\begin{ejercicio}
\end{ejercicio}

\begin{ejercicio}
\end{ejercicio}

\begin{ejercicio}
\end{ejercicio}

    \or \myheader{Guía 9: Filtros Digitales, Sistemas de Fase Mínima y Pasa-Todo, Cuantificación}


\begin{ejercicio}
\end{ejercicio}

\begin{ejercicio}
\end{ejercicio}

\begin{ejercicio}
\end{ejercicio}

\begin{ejercicio}
\end{ejercicio}

\begin{ejercicio}
\end{ejercicio}

\begin{ejercicio}
\end{ejercicio}

\begin{ejercicio}
\end{ejercicio}

\begin{ejercicio}
\end{ejercicio}

\begin{ejercicio}
\end{ejercicio}

\begin{ejercicio}
\end{ejercicio}

\begin{ejercicio}
\end{ejercicio}

\begin{ejercicio}
\end{ejercicio}

\begin{ejercicio}
\end{ejercicio}

\begin{ejercicio}
\end{ejercicio}

\begin{ejercicio}
\end{ejercicio}

\begin{ejercicio}
\end{ejercicio}

\begin{ejercicio}
\end{ejercicio}

\begin{ejercicio}
\end{ejercicio}

\begin{ejercicio}
\end{ejercicio}

\begin{ejercicio}
\end{ejercicio}

\begin{ejercicio}
\end{ejercicio}

\begin{ejercicio}
\end{ejercicio}

\begin{ejercicio}
\end{ejercicio}

\begin{ejercicio}
\end{ejercicio}

\begin{ejercicio}
\end{ejercicio}

\begin{ejercicio}
\end{ejercicio}

\begin{ejercicio}
\end{ejercicio}

\begin{ejercicio}
\end{ejercicio}

\begin{ejercicio}
\end{ejercicio}

\begin{ejercicio}
\end{ejercicio}
    \else UNIDAD TEMÁTICA NO SELECCIONADA!!
\fi


\end{document}